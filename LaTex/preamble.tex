\documentclass[a4paper]{book}
\usepackage{verbatim}
\usepackage[utf8]{inputenc}
\usepackage[italian]{babel}
\usepackage{amsmath}
\usepackage{mathtools}
\usepackage{amsbsy,amssymb,amsfonts, amsthm}
\usepackage{floatrow}
\usepackage[version=4]{mhchem}
\usepackage[export]{adjustbox}
\usepackage{graphicx}
\usepackage[includemp,
	    paperwidth=18.90cm,
	    paperheight=24.58cm,
	    head=100.0pt,
	    top=2.170cm,
	    bottom=3.510cm,
	    inner=2.1835cm,
	    outer=2.1835cm,
	    footskip=2.1cm,
	    marginparwidth=4cm,
	    marginparsep=0.4cm]{geometry}
\usepackage{lipsum}  
\usepackage[skip=1pt]{caption}  
\usepackage{marginfix}  
\usepackage{scrlayer-scrpage}  
\usepackage{xcolor}
\usepackage[hypertexnames=false]{hyperref}
\usepackage{nameref}
\usepackage{tikz}
\usetikzlibrary{math}
\usetikzlibrary{shapes,arrows}
\usepackage{pgfplots}
\pgfplotsset{
    compat=1.16
}
\usepackage{sidenotes}

\extrafloats{100}
\newcounter{Sec}
\renewcommand*\thesection{\arabic{section}}

\definecolor{mygreen}{rgb}{0.0, 0.4, 0.}

% figure support
\usepackage{import}
\usepackage{xifthen}
\pdfminorversion=7
\usepackage{pdfpages}
\usepackage{transparent}
\newcommand{\incfig}[1]{%
    \def\svgwidth{0.95\marginparwidth}
    \fbox{\import{./figures/}{#1.pdf_tex}}
    \vspace*{-8pt}
}

\newcommand{\mygeo}{%
    \newgeometry{top=2.170cm,
	bottom=3.510cm,
	inner=2.1835cm,
	outer=2.1835cm,
    	ignoremp}
}


\newtheoremstyle{break}%
	{1em}{1em}%
	{}{}%
	{\bfseries}{}% % Note that final punctuation is omitted.
	{\newline}
	{#1 #2: \normalfont(\textcolor{mygreen}{#3})}

\newtheoremstyle{defn}
	{\topsep}   % ABOVESPACE
	{\topsep}   % BELOWSPACE
	{\itshape}  % BODYFONT
	{0pt}       % INDENT (empty value is the same as 0pt)
	{\bfseries} % HEADFONT
	{.}         % HEADPUNCT
	{5pt plus 1pt minus 1pt} % HEADSPACE
	{#1 #2: \normalfont(\textcolor{red}{#3})}          % CUSTOM-HEAD-SPEC

\newtheoremstyle{thm}% name of the style to be used
	{\topsep}% measure of space to leave above the theorem. E.g.: 3pt
	{\topsep}% measure of space to leave below the theorem. E.g.: 3pt
	{\itshape}% name of font to use in the body of the theorem
	{0pt}% measure of space to indent
	{\bfseries}% name of head font
	{.}% punctuation between head and body
	{ }% space after theorem head; " " = normal interword space
	{#1 #2: \normalfont(\textcolor{red}{\underline{#3}})}

\theoremstyle{thm}
\newtheorem{thm}{Teorema}[subsection] % reset theorem numbering for each chapter

\theoremstyle{defn}
\newtheorem{defn}{Definizione}[subsection] % definition numbers are dependent on theorem numbers

\theoremstyle{break}
\newtheorem{exmp}{Esempio}[subsection]
\newtheorem{ex}{Esercizio}[subsection] % definition numbers are dependent on theorem numbers

\newcommand{\ffrac}[2]{\ensuremath{\frac{\displaystyle #1}{\displaystyle #2}}}
\newcommand{\angstrom}{\mbox{\normalfont\AA}}
\newcommand{\vect}[1]{\boldsymbol{#1}}
%\renewcommand{\[}{\begin{equation}}
%\renewcommand{\]}{\end{equation}}
\renewcommand{\theequation}{\thesection.\arabic{equation}}
\counterwithin*{equation}{section}
\newcommand{\rom}[1]{\uppercase\expandafter{\romannumeral #1\relax}}



\newlength{\overflowingheadlen}
\setlength{\overflowingheadlen}{\linewidth}
\addtolength{\overflowingheadlen}{\marginparsep}
\addtolength{\overflowingheadlen}{\marginparwidth}

\newlength{\mynewlength}
\setlength{\mynewlength}{\linewidth}
\addtolength{\mynewlength}{\marginparsep}
\setlength{\mynewlength}{\marginparwidth}

\renewpagestyle{scrheadings}{
    {\hspace{-\marginparwidth}\hspace{-\marginparsep}\makebox[\overflowingheadlen][l]{\makebox[2em][r]{\thepage}\quad\rule{1pt}{100pt}\quad{}SEZIONE \rightmark}}%
    {\makebox[\overflowingheadlen][r]{SEZIONE \rightmark\quad\rule{1pt}{100pt}\quad\makebox[2em][l]{\thepage}}}%
    {}
}{
    {\hspace{2.03\marginparwidth}\makebox[\mynewlength][r]{\leftmark\quad\rule[-40pt]{1.3pt}{50pt}\quad\makebox[2em][l]{\thepage}}}%
    {\hspace{0.13\marginparwidth}\makebox[\mynewlength][l]{\thepage\quad\rule[-40pt]{1.3pt}{50pt}\quad\makebox[2em][l]{\leftmark}}}%
    {}
}
\renewpagestyle{plain.scrheadings}{
    {}%
    {}%
    {}
}{
    {}%
    {}%
    {}
}


\floatsetup[figure]{margins=hangoutside,
    facing=yes,
    capposition=beside,
    capbesideposition={center,outside},
floatwidth=\textwidth}
\floatsetup[table]{margins=hangoutside,
    facing=yes,
    capposition=beside,
    capbesideposition={center,outside},
floatwidth=\textwidth}

\setcounter{tocdepth}{1}
