\documentclass[a4paper]{book}
\usepackage{float}
\usepackage{verbatim}
\usepackage[utf8]{inputenc}
\usepackage[italian]{babel}
\usepackage{amsmath}
\usepackage{mathtools}
\usepackage{amsbsy,amssymb,amsfonts, amsthm}
\usepackage[version=4]{mhchem}
\usepackage{graphicx}
\usepackage[includemp,
	    paperwidth=18.90cm,
	    paperheight=24.58cm,
	    top=2.170cm,
	    bottom=3.510cm,
	    inner=2.1835cm,
	    outer=2.1835cm,
	    marginparwidth=4cm,
	    marginparsep=0.4cm]{geometry}
\usepackage{lipsum}  
\usepackage{caption}  
\usepackage{marginfix}  
\usepackage{scrlayer-scrpage}  
\usepackage{xcolor}
\usepackage[hypertexnames=false]{hyperref}
\usepackage{nameref}
%\usepackage{framed}
%\usepackage[framemethod=TikZ]{mdframed}
%\usepackage{appendix}
\usepackage{tikz}
\usetikzlibrary{math}
\usetikzlibrary{shapes,arrows}
\usepackage{pgfplots}
%\usepackage{tcolorbox}
%\usepackage{listings}
%\usepackage{witharrows}
\usepackage{circuitikz}
\pgfplotsset{
    compat=1.16
}
\usepackage[footnote]{snotez}
\setsidenotes{text-mark-format=\textsuperscript{\normalfont #1},
                  note-mark-format=#1:,
		  note-mark-sep=\enskip}

\extrafloats{100}

\newcounter{Sec}

\renewcommand*\thesection{\arabic{section}}
%
% \lstset{
%       basicstyle=\ttfamily,
%         columns=fullflexible,
% 	frame=single,
%       	breaklines=true,
%       	postbreak=\mbox{\textcolor{red}{$\hookrightarrow$}\space},
% }

% figure support
\usepackage{import}
\usepackage{xifthen}
\pdfminorversion=7
\usepackage{pdfpages}
\usepackage{transparent}
\newcommand{\incfig}[1]{%
    \def\svgwidth{\columnwidth}
    \import{./figures/}{#1.pdf_tex}
}

\newcommand{\mygeo}{%
    \newgeometry{top=2.170cm,
	bottom=3.510cm,
	inner=2.1835cm,
	outer=2.1835cm,
    	ignoremp}
}

\theoremstyle{plain}
\newtheorem{thm}{Theorem}[subsection] % reset theorem numbering for each chapter

\theoremstyle{definition}
\newtheorem{defn}{Definizione}[subsection] % definition numbers are dependent on theorem numbers
\newtheorem{exmp}{Esempio}[subsection]
\newtheorem{ex}{Esercizio}[subsection] % definition numbers are dependent on theorem numbers

\newcommand{\ffrac}[2]{\ensuremath{\frac{\displaystyle #1}{\displaystyle #2}}}
\newcommand{\angstrom}{\mbox{\normalfont\AA}}
\newcommand{\vect}[1]{\boldsymbol{#1}}
%\renewcommand{\[}{\begin{equation}}
%\renewcommand{\]}{\end{equation}}
\renewcommand{\theequation}{\thesection.\arabic{equation}}
\counterwithin*{equation}{section}
\newcommand{\rom}[1]{\uppercase\expandafter{\romannumeral #1\relax}}



\newlength{\overflowingheadlen}
\setlength{\overflowingheadlen}{\linewidth}
\addtolength{\overflowingheadlen}{\marginparsep}
\addtolength{\overflowingheadlen}{\marginparwidth}

\renewpagestyle{scrheadings}{
    {\hspace{-\marginparwidth}\hspace{-\marginparsep}\makebox[\overflowingheadlen][l]{\makebox[2em][r]{\thepage}\quad\rule{1pt}{100pt}\quad{}\leftmark}}%
    {\makebox[\overflowingheadlen][r]{\rightmark\quad\rule{1pt}{100pt}\quad\makebox[2em][l]{\thepage}}}%
    {}
}{
    {}%
    {}%
    {}
}
\renewpagestyle{plain.scrheadings}{
    {}%
    {}%
    {}
}{
    {\thepage}%
    {}%
    {}
}


