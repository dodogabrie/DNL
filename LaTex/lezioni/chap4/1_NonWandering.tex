\section{Stati non-Wandering ed $\omega$ limit set}%
\subsection{Stati non-Wandering}%
\marginpar{
        \captionsetup{type=figure}
        \incfig{4_1_1}
        \caption{\scriptsize $\v{P}$ è uno stato non-Wandering: preso un punto in $U$ la sua dinamica si intersecherà nuovamente con $U$}
    \label{fig:4_1_1}
    }
Ricordiamo dapprima la definizione di un insieme $A$ invariante rispetto alla dinamica indotta dal flusso di fase $\varphi_t$:
\[
    A \subset \mathbb{R}^n \implies  \varphi_t(A) \subset A
.\] 
\begin{defn}[Stati non-Wandering per SD a tempo continuo]
Preso un campo vettoriale con il flusso di fase $\varphi_t:\mathbb{R}^n\to \mathbb{R}^n$ diciamo che $\v{P} \in \mathbb{R}^n$ è non-Wandering se $\forall U$ intorno di $\v{P}$ e $\forall T > 0$ allora $\exists t: \left|t\right|>T$ tale per cui
\[
    \varphi_t(U) \cap U \neq 0 
.\] 
\end{defn}
\noindent
\begin{defn}[Stati non-Wandering per SD a tempo discreto]
    Data la mappa ricorsiva
    \[
	\v{x}_{k+1} = G(\v{x}_k) \quad  \v{x}_k \in \mathbb{R}^n
    .\] 
    Diciamo che $\v{P}$ è non-Wandering se $\forall U$ intorno di $\v{P}$ esiste $n\in \mathbb{Z}$ tale per cui
    \[
	G^n(U)  \cap U \neq 0
    .\] 
\end{defn}
\noindent
\begin{defn}[Non-Wanderning set]
    L'insieme degli stati non-Wandering è denominato non-Wandering set.
\end{defn}
\noindent
\begin{defn}[Stati Wandering]
    Uno stato si dice Wandering se non è non-Wandering (si ha quindi anche lo Wandering set).
\end{defn}
\noindent
\begin{exmp}[Sella iperbolica]
    Prendiamo uno stato sella $\v{P}$ e consideriamone il manifold locale stabile $W_{loc}^s(\v{P})$ e instabile $W_{loc}^u(\v{P})$. Questo stato stazionario è non-Wandering: in ogni intorno del punto $\v{P}$ valgono le proprietà descritte sopra.
\end{exmp}
\noindent
\begin{exmp}[Orbite periodiche]
    Le orbite periodiche sono non-Wandering, questo perchè l'orbita torna proprio nello stato per definizione\ldots
\end{exmp}
\noindent
Intuitivamente gli stati Wandering hanno a che vedere con la dinamica transiente del sistema, viceversa quelli non-Wandering hanno ha che vedere con la dinamica asintotica del sistema.
\subsection{$\omega (\alpha)$ limit set}%
La nozione di $\omega$-limit set ha a che vedere con il comportamento asintotico delle orbite del sistema: caratterizza il comportamento asintotico di un SD sia nel futuro ($\omega$) che nel passato ($\alpha$). 
\begin{defn}[$\omega$-limit state di un campo vettoriale]
    Dato il sistema dinamico $\frac{\text{d} \v{x}}{\text{d} t} = F(\v{x})$, $\v{x}\in \mathbb{R}^n$ con il flusso di fase $\varphi_t:\mathbb{R}^n\to \mathbb{R}^n$. Diciamo che $\v{x}_0\in \mathbb{R}^n$ è un $\omega$-limit state di $\v{x}\in \mathbb{R}^n$ ($\v{x}_0 \in \omega (\v{x})$) se esiste una sequenza $t_i$:
    \[
        t_i \to \infty \quad  i \to \infty
    .\] 
    Tale per cui:
    \[
	\lim_{t_i \to \infty} \varphi_{t_i}(\v{x}) = \v{x}_0
    .\] 
\end{defn}
\noindent
Perché utilizzare una collezione discreta di tempi? Perché vedremo che in alcuni sistemi prendendo un limite continuo non si ha una buona definizione di questo insieme (come vedremo).
\begin{defn}[$\omega$-limit state di una mappa ricorsiva]
    Data la mappa ricorsiva $\v{x}_{k+1}=G(\v{x}_k)$ con $G:\mathbb{R}^n\to \mathbb{R}^n$ e $\v{x}_k \in \mathbb{R}^n$. Diciamo che $\v{x}_0 \in \mathbb{R}^n$ è un $\omega$-limit state di $\v{x}$ se esiste una sequenza $\left\{n_i|i=1, 2, \ldots, \infty \ ; \ n_i \to \infty\right\}$ tale per cui
    \[
	\lim_{n_i \to \infty} G^{n_i}(\v{x}) = \v{x}_0
    .\] 
\end{defn}
\noindent
\begin{defn}[$\alpha$-limit state per un campo vettoriale]
    Dato il sistema dinamico $\frac{\text{d} \v{x}}{\text{d} t} = F(\v{x})$, $\v{x}\in \mathbb{R}^n$ con il flusso di fase $\varphi_t:\mathbb{R}^n\to \mathbb{R}^n$. Diciamo che $\v{x}_0\in \mathbb{R}^n$ è un $\alpha$-limit state di $\v{x}\in \mathbb{R}^n$ ($\v{x}_0 \in \alpha (\v{x})$) se esiste una sequenza $t_i$:
    \[
        t_i \to -\infty \quad  i \to \infty
    .\] 
    Tale per cui:
    \[
	\lim_{t_i \to -\infty} \varphi_{t_i}(\v{x}) = \v{x}_0
    .\] 
\end{defn}
\noindent 
\begin{defn}[$\alpha$-limit state di una mappa ricorsiva]
    Data la mappa ricorsiva $\v{x}_{k+1}=G(\v{x}_k)$ con $G:\mathbb{R}^n\to \mathbb{R}^n$ e $\v{x}_k \in \mathbb{R}^n$. Diciamo che $\v{x}_0 \in \mathbb{R}^n$ è un $\alpha$-limit state di $\v{x}$ se esiste una sequenza $\left\{n_i|i=1, 2, \ldots \infty \ ; \ n_i \to -\infty\right\}$ tale per cui
    \[
	\lim_{n_i \to -\infty} G^{n_i}(\v{x}) = \v{x}_0
    .\] 
\end{defn}
\noindent
\begin{defn}[$\omega$($\alpha$)-limit set]
    L'insieme di tutti gli $\omega$($\alpha$)-limit states è denominato $\omega$($\alpha$)-limit set.
\end{defn}
\noindent
\subsection{Proprietà degli  $\omega$($\alpha$)-limit set}%
Consideriamo un campo vettoriale 
\[
    \frac{\text{d} \v{x}}{\text{d} t} = F(\v{x}) \quad  \v{x}\in \mathbb{R}^n \quad  F: \mathbb{R}^n\to \mathbb{R}^n
.\] 
con associato il campo vettoriale $\varphi_t:\mathbb{R}^n\to \mathbb{R}^n$. Sia inoltre $A \subset \mathbb{R}^n$ un insieme compatto e positivamente invariante rispetto alla dinamica:
\[
    \varphi_t(A) \subset A
.\] 
Formuliamo il teorema che caratterizza la struttura dei set asintotici e quindi l'intera dinamica asintotica.
\begin{thm}[Proprietà del $\omega$-limit set]
    Sia dato un campo vettoriale 
    \[
	\frac{\text{d} \v{x}}{\text{d} t} = F(\v{x}) \quad  \v{x}\in \mathbb{R}^n \quad  F:\mathbb{R}^n\to \mathbb{R}^n \quad  F \in C^r, \ r\ge 1
    .\] 
    ed $A \in \mathbb{R}^n$ sia compatto e positivamente invariante. Allora $\forall \v{P} \in A$ l'evoluzione asintotica di $\v{P}$ (che va in $\omega (\v{P})$) si ha:
    \begin{enumerate}
	\item $\omega (\v{P}) \neq 0$.
	\item $\omega (\v{P})$ è un insieme chiuso.
	\item $\omega (\v{P})$ è invariante rispetto alla dinamica.
	\item $\omega (\v{P})$ è connesso.
    \end{enumerate}
\end{thm}
\noindent 
\begin{proof}
Partiamo da 1.\\
Dato un compatto $A$ grazie al teorema di Bolzano-Weierstrass sappiamo che:
\begin{center}
    Data una sequenza di eventi $\in A$ allora ogni sottosuccessione è convergente ad un elemento di $A$.
\end{center}
Per ipotesi possiamo prendere una sequenza di tempi tale che $t_i\to \infty$ quando $i\to \infty$. Per ipotesi scegliamo la sequenza $\varphi_{t_i}(\v{P}) \in A$ (per ipotesi di positivamente invariate).
\[
    \varphi_{t_1}(\v{P}) \quad  \varphi_{t_2}(\v{P}) \quad \ldots \quad \varphi_{t_n}(\v{P}) 
.\] 
Da questa possiamo estrarre una sottosuccessione di $\overline{t}_i: \ \varphi_{\overline{t}_i}$ con $\left\{\left\{\overline{t}_i\right\} \subset \left\{t_J\right\}\right\}$. Per il teorema BW si ha:
\[
    \lim_{i \to \infty} \varphi_{\overline{t}_i}(\v{P}) = \omega_1 \in A \implies  \omega (\v{P}) \neq 0
.\] 
Per quanto riguarda la 2. invece:\\
Dobbiamo dimostrare che, preso $\v{s}\notin \omega(\v{P})$ e $\v{s}\in A$, si ha che tutto un intorno aperto di $\v{s}$ (che chiamiamo $M$) non appartiene a $\omega (\v{P})$. 
\end{proof}
\begin{exmp}[Perché prendere una sequenza discreta]
    Prendiamo un sistema dinamico in $\mathbb{R}^2$ con un'orbita periodica (un ciclo limite).
   \marginpar{
            \captionsetup{type=figure}
            \incfig{4_1_2}
            \caption{\scriptsize Orbita periodica per il sistema in $\mathbb{R}^2$}
        \label{fig:4_1_2}
        }
    Su tale orbita scegliamo un punto $\v{x}_0$ ed consideriamo inoltre un punto $\v{x}$ esterno all'orbita. \\
    Se facciamo evolvere $\v{x}$ (pensata come condizione iniziale) allora abbiamo l'evoluzione $\varphi_t(\v{x})$ $(t\ge 0)$ che finirà sul ciclo limite.\\
    Facendo questa operazione con un tempo continuo finiremo su tutti i possibili stati appartenenti al ciclo limite, noi invece vorremmo proprio finire su $\v{x}_0$.\\
    Utilizzando una sequenza discreta di tempi possiamo fare in modo di collare precisamente sul punto $\v{x}_0$ come mostrato in figura \ref{fig:4_1_3}
    \marginpar{
            \captionsetup{type=figure}
            \incfig{4_1_3}
            \caption{\scriptsize Sequenza discreta di tempi per raggiungere $\v{x}_0$}
        \label{fig:4_1_3}
        }
\end{exmp}
\noindent
