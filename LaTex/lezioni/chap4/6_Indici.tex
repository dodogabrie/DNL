\section{Teoria degli indici}%
\label{sub:Teoria degli indici}
Sia dato un campo vettoriale 
\[
\begin{dcases}
    \frac{\text{d} x}{\text{d} t} = F(x, y) \\
    \frac{\text{d} y}{\text{d} t} = G(x, y) 
\end{dcases}
\]
Definito nell'insieme $D \subset \mathbb{R}^2$ e $D$ semplicemente connesso.
\begin{exmp}[]
    Prendiamo una sorgente, il campo vettoriale\ldots Da riascoltare. 
\end{exmp}
\noindent
\begin{defn}[Indice di Curva chiusa]
    Sia dato il sistema dinamico di sopra, si definisce indice di una curva chiusa $\Gamma$ la seguente quantità:
    \[
	K \ (\text{indice}) = \frac{1}{2\pi}\oint \frac{FdG-GdF}{F^2+G^2}
    .\] 
    Con $dF$ e $dG$ differenziali dei relativi funzionali.
\end{defn}
\noindent
