\section{Stabilità di orbite periodiche (Floquet) }%
Prendiamo il sistema dinamico
\[
    \frac{\text{d} \v{x}}{\text{d} t} = F(\v{x}), \quad  \v{x}\in \mathbb{R}^n, \quad F:\mathbb{R}^n\to \mathbb{R}^n, \quad F\in C^r \ r\ge 2
.\] 
Supponiamo che esista una orbita periodica $\gamma = \v{x}_p(t) $ per il sistema:
\[
    \v{x}_p(t) = \v{x}_p(t+T) \quad  \forall t\ge 0
.\] 
Vogliamo studiare la stabilità di $\gamma$: definiamo allora la quantità
\[
    \v{x}(t) = \v{x}_p(t) + \v{y}(t) \quad  \left|\v{y}(t) \right|\ll 1
.\] 
Quindi sviluppiamo al primo ordine in $\v{y}$:
\[
    \frac{\text{d} \v{x}_p(t) }{\text{d} t} + \frac{\text{d} y(t) }{\text{d} t} = F(\v{x}_p+\v{y}) \simeq 
    F(\v{x}_p) + D F(\v{x}_p) \v{y}
.\] 
Quindi eliminando l'identità si ha l'evoluzione della perturbazione nel tempo:
\[
    \frac{\text{d} \v{y}}{\text{d} t} = DF(\v{x}_p) \v{y}
.\] 
Questo sistema è lineare ma \textbf{non autonomo}. Quest'ultima proprietà esclude l'utilizzo di tutta la teoria della linearizzazione per determinare la stabilità.
Possiamo utilizzare come unico vincolo la periodicità:
\[
    A(t) \equiv DF(\v{x}_p) \implies  A(t) = A(t+ T) 
.\] 
Possiamo riscrivere il sistema dinamico perturbato nella forma compatta:
\[
    \frac{\text{d} \v{y}}{\text{d} t} = A(t) \v{t}
.\] 
e dare la definizione:
\begin{defn}[Sistema fondamentale di soluzioni]
    Dato l'insieme di soluzioni:
    \[
	\left\{\v{y}_J(t) | \frac{\text{d} \v{y}_J}{\text{d} t} = A(t) \v{y}_J, \ J = 1, 2, \ldots, n  \right\} = S_F
    .\] 
    Allora $S_F$ è un sistema fondamentale di soluzioni se 
    \[
	\sum_{J=1}^{n} c_J\v{y}_J(t) = 0
    .\] soltanto se $c_J = 0 $ $\forall J = 1, 2, \ldots, n$ (quindi sono linearmente indipendenti).
\end{defn}
\noindent
\begin{thm}[Costruzione di un sistema fondamentale di soluzioni]
   Dato il sistema lineare:
   \[
       \frac{\text{d} \v{y}}{\text{d} t} = A(t) \v{y }(t) \quad  \v{y}\in \mathbb{R}^n
   .\] Allora esiste sempre un sistema fondamentale di soluzioni $S_F$.
\end{thm}
\noindent
\begin{proof}
    Prendiamo l'insieme costituito dai vettori della base canonica in $\mathbb{R}^n$.
    \[
        \left\{\hat{e}_J | J = 1, 2, \ldots, n\right\}
    .\] 
    Sia $\v{y}_J(t)$  la soluzione del sistema dinamico che soddisfa 
    \[
	\v{y}_J(0) = \hat{e}_J
    .\] Questa soluzione esiste per il teorema di esistenza ed unicità della soluzione.\\
    Dobbiamo dimostrare che una qualsiasi combinazione lineare è nulla solo con i oefficienti tutti nulli.
    \[
	\sum_{J=1}^{n} c_J\v{y}_J(t) = 0 \implies  c_J = 0
    .\] 
    Ma questa espressione è vera in quanto deve valere per ogni $t$ positivo, in particolare per $t = 0$. 
    \[
    \sum_{J = 1}^{n} c_J e_J
    .\] 
\end{proof}
\begin{defn}[Matrice fondamentale]
    Sia $S_F$ un sistema fondamentale di soluzioni di 
    \[
	\frac{\text{d} \v{y}}{\text{d} t} = A(t) \v{y}(t) 
    .\] 
    La matrice fondamentale associata a tale sistema dinamico è data da:
    \[
	\phi  = \left[\v{y}_1(t) \ldots \v{y}_n(t) \right]
    .\] 
\end{defn}
\noindent
Dalla definizione data si evince che vale 
\[
    \frac{\text{d} \phi}{\text{d} t} = A(t) \phi
.\] 
\begin{thm}[]
    Sia $A(t) $ tale che $A(t) = A(t+T)$,  allora $\phi (t+T)$  è soluzione di 
    \[
	\frac{\text{d} \phi}{\text{d} t} = A(t) \phi
    .\] 
\end{thm}
\noindent
\begin{proof}
    Facciamo il cambio di variabile:
    \[
        t' = t - T
    .\] Inserendo nella equazione:
    \[
	\frac{\text{d} \phi (t) }{\text{d} t} = \frac{\text{d} \phi (t' + T) }{\text{d} t'} = A(t) \phi (t) = 
	A(t'+T) \phi (t' + T) 
    .\] 
    \[
	\frac{\text{d} \phi (t'+T) }{\text{d} t} = A(t'+T) \phi (t'+T) \implies 
	\frac{\text{d} \phi (t' + T) }{\text{d} t} = A(t') \phi (t'+T)
    .\] 
    Quindi $\phi (t'+T)$ è soluzione.
\end{proof}
\noindent 
Quindi ora sappiamo che la matrice fondamentale e la sua traslata di un periodo $T$ sono entrambe soluzioni. Questo implica che $\phi (t+T)$ si deve ottenere dalla combinazione lineare di $\phi (t) $.
\[
    \phi (t+T) = \phi (t) \cdot M
.\] 
Con $M$  definita come \textbf{Matrice di Monodromia}.\\
A questa particolare matrice (di numeri, costanti) è legata la stabilità del sistema.\\
Dato che $\phi (t) $  è costruita utilizzando il sistema canonico $S_F$ implica che 
\[
    \phi (0) = \left[y_1(0) \ldots y_n(0) \right] \equiv \mathbb{I}
.\] 
Grazie a questa inizializzazione ed alla equazione sopra si ha che:
\[
    \phi (0 + T) = \phi (0) M \implies  \phi (T) = M
.\] 
Questa equazione è importantissima perchè ci consente di calcolare la matrice $M$  semplicemente trovando le $n$  soluzioni e sostituendovi il periodo $T$.\\
Sempre valuando l'equazione per $t = T$  si ha che:
\[
    \phi (T + T) = \phi (T) M \implies\phi (2T) = M^2=\phi (T) ^2
.\] 
Ed in generale abbiamo che:
\[
    \phi (nT) = M^n = \phi (T)^n 
.\] 
\begin{defn}[Moltiplicatori di Floquet]
    Gli autovalori della matrice di Monodromia sono definiti autovalori di Floquet.
\end{defn}
\noindent
Supponiamo che esista una matrice $P$  (non singolare) tale che 
\[
    D = P\phi (T) P^{-1} = PMP^{-1}
.\] 
In cui stiamo implicitamente assumento che $M$  abbia $n$  autovalori distinti.\\
Definiamo adesso la seguente matrice:
\[
    V(t) = \left[\v{v}_1(t) \ldots\v{v}_n(t) \right]= \phi (t) P^{-1}
.\] 
Inoltre possiamo scrivere che:
\[\begin{aligned}
    \phi (t+T) = \phi (t) \phi (T) 
.\end{aligned}\]
Moltiplicando a sinistra per $P^{-1}$  si ha:
\[\begin{aligned}
    \phi (t + T) P^{-1}= \phi (t) \phi (T) P^{-1}=\phi (t) P^{-1}P\phi (T) P^{-1} = v(t) D
.\end{aligned}\]
Se ne conclude allora che:
\[
    V(t+T) = V(t) D
.\] 
E questa relazione ci aiuta a decretare la stabilità dell'orbita periodica. Esplicitiamola diversamente:
\[
    \left[\v{v}_1\ldots\v{v}_n\right]_{t + T} = \left[\v{v}_1\ldots\v{v}_n\right]_{t} D
.\] 
Quindi i vettori $\v{v}_i$  sono i vettori di un cambiamento di base che sfrutta la matrice di monodronia.
\[
    \v{v}_J(t+T) = \v{v}_J(t) \lambda_J\implies\v{v}_J(t+nT) = \v{v}_J(t) \lambda^n_J
.\] 
Quindi possiamo vedere che
\begin{itemize}
    \item Se $\left|\lambda_J\right|>1$ allora $\v{v}_J$ diverge: quindi abbiamo una instabilità.
    \item Se $\left|\lambda_J\right|<1$ allora $\v{v}_J$ decade a zero, quindi l'orbita perturbata dovrà decadere su quella periodica non perturbata.
\end{itemize}
\begin{thm}[]
    Sia $\v{x}_p(t) $  un'orbita periodica del sistema dinamico
    \[
	\frac{\text{d} \v{x}}{\text{d} t} = F(\v{x}) 
    .\] 
    Allora la matrice $M$ di monodromia ha sempre un autovalore $\lambda =1$.
\end{thm}
\noindent
Una giustificazione intuitiva del precedente teorema è data dal fatto che, perturbando il sistema lungo la direzione dell'orbita periodica si deve ottenere sempre l'orbita periodica stessa.
\begin{defn}[Stabilità]
    Un'orbita periodica $\v{x}_p(t)$ è stabile se $\left|\lambda_J\right|<1$ $\forall J$ ( tranne uno).
\end{defn}
\noindent
\begin{defn}[Orbita iperbolica]
    L'orbita periodica $\v{x}_p(t) $ è iperbolica se nessuno (tranne uno) dei moltiplicatori di Floquet ha modulo 1.
\end{defn}
\noindent
\begin{defn}[Selle]
    $\v{x}_p(t)$ è una orbita periodica di tipo sella se esistono moltiplicatori di Floquet di modulo sia maggiore che minore di 1 (tranne uno).
\end{defn}
\noindent
\begin{defn}[Orbita periodica non iperbolica]
    $\v{x}_p(t)$ è una orbita periodica non iperbolica se almeno 1 dei moltiplilcatori di Floquet ha modulo 1.
\end{defn}
\noindent
\begin{exmp}[]
    Prendiamo il seguente sistema dinamico:
    \[
    \begin{dcases}
    \frac{\text{d} x}{\text{d} t} = x-y-x^3-xy^2\\
    \frac{\text{d} y}{\text{d} t} = x + y - x^2y-y^3\\
    \frac{\text{d} z}{\text{d} t} = \lambda z
    \end{dcases}
    \]
    Verificare (per casa) che una soluzione è:
    \[
	\v{x}_p(t) = (\cos t, \sin t, 0)
    .\] 
    E che tale soluzione è periodica (di periodo $T = 2\pi$). Vogliamo studiare la stabilità di $\v{x}_p(t) $, per farlo partiamo dalla Jacobiana:
    \[
	J(x, y, z) = 
	\begin{pmatrix}
	    1-3x^2 -y^2 & -1-2xy  & 0 \\
	    1-2xy & 1-x^2-3y^2 & 0 \\
	    0 & 0 & \lambda \\
	\end{pmatrix}
    .\] 
    La matrice valutata nell'orbita periodica è:
    \[
	\left.J(x, y, z)\right|_{\v{x}_p(t) }=
	   \begin{pmatrix}
	       -2\cos^2t & -1-\sin (2t)  & 0 \\
	       1-\sin (2t)  & -2\sin^2t & 0 \\
	       0 & 0 & \lambda \\
	   \end{pmatrix}
    .\] 
    Perturbiamo adesso il sistema:
    \[
	\v{x}(t) = \v{x}_p(t) + \v{ q}(t) \quad  \left|\v{ q}\right|\ll 1
    .\] 
    \[
	\frac{\text{d} \v{q }}{\text{d} t} = A(t) \v{ q}(t) 
    .\] 
    Determiniamo $S_F$: il sistema fondamentale di soluzioni (per casa). Data la condizione iniziale (vettore e soluzione associata):
    \[
	\hat{e}_1 = \begin{pmatrix} 1 \\ 0 \\ 0 \end{pmatrix} \implies  \v{q}_1(t) = (e^{-2t}\cos t, e^{-2t}\sin t, 0)
    .\] 
    \[
	\hat{e}_2 = \begin{pmatrix} 0 \\ 1 \\ 0 \end{pmatrix} \implies  \v{q}_2(t) = (-\sin t, \cos t, 0)
    .\] 
    \[
	\hat{e}_3 = \begin{pmatrix} 0 \\ 0 \\ 1 \end{pmatrix} \implies  \v{q}_3(t) = (0, 0, e^{\lambda t})
    .\] 
    Si ha il sistema fondamentale di soluzioni (per il teorema dimostrato):
    \[
	S_F = \left\{\v{q }_1(t) , \v{q }_2(t) , \v{q }_3(t) \right\}
    .\] 
    A questo punto calcoliamo la matrice fondamentale $\phi$ e la matrice di Monodromia $M$. 
    \[
	\phi (t) = \left(\v{q }_1(t) , \v{q }_2(t) , \v{q }_3(t)\right)
    .\] 
    Che soddisfa ovviamente la condizione $\phi (0) = \mathbb{I}$. Inserendo il periodo nella matrice fondamentale otteniamo subito la matrice di Monodromia:
    \[
	M = \phi (T) = 
\begin{pmatrix}
    e^{-4\pi} & 0 & 0 \\
    0 & 1 & 0 \\
    0 & 0 & e^{2\pi\lambda} \\
\end{pmatrix}
    .\] 
    Quindi i moltiplicatori di Floquet sono:
    \[\begin{aligned}
        & \lambda_1 = e^{-4\pi}<1\\
	& \lambda_2=1\\
	& \lambda_3=e^{2\pi\lambda}
    .\end{aligned}\]
    Quindi la stabilità dell'orbita periodica dipende dal segno di $\lambda$. 
\end{exmp}
\noindent
