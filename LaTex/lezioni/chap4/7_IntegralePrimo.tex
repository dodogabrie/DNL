\section{Sistemi dinamici con integrale Primo}%
\label{sub:Sistemi dinamici con integrale Primo}
Dato un sistema dinamico 
\[
    \frac{\text{d} \v{x}}{\text{d} t} = F(\v{x}) \quad  \v{x}\in \mathbb{R}^n, \quad F:\mathbb{R}^n\to \mathbb{R}^n, \quad  F \in C^r \ r \ge 1
.\] 
Supponiamo $\exists I (\v{x}) $  un funzionale:
\[
    I(\v{x}): \mathbb{R}^n\to \mathbb{R}^n
.\] 
Avente derivata orbitale nulla:
\[
    \frac{\text{d} I(\v{x}) }{\text{d} t} =  \nabla I(\v{x}) \cdot F = 0
.\] 
\marginpar{
        \captionsetup{type=figure}
        \incfig{4_7_1}
        \caption{\scriptsize Integrale del moto e orbite del sistema.}
    \label{fig:4_7_1}
    }
Allora si dice che il sistema dinamico ha un integrale primo del moto.\\
Il fatto che un sistema ammetta un integrale primo del moto implica dei pesanti vincoli geometrici sulla struttura delle orbite (vedi ad esempio i sistemi Hamiltoniani che devono conservare l'energia).\\
Osservando la definizione di derivata orbbitale si deduce che le orbite del sistema giaciono tutte sulle superfici $I(\v{x}) = c$, questo perchè deve esser sempre vera la condizione sulla derivata orbitale $\nabla I(\v{x}) \cdot F(\v{x}) = 0$ (vedere figura \ref{fig:4_7_1}).
\begin{exmp}[]
    \[
    \begin{dcases}
    \frac{\text{d} x}{\text{d} t} = y\\
    \frac{\text{d} y}{\text{d} t} = x-x^3
    \end{dcases}
    \]
    Questo campo vettoriale conserva una quantità, possiamo notarlo riscrivendo il sistema:
    \[
        \frac{\text{d} }{\text{d} t} \left[\frac{1}{2}\left(\frac{\text{d} x}{\text{d} t} ^2\right)- \frac{x^2}{2}+\frac{x^4}{4}\right]= 0
    .\] 
    Quindi abbiamo che:
    \[
	I(x, y) = \frac{y^2}{2}-\frac{x^2}{2}+\frac{x^4}{4}= \text{cost}\equiv H
    .\] 
    Quindi possiamo risalire alle orbite come:
    \[
	y(x) = \pm \sqrt{2(H + \frac{x^2}{2}-\frac{x^4}{4}) } 
    .\] 
    Notiamo che tutte le orbite che attraversano l'asse $x$ lo fanno perpendicolarmente.
\end{exmp}
\noindent
