\section{Studio della stabilità mediante linearizzazione}%
\label{sub:Studio della stabilità mediante linearizzazione per SD a tempo continuo autonomo}
Prendiamo il sistema dinamico autonomo a tempo continuo:
\[
    \frac{\text{d} \vect{x}}{\text{d} t} = F(\vect{x}),  \quad \vect{x}\in \mathbb{R}^n, \quad F:\mathbb{R}^n\to \mathbb{R}^n, \quad F \in C^r,\ r\ge 2
\] 
Supponiamo che le soluzioni esistano globalmente. \\
\marginpar{
    \captionsetup{type=figure}
        \incfig{13_1}
	\caption{\scriptsize Perturbazione ($y(t)$) della soluzione $x_p(t)$.}
    \label{fig:13_1}
}
Figura: Se l'orbita è stabile allora $y(t)$  è confinata ad un tubo di flusso\\
La soluzione allora è definita dalla somma della soluzione imperturbata e del disturbo.
\[
    \vect{x(t)} =\vect{x}_p(t)+\vect{y} (t) \qquad \text{ con } \left|\vect{y} (t)\right|\ll 1
\] 
Quindi possiamo dire che:
\[
    \frac{\text{d} \vect{x}}{\text{d} t} = \frac{\text{d} \vect{x}_p}{\text{d} t} + \frac{\text{d} \vect{t}}{\text{d} t} = F(\vect{x}_p + \vect{y}_p)
\] 
\begin{defn}[Funzione differenziabilie]
    Sia $F:I\to J$ con $I \subset \mathbb{R}^n, \ J \subset \mathbb{R}^n$. $I$ è un aperto e $\vect{x}_0 \in I$.\\
    Si dice che $F$ è differenziabile in $\vect{x}_0$ se $\exists$ $DF(\vect{x}_0) \in L(\mathbb{R}^n, \mathbb{R}^n)$ (spazio delle applicazioni lineari) tale che:
    \[
	\lim_{\left|\vect{h}\right| \to 0} \frac{\left|F(\vect{x}_0+ \vect{h})-F(\vect{x}_0)-DF(\vect{x}_0) \vect{h}\right|}{\left|h\right|} = 0
    \] 
\end{defn}
\noindent
\begin{thm}[Sullo Jacobiano]
    Sia $F:I\to J$, $I \subset \mathbb{R}^n, \ J \subset \mathbb{R}^n$, supponiamo $I$  aperto.\\
    Se $F$  è differenziabile in $\vect{x}_0$  allora:
    \begin{enumerate}
        \item esistono le derivate parziali: $\left.\frac{\partial F_I}{\partial x_J}\right|_{\vect{x}_0} $ con $i, J = 1, 2, \ldots n$. 
	    \item $\forall \ \vect{h}  \in I$ si ha che:
		\[
		    \left.\left[DF(\vect{x}_0) \vect{h}\right]\right|_{i} = 
			\sum_{J=1}^{n} \frac{\partial F_i}{\partial x_J} h_J, 
			\qquad \left.\left[DF(x_0)\right]\right|_{i, J} = \frac{\partial F_i}{\partial X_J} 
		\] 
    \end{enumerate}
\end{thm}
\begin{ex}[Calcolo di DF]
    Presa la mappa:
    \[
        F = \begin{pmatrix} F_1 \\ F_2 \end{pmatrix} = 
	\begin{pmatrix} x_1 -x_2^2 \\ x_1x_2-x_2 \end{pmatrix} 
    \] 
    Calcolare $DF(\vect{V}_0)$ nel punto $\vect{V}_0=\begin{pmatrix} 1\\ -1 \end{pmatrix} $.
\end{ex}
\noindent
\noindent
Tornando alla nostra linearizzazione:
\[
    \frac{\text{d} \vect{x}_p(t)}{\text{d} t} + \frac{\text{d} \vect{y} (t)}{\text{d} t} = 
    F(\vect{x}_p(t)+ \vect{y} (t)) \simeq 
    F(\vect{x}_p(t))+DF(\vect{x}_p(t))\vect{y}  + O(\vect{y})
\] 
Quindi eliminando l'identità nella precedente equazione ci si riduce alla sola dinamica della perturbazione:
\[
    \frac{\text{d} \vect{y} (t)}{\text{d} t} = DF(\vect{x}_p(t))\vect{y}\equiv
    J(\vect{x}_p)\vect{y}
\] 
Questo campo vettoriale ha una soluzione soluzione stazionaria: 
\[
\vect{y}_s = \begin{pmatrix} 0 \\ \vdots \\ 0 \end{pmatrix}
\] 
L'unica cosa da tenere a mente è che $J(\vect{x}_p(t))$  potrebbe dipendere dal tempo (se $\vect{x}_p(t)$ non è una soluzione stazionaria il sistema non è autonomo). Per adesso ci limitiamo a considerare le soluzioni $\vect{x}_p(t)$  stazionarie: $\vect{x}_s$:
\[
    \frac{\text{d} \vect{y}}{\text{d} t} = J(\vect{x}_s)\vect{y}  \qquad \vect{J(\vect{x}_s)}  \text{: matrice costante}
\] 
\begin{thm}[Stabilità delle soluzioni stazionarie]
    Dato $\frac{\text{d} \vect{x}}{\text{d} t} = F(\vect{x})$ con $\vect{x}\in\mathbb{R}^n$, $F:\mathbb{R}^n \to \mathbb{R}^n$, $F \in C^r, \ r\ge 2$ e sia $\vect{x}_s$ tale che $F(\vect{x}_s)=0$.\\
    Se tutti gli autovalori di $J(\vect{x}_s)$ hanno parte reale negativa allora $\vect{x}_s$ è asintoticamente stabile.
\end{thm}
\noindent
Teniamo presente che la stabilità espressa dal teorema è stabile, se si può dimostrare anche che la condizione vale anche per intorni arbitrari allora possiamo decretare anche la stabilità globale.\\
Potrei utilizzare lo stesso approccio per un sistema non autonomo? Potrei trovare gli autovalori della matrice $J$ non autonoma. \\
Se trovo autovalori con parte reale minore di zero potrei concludere la stabilità del sistema? \textbf{NO}.
\begin{exmp}[Il teorema non funziona per sistemi non autonomi]
    \[
	\frac{\text{d} }{\text{d} t} \begin{pmatrix} x_1 \\ x_2 \end{pmatrix} = A(t)\begin{pmatrix} x_1 \\ x_2 \end{pmatrix} 
    \] 
    Con:
    \[
	A(t) = 
	\begin{pmatrix} 
	    -1 + \frac{3}{2}\cos^2t & 1- \frac{3}{2}\sin t\cos t \\
	    -1 -3\sin t \cos t & -1 + \frac{3}{2}\sin^2t
	\end{pmatrix} 
    \] 
    Prendendo il vettore nullo avrei uno stato stazionario per il sistema. Cerchiamo gli autovalori della matrice $A$:
    \[\begin{aligned}
	\text{det}(A(t)- \Lambda  \mathbb{I}) =& (-1 + \frac{3}{2}\cos^2t - \Lambda)(-1+\frac{3}{2}\sin^2t -\Lambda)+\\
					       &+(1+\frac{3}{2}\sin t\cos t)(1-\frac{3}{2}\sin t\cos t)=0
    .\end{aligned}\]
    Dalla equazione secolare si ottiene:
    \[
        \Lambda^2 + \frac{\Lambda}{2}+\frac{1}{2}=0 \implies  \Delta  = \frac{1}{4}-2 = -\frac{7}{4} <0
    \] 
    Quindi abbiamo autovalori complessi coniugati (CC):
    \[
	\Lambda_{12} = \frac{-1 \pm i \sqrt{\frac{7}{4}}}{4} \implies  \text{Re}(\Lambda_{12})=-\frac{1}{4}
    \] 
    Applicando alla lettera il teorema la soluzione stazionaria deve essere stabile. Se consideriamo invece:
    \[
	\vect{x}_1(t)=e^{t /2}\begin{pmatrix} -\cos t \\ \sin t \end{pmatrix}  \qquad \vect{x}_1(t)=e^{-t}\begin{pmatrix} \sin t \\ \cos t \end{pmatrix} 
    \] 
    Si scopre che queste due sono soluzioni indipendenti (per casa). \\
    Quindi prendendo queste due soluzioni per descrivere la soluzione del sistema dinamico e ponendoci in un intorno della soluzione stazionaria si vede che una direzione non è stabile ($\vect{x_1}$), mentre una direzione è stabile $\vect{x}_2$. Il fatto che la soluzione in $\vect{x}_1$ diverga rende il punto $\vect{V}_0 = \begin{pmatrix} 0\\ 0 \end{pmatrix} $ instabile. \\
   Questo dimostra che quando la matrice Jacobiana non è autonoma il teorema non si applica.
\end{exmp}
\noindent
\begin{exmp}[Sistema autonomo]
    \[\begin{dcases}
	\frac{\text{d} x}{\text{d} t} = - y + x(x^2+y^2)\\
	\frac{\text{d} y}{\text{d} t} = x + y(x^2+y^2)
    \end{dcases}
    =
    \begin{pmatrix} F_1(x, y)\\ F_2(x, y) \end{pmatrix} 
    \] 
    L'unico stato stazionario in questo caso è il vettore $\vect{V}_s = (0,0)$.\\
    Per determinare la stabilità come prima cosa dobbiamo calcolare la generica matrice $J$:
    \[
	J(\vect{V})=
        \left.
	\begin{pmatrix} 
	    3x^2 + y^2 & -1 + 2xy \\
	    1 + 2xy & x^2 + 3y^2
	\end{pmatrix} 
        \right|_{\vect{V} = \vect{V}_s} = 
	\begin{pmatrix} 0 & -1 \\ 1 & 0 \end{pmatrix} 
    \] 
    Gli autovalori di questa matrice sono dati dalla equazione secolare:
    \[
        \Lambda^2+1 = 0 \implies  \Lambda_{12} = \pm i
    \] 
    Quindi la parte reale è nulla\ldots\\
    Notiamo che la sola parte lineare di questo sistema rappresenta un oscillatore armonico, l'unico punto fisso dell'oscillatore armonico (l'origine) è stabile secondo Lyapunov. In realtà questa conclusione è errata: lo stato stazionario non è stabile. \\
    Dobbiamo allora stare attenti al fatto che quando qualcuno degli autovalori ha una parte reale nulla c'è bisogno di molta cautela nella interpretazione dei risultati.\\
    Possiamo dimostrare l'instabilità di tale punto fisso sfruttando la simmetria del termine non lineare:
    \[\begin{aligned}
	&x(t)= r(t)\cos (\theta (t))\\
	&y(t)= r(t)\sin (\theta (t))
    .\end{aligned}\]
    con $x^2+y^2 = r^2$.
    \[\begin{aligned}
	&\frac{\text{d} x(t)}{\text{d} t} = \frac{\text{d} r}{\text{d} t} \cos (\theta)- r\sin\theta\frac{\text{d} \theta}{\text{d} t} \\
	&\frac{\text{d} y}{\text{d} t} = \frac{\text{d} r}{\text{d} t} \sin\theta  + r\cos\theta  \frac{\text{d} \theta}{\text{d} t} 
    .\end{aligned}\]
    Mettendo nelle equazioni del moto:
    \[
	-r\sin\theta  \frac{\text{d} \theta}{\text{d} t} + \frac{\text{d} r}{\text{d} t} \cos\theta  = 
	    - r(t)\sin\theta  + r\cos\theta r^2
    \] 
    \[
        r\cos\theta  \frac{\text{d} \theta}{\text{d} t} + \frac{\text{d} r}{\text{d} t} \sin\theta  = r\cos\theta  + r \sin\theta r^2
    \] 
    Moltiplicando la prima equazione per il seno di $\theta$ e la seconda per il coseno di $\theta$ e sottraendo membro a membro le equazioni:
    \[
	r \frac{\text{d} \theta}{\text{d} t} = r \implies  \frac{\text{d} \theta}{\text{d} t} = 1 \implies  \theta (t)=t + \theta_0
    \] 
    Per casa: moltiplicare la la prima equazione per $\cos\theta$ e la seconda per $\sin\theta$ e sommarle. Si ottiene che:
    \[
        \frac{\text{d} r}{\text{d} t} = r^3
    \] 
    Quindi il sistema dinamico di partenza si è ridotto a:
    \[\begin{dcases}
        \frac{\text{d} r}{\text{d} t} = r^3\\
	\frac{\text{d} \theta}{\text{d} t} =1
    \end{dcases}\] 
    Abbiamo un sistema dinamico definito in un manifold: 
    \[
        S^1 \times R^+ \cup \left\{0\right\}
    \] 
    L'equazione interessante è la prima: questa ci dice che il sistema evolve sempre verso $r \to \infty$ per qualsiasi intorno del punto fisso. Quindi lo stato stazionario non è stabile.
\end{exmp}
\noindent
\begin{defn}[Soluzione stazionaria iperbolica]
    Dato il seguente campo vettoriale: $\frac{\text{d} \vect{x}}{\text{d} t} = F(\vect{x})$ con $\vect{x}\in\mathbb{R}^n$, $F:\mathbb{R}^n \to \mathbb{R}^n$ e $\vect{x}_s$ tale che $F(\vect{x}_s)=0$.\\
    Diciamo che $\vect{x}_s$ è una soluzione stazionaria iperbolica se nessuno degli autovalori di $J(\vect{x}_s)$ ha parte reale nulla.
\end{defn}
\noindent
\begin{defn}[Soluzione stazionaria non iperbolica]
    Dato il seguente campo vettoriale: $\frac{\text{d} \vect{x}}{\text{d} t} = F(\vect{x})$ con $\vect{x}\in\mathbb{R}^n$, $F:\mathbb{R}^n \to \mathbb{R}^n$ e $\vect{x}_s$ tale che $F(\vect{x}_s)=0$.\\
    Diciamo che $\vect{x}_s$ è una soluzione stazionaria non iperbolica se non è iperbolica\ldots
\end{defn}
\noindent
