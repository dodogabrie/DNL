\section{Spazio delle fasi esteso (SD a tempi continui)}%
\label{sub:Spazio delle fasi esteso (SD a tempi continui)}
Si prende un sistema dinamico a tempi continui autonomo e lo si perturba con una componente dipendente dal tempo (un fattore esterno). Il sistema in questo modo diventa non autonomo, l'equazione generale che regola questo tipo di sistema è:
\[
    \frac{\text{d} \vect{x}}{\text{d} t} = F(\vect{x},t) \qquad 
    \vect{x}\in \mathbb{R}^n; \quad
    F: \mathbb{R}^n \times \mathbb{R}\to \mathbb{R}^n
.\] 
Possiamo ricondurre questo sistema ad un sistema autonomo tramite una trasformazione nella variabile temporale:
\[
    t = m(s) = s \implies  \frac{\text{d} }{\text{d} t} = \frac{\text{d} s}{\text{d} t} \frac{\text{d} }{\text{d} s} 
.\] 
Inserendo nella equazione del moto:
\[
    \frac{\text{d} \vect{x}}{\text{d} t} = \frac{\text{d} s}{\text{d} t} \frac{\text{d} \vect{x}}{\text{d} s} = F(\vect{x}, t)
.\] 
Possiamo definire il differenziale di $t$  rispetto a $s$: $dt /ds = 1$.
\[
    \begin{cases}
	\frac{\text{d} \vect{x} (s)}{\text{d} s} = F(\vect{x}, t)\\
	\frac{\text{d} t}{\text{d} s} = 1
    \end{cases}
\] 
\begin{defn}[Spazio delle fasi esteso]
    Si definisce spazio delle fasi esteso la quantità:
    \[
	\vect{y} =(\vect{x}, t) \in \mathbb{R}^n \times \mathbb{R}
    .\] 
\end{defn}
\noindent
In questo modo, definendo anche il funzionale esteso:
\[
    H = (F(\vect{x}, t), 1)
.\] 
Si possono generalizzare le equazioni del moto come:
\[
    \frac{\text{d} \vect{y}}{\text{d} s} = H(\vect{y})
.\] 
Per quanto il problema sia formalmente risolto si deve tenere in considerazione che il nuovo spazio delle fasi potrebbe non essere più un compatto.\\
Questa mancanza potrebbe diventare un problema nei nostri scopi in quanto siamo spesso interessati alla soluzione asintotica del sistema (che potrebbe smettere di esistere).\\
In ogni caso aggiungiamo che, se la forzante è periodica, il sistema può essere sempre gestito con questo metodo.
\begin{exmp}[Forzante oscillante]
    \[
	\frac{\text{d} ^2x}{\text{d} t^2} = -x + A\sin (\omega t)
    .\] 
    Come sempre si riporta l'equazione ad una di primo ordine:
    \[
        \begin{cases}
            \frac{\text{d} x}{\text{d} t} = y \\
	    \frac{\text{d} y}{\text{d} t} = -x + A \sin (\omega t)
        \end{cases}
    \] 
    Adesso si introduce la variabile $\theta (t)=\omega t$. Il nuovo sistema, con questa variabile, è descritto nello spazio delle fasi generalizzato e le equazioni sono le seguenti:
    \[
        \begin{cases}
            \frac{\text{d} x}{\text{d} t} = y\\
	    \frac{\text{d} x}{\text{d} t} = -x + A\sin\theta\\
	    \frac{\text{d} \theta}{\text{d} t} =\omega
        \end{cases}
    \] 
    Si noti che la variabile $\theta$ non è limitata, quindi lo spazio delle fasi non è più un compatto.
\end{exmp}
