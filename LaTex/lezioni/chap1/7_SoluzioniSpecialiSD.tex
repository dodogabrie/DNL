\section{Soluzioni speciali di Sistemi Dinamici}%
\label{sub:Soluzioni speciali di Sistemi Dinamici}
Analizziamo il regime asintotico di un sistema dinamico, i tipi di soluzione che si possono incontrare sono:
\begin{enumerate}
    \item Stati Stazionari Costanti.
    \item Stati Stazionari Dinamici.
	\begin{itemize}
	    \item Periodici.
	    \item Quasi Periodici.
	    \item Complessi.
	\end{itemize}
\end{enumerate}
\subsection{Stati Stazionari Costanti}%
\label{sub:Stati stazionari Costanti}
Questi stati sono indipendenti dal tempo, ipotizzando che la soluzione stazionaria si $\vect{x}(t)$ allora:
\[
    \vect{x} (t + \Delta t)=\vect{x} (t)
.\] 
Nei libri sono spesso chiamati Punti Singolari, Punti Critici, Soluzioni Stazionarie.
\subsection{Stati Stazionari Dinamici}%
\label{sub:Stati Stazionari Dinamici}
Lo stato per questi sistemi non è costante nel tempo, analizziamo le più comuni situazioni che si possono presentare in questi sistemi.
\paragraph{Orbite periodiche}%
\label{par:Orbite periodiche}
L'orbita di uno stato stazionario dinamico periodico è un'orbita che si ripete nel tempo.
\begin{exmp}[Oscillatore non lineare]
    \marginpar{
        \captionsetup{type=figure}
            \incfig{7_1}
        \caption{\scriptsize Orbita Periodica che attrae la dinamica nello spazio delle fasi.}
        \label{fig:7_1}
    }
    Un oscillatore non lineare è un sistema che presenta un'orbita periodica come in figura \ldots\\
    Se lo stato $\vect{x}_1$ si trova (con le condizioni iniziali) sull'orbita allora rimarrà su tale orbita a stazionarietà. Se uno stato $\vect{x}_2$ si trova invece in un altro punto dello spazio delle fasi inizialmente allora evolverà per raggiungere l'orbita stabile (a stazionarietà).
\end{exmp}
\noindent
\paragraph{Orbite quasi periodice}%
\label{par:Orbite quasi periodice}
Sono orbite che non si ripetono nel tempo, sono più complesse delle orbite periodiche. La loro struttura verrà approfondita nel seguito.
\paragraph{Comportamenti complessi}%
\label{par:Comportamenti complessi}
Quando un sistema presenta, ad esempio, caos deterministico.
\subsection{Orbite periodiche di sistema dinamico}%
\label{sub:Orbite periodiche di sistema dinamico}
\begin{defn}[Orbita periodica per SD a tempo continuo]
Prendiamo un Sistema Dinamico a tempo continuo:
\[
    \begin{dcases}
	\frac{\text{d} \vect{x}}{\text{d} t} = F(\vect{x}, t)\\
	\vect{x} (t_0) = \vect{x}_0
    \end{dcases}
    \qquad
    F: \ I \times \mathbb{R}^n \to \mathbb{R}^n
.\] 
 Sia $\vect{x}_p(t)$ la soluzione dell'IVP, diciamo che $\vect{x}_p(t)$ è periodica se:
\[
    \exists \ T \in \mathbb{R}^+: \ \vect{x}_p(t) = \vect{x}_p (t + T) \  \forall t \in I
.\]    
\end{defn}
\noindent
\begin{defn}[Orbita q-periodica per SD a tempo discreto]
    Dato un Sistema Dinamico a tempo discreto:
    \[
	\vect{x}_{k+1} = G(\vect{x}_k) \qquad \vect{x}_k \in \mathbb{R}^n
    .\] 
    Diciamo che $\vect{x}_p$ è una orbita $q$-periodica con $q \in \mathbb{N}$ se:
    \[
	\exists \ q \in \mathbb{N}: \
	G^{q} (\vect{x}_p) = \vect{x}_p
    .\] 
\end{defn}
\noindent
Prima di procedere definiamo la seguente categoria di funzioni:
\begin{defn}[Funzioni quasi periodiche]
    Una funzione $H$ si dice Quasi Periodica se può essere rappresentata nella seguente forma:
    \[
	H(t) = H(\omega_1t, \omega_2t, \ldots, \omega_nt)
    .\] 
    Con l'insieme di frequenze $\left\{\omega_i\right\}$ tra di loro Incommensurabili.\\
    Questo significa che non esiste una combinazione lineare di queste frequenze con coefficienti in $\mathbb{Q}$ che si annulla.
\end{defn}
\noindent
Preso un sistema a tempo continuo non autonomo e supponiamo di avere uno spazio delle fasi con un'orbita chiusa: l'orbita è necessariamente periodica? No.
