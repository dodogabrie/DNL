\section{Soluzioni speciali di Sistemi Dinamici}%
\label{sub:Soluzioni speciali di Sistemi Dinamici}
Analizziamo il regime asintotico di un sistema dinamico, i tipi di soluzione che si possono incontrare sono:
\begin{enumerate}
    \item Stati Stazionari Costanti.
    \item Stati Stazionari Dinamici.
	\begin{itemize}
	    \item Periodici.
	    \item Quasi Periodici.
	    \item Complessi.
	\end{itemize}
\end{enumerate}
\subsection{Stati Stazionari Costanti}%
\label{sub:Stati stazionari Costanti}
Questi stati sono indipendenti dal tempo, ipotizzando che la soluzione stazionaria si $\vect{x}(t)$ allora:
\[
    \vect{x} (t + \Delta t)=\vect{x} (t)
.\] 
Nei libri sono spesso chiamati Punti Singolari, Punti Critici, Soluzioni Stazionarie.
\subsection{Stati Stazionari Dinamici}%
\label{sub:Stati Stazionari Dinamici}
Lo stato per questi sistemi non è costante nel tempo, analizziamo le più comuni situazioni che si possono presentare in questi sistemi.
\paragraph{Orbite periodiche}%
\label{par:Orbite periodiche}
L'orbita di uno stato stazionario dinamico periodico è un'orbita che si ripete nel tempo.
\begin{exmp}[Oscillatore non lineare]
    \marginpar{
        \captionsetup{type=figure}
            \incfig{7_1}
        \caption{\scriptsize Orbita Periodica che attrae la dinamica nello spazio delle fasi.}
        \label{fig:7_1}
    }
    Un oscillatore non lineare è un sistema che presenta un'orbita periodica come in figura \ldots\\
    Se lo stato $\vect{x}_1$ si trova (con le condizioni iniziali) sull'orbita allora rimarrà su tale orbita a stazionarietà. Se uno stato $\vect{x}_2$ si trova invece in un altro punto dello spazio delle fasi inizialmente allora evolverà per raggiungere l'orbita stabile (a stazionarietà).
\end{exmp}
\noindent
\paragraph{Orbite quasi periodice}%
\label{par:Orbite quasi periodice}
Sono orbite che non si ripetono nel tempo, sono più complesse delle orbite periodiche. La loro struttura verrà approfondita nel seguito.
\paragraph{Comportamenti complessi}%
\label{par:Comportamenti complessi}
Quando un sistema presenta, ad esempio, caos deterministico.
\subsection{Orbite periodiche di sistema dinamico}%
\label{sub:Orbite periodiche di sistema dinamico}
\begin{defn}[Orbita periodica per SD a tempo continuo]
Prendiamo un Sistema Dinamico a tempo continuo:
\[
    \begin{dcases}
	\frac{\text{d} \vect{x}}{\text{d} t} = F(\vect{x}, t)\\
	\vect{x} (t_0) = \vect{x}_0
    \end{dcases}
    \qquad
    F: \ I \times \mathbb{R}^n \to \mathbb{R}^n
.\] 
 Sia $\vect{x}_p(t)$ la soluzione dell'IVP, diciamo che $\vect{x}_p(t)$ è periodica se:
\[
    \exists \ T \in \mathbb{R}^+: \ \vect{x}_p(t) = \vect{x}_p (t + T) \  \forall t \in I
.\]    
\end{defn}
\noindent
\begin{defn}[Orbita periodica per SD a tempo continuo]
    Dato un Sistema Dinamico a tempo discreto:
    \[
	\vect{x}_{k+1} = G(\vect{x}_k) \qquad \vect{x}_k \in \mathbb{R}^n
    .\] 
    Diciamo che $\vect{x}_p$ è una orbita $q$-periodica con $q \in \mathbb{N}$ se:
    \[
	G^{q} (\vect{x}_p) = \vect{x}_p
    .\] 
\end{defn}
\noindent
Prima di procedere definiamo la seguente categoria di funzioni:
\begin{defn}[Funzioni quasi periodiche]
    Una funzione $H$ si dice Quasi Periodica se può essere rappresentata nella seguente forma:
    \[
	H(t) = H(\omega_1t, \omega_2t, \ldots, \omega_nt)
    .\] 
    Con l'insieme di frequenze $\left\{\omega_i\right\}$ tra di loro Incommensurabili.\\
    Questo significa che non esiste una combinazione lineare di queste frequenze con coefficienti in $\mathbb{Q}$ che si annulla.
\end{defn}
\noindent
Preso un sistema a tempo continuo non autonomo e supponiamo di avere uno spazio delle fasi con un'orbita chiusa: l'orbita è necessariamente periodica? No.
\begin{ex}[Sistema in $\mathbb{R}^2$]
    Prendiamo il seguente:
    \[
        \begin{dcases}
            \frac{\text{d} x}{\text{d} t} = nt^{n-1}y\\
	    \frac{\text{d} y}{\text{d} t}  = -nt^{n-1}x
        \end{dcases}
    \] 
    Dimostrare che la soluzione è:
    \[\begin{aligned}
	&x(t) = A\sin (t^n) + B\sin (t^n)\\
	&y(t) = A\cos (t^n) - B\sin (t^n)
    .\end{aligned}\]
    Verificare che $x^2+y^2 = A^2+B^2$.\\
    Le soluzioni formano un cerchio di raggio $R^2 = A^2+B^2$. Nonostante questo la soluzione non è periodica perché:
    \[
	\nexists \ T \ \text{t.c.} \ t^{u} = (t + T)^u
    .\] 
\end{ex}
\noindent
\begin{ex}[Verifica di non periodicità]
    Data la seguente equazione differenziale:
    \[
	\frac{\text{d} x}{\text{d} t} = (1+\sin (t))\cdot x = F(x,t)
    .\] 
    Dimostrare che, anche se il coefficiente $1+\sin t$ è periodico, la soluzione non è periodica risolvendo il seguente IVP:
    \[
        \begin{dcases}
	    \frac{\text{d} x}{\text{d} t} = (1+\sin t)\cdot x\\
	    x(0) = x_0
        \end{dcases}
    \] 
    Dimostrare che la seguente funzione è soluzione:
    \[
	x(t) = x_0e^{1+t-\text{cost}}
    .\] 
    e che questa funzione non è mai periodica $\forall x_0 \in \mathbb{R}$.
\end{ex}
\noindent
\begin{ex}[Esercizio con Simulazione]
    Presa la seguente equazione differenziale:
    \[
	\frac{\text{d} ^2\theta}{\text{d} t^2} = - \frac{g}{l}\sin (\theta) - \frac{\gamma}{ml}\frac{\text{d} \theta}{\text{d} t} + \frac{r}{ml}\sin (\Omega  t)
    .\] 
    Ridefinire la variabile temporale e gli opportuni parametri per ricondurlo a:
    \[
	\frac{\text{d} ^2\theta}{\text{d} t^2} = - \sin (\theta) - b \frac{\text{d} \theta}{\text{d} t} + A\sin (\Omega  t)
    .\] 
    Verificare numericamente che per $b=0.05$, $a = 0.6$, $\Omega  = 0.7$ il sistema presenta un comportamento asintotico complesso.
\end{ex}
