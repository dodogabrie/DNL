\section{Introduzione ai Manifold}%
\label{sub:Introduzione ai Manifold}
Abbiamo fin'ora affermato che lo stato di un sistema dinamico è descritto da un vettore di $\mathbb{R}^n$, in questa sezione cerchiamo di essere più precisi riguardo a questa quantità.
\begin{exmp}[Pendolo nello spazio delle fasi]
    \[
        \begin{cases}
            \frac{\text{d} \theta}{\text{d} t} = y\\
	    \frac{\text{d} y}{\text{d} t} = -\frac{g}{l}\sin\theta
        \end{cases}
    \] 
    In questo caso abbiamo che lo stato $\vect{x} = (\theta, y)$ non è un vettore di $\mathbb{R}^n$ generico: $\theta$ è un angolo, $y$ è una velocità angolare.\\
    Lo stato è descritto in $\mathbb{R}^2$, la dinamica del sistema giace su una superficie dello spazio delle fasi detto \textbf{Manifold}.\\
    Il manifold per il problema del pendolo è una superficie cilindrica, si ha infatti che $\theta\in S_1$ e $y \in \mathbb{R}$ con $S_1$ cerchio di raggio unitario.
\end{exmp}
\noindent
\begin{defn}[Omomorfismo]
    Sia $h: U\to V$ con $U, V \subset \mathbb{R}^n$. Supponiamo che $\exists \ h^{-1}$, allora $h$ è omomorfismo se $h$ e $h^{-1}$ sono entrambe continue.
\end{defn}
\noindent 
\begin{defn}[Diffeomorfismo $C^r$]
    Siano $U, V \subset \mathbb{R}^n$ e $F:U\to V$. Diciamo che $F$ è un diffeomorfismo $C^r$ ($r\ge 1$) se esiste $F^{-1}$ e inoltre sia $F$ che $F^{-1}$ sono entrambe $C^r$ (derivata continua fino all'ordine $r$).
\end{defn}
\noindent
\begin{defn}[Manifold n-dimensionale]
    Sia $M \in \mathbb{R}^N$, diciamo che $M$ è un $K$ Dimensional Manifold se ($K<N$) se:
    \begin{itemize}
	\item $\forall m \in M $ esiste un \textbf{Intorno aperto} $W_i$ di esso con la proprietà: $M = \bigcup_i W_i$.
	\item $\forall W_i$ esiste un omomorfismo $\phi_i: W_i \to V_i \subset \mathbb{R}^N$.
	\item Se $W_i \cap W_j \neq 0$ la mappa $\phi_i \circ \phi_j = c_{ij}$  definita in $\phi_j(W_i \circ W_j)$ è un diffeomorfismo $C^r$ ($r\ge 1$).
    \end{itemize}
\end{defn}
\noindent
Ogni coppia $(W_i, \phi_i)$ è detta \textcolor{red}{Carta}, mentre l'insieme di tutte le carte $\left\{(W_i, \phi_i) \right\}$ è detto \textcolor{red}{Atlante}.\\
La cosa importante è che tramite i funzionali $\phi$ è possibile introdurre le proprietà di differenziabilità sul manifold utilizzando le definizioni di differenziabilità su $\mathbb{R}^N$ (euclidee) che sono ben definite.
\begin{figure}[H]
    \centering
    \fbox{\import{figures/chap1/}{3_2.pdf_tex}}
    \caption{\scriptsize Azione dell'omomorfismo sul manifold.}
    \label{fig:3_2}
\end{figure}
\subsection{Mappare la dinamica di un Manifold in uno spazio euclideo $\mathbb{R}^K$}%
\label{sub:Mappare la dinamica di un Manifold in Rn }
Supponiamo di avere la mappa $G: W_i\to W_i$, ovvero manda punti del sottoinsieme $W_i$ (un intorno del punto $m$) del Manifold in punti di $W_i$.\\
Prendiamo $\vect{x}_1 \in W_i$: $\vect{x}_2 = G(\vect{x}_1)\in W_i$.\\
Possiamo mappare la $G$ in $\mathbb{R}^K$ nel seguente modo:
\[
    \vect{y}_1 = \phi_i(\vect{x}_1); \qquad \vect{y}_2 = \phi_i(\vect{x}_2)
.\] 
I punti $\vect{y}_{1,2}$ appartengono a $\mathbb{R}^K$. Il modo in cui si trasporta la differenziabilità all'interno del manifold è il seguente:
\[
    \vect{y}_2 = \phi_i(G(\vect{x}_1)) = \phi_i(G(\phi_i^{-1}(\vect{y}_1)))
.\] 
Visto che $\phi_i$ e $G$ sono note, che $\phi_i$ è omomorfismo e che $\vect{y}_1, \vect{y}_2 \in \mathbb{R}^K$ abbiamo che le proprietà di diff. sono applicabili ai funzionali sul manifold nello stesso modo in cui le applichiamo su $\mathbb{R}^K$. Inoltre si ha che:
\[
    \v{y}_2 = \phi_i \circ G \circ \phi_i^{-1}(\v{y}_1) 
.\] 
Definisce l'evoluzione dinamica del sistema su $\mathbb{R}^K$ .
