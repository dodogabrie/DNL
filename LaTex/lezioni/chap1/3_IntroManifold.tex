\section{Introduzione ai Manifold}%
\label{sub:Introduzione ai Manifold}
Abbiamo fin'ora affermato che lo stato di un sistema dinamico è descritto da un vettore di $\mathbb{R}^n$, in questa sezione cerchiamo di essere più precisi riguardo a questa quantità.
\begin{exmp}[Pendolo nello spazio delle fasi]
    \[
        \begin{cases}
            \frac{\text{d} \theta}{\text{d} t} = y\\
	    \frac{\text{d} y}{\text{d} t} = -\frac{g}{l}\sin\theta
        \end{cases}
    \] 
    In questo caso abbiamo che lo stato $\vect{x} = (\theta, y)$ non è un vettore di $\mathbb{R}^n$ generico: 
    \begin{itemize}
        \item $\theta$ è un angolo. 
	\item $y$ è una velocità angolare.
    \end{itemize}
    Lo stato è descritto in $\mathbb{R}^2$, la dinamica del sistema giace su una superficie dello spazio delle fasi detto \textbf{Manifold}.\\
    Il manifold per il problema del pendolo è una superficie cilindrica:
    \[
        \theta\in S_1 \qquad y \in \mathbb{R}\qquad \text{Con $S_1$ cerchio}
    .\] 
\end{exmp}
\noindent
Anche se un manifold non coincide con $\mathbb{R}^n$ localmente (sulla varietà) può essere caratterizzato da $\mathbb{R}^n$.
\begin{defn}[Omomorfismo]
    Sia $h: U\to V$ con $U, V \subset \mathbb{R}^n$. Supponiamo che $\exists \ h^{-1}$, allora $h$ è omomorfismo se $h$ e $h^{-1}$ sono entrambe continue.
\end{defn}
\noindent
\begin{defn}[Manifold n-dimensionale]
    Sia $M\subset \mathbb{R}^n$ e $\vect{x}  \in M$, sia $W$ un intorno di $\vect{x}$. Diciamo che $M$ è un manifold $k$-dimensionale ($k<n$ ) se $\exists$ un omomorfismo $h: W\to \mathbb{R}^n$.
\end{defn}
\noindent
In pratica l'omomorfismo manda i punti appartenenti al manifold in un sottoinsieme $U \subset \mathbb{R}^n$. L'insieme $U$, in cui viene mappato l'intorno $W$ di $\vect{x}\in M$ è detto carta del manifold: $U = h(W)$.\\
\begin{defn}[Atlante di un manifold]
Se è possibile costruire per tutti i punti di $M$ un intorno in cui vale l'omomorfismo allora l'insieme $U \subset \mathbb{R}^n$ in cui i punti di $M$ vengono mappati è detto Atlante di $M$.
\end{defn}
\noindent
La cosa importante è che tramite $h$ è possibile introdurre le proprietà di differenziabilità sul manifold utilizzando le definizioni di differenziabilità su $\mathbb{R}^n$ che sono ben definite.
\begin{figure}[H]
    \centering
    \fbox{\import{./figures/}{3_2.pdf_tex}}
    \caption{\scriptsize Azione dell'omomorfismo sul manifold.}
    \label{fig:3_2}
\end{figure}

\subsection{Mappare la dinamica di un Manifold in $R^n$}%
\label{sub:Mappare la dinamica di un Manifold in Rn }
Supponiamo di avere la mappa $G: W\to W$, ovvero manda punti di $W$ (un intorno del punto $\vect{x}  \in M$) in punti di $W$.\\
Prendiamo $\vect{x}_1 \in W$: $\vect{x}_2 = G(\vect{x}_1)\in W$.\\
Possiamo mappare la $G$ in $\mathbb{R}^n$ nel seguente modo:
\[
    \vect{y}_1 = h(\vect{x}_1); \qquad \vect{y}_2 = h(\vect{x}_2)
.\] 
I punti $\vect{y}_{1,2}$ appartengono a $\mathbb{R}^n$. Il modo in cui si trasporta la differenziabilità all'interno del manifold è il seguente:
\[
    \vect{y}_2 = h(G(\vect{x}_1)) = h(G(h^{-1}(\vect{y}_1)))
.\] 
Visto che $h$ e $G$ sono note, che $h$ è omomorfismo e che $\vect{y}_1, \vect{y}_2 \in \mathbb{R}^n$ abbiamo che le proprietà di diff. sono applicabili ai funzionali sul manifold nello stesso modo in cui gli applichiamo su $\mathbb{R}^n$.

