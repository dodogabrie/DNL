\section{Esistenza ed unicità della soluzione di IVP}%
\label{sub:Esistenza ed unicità della soluzione di IVP}
Dato un SD a tempo continuo ed un IVP (initial value problem) vorremmo sapere, per studiare la dinamica, se:
\begin{itemize}
    \item Il problema ha soluzione?
    \item La soluzione, se esiste, è unica?
\end{itemize}
In assenza di unicità il sistema non può essere deterministico. I sistemi dinamici che studiamo devono sempre essere deterministici.
\begin{exmp}[Due soluzioni]
    \[
        \begin{cases}
	    \frac{\text{d} x}{\text{d} t} = 3x^{2 /3} = F(x)\\
	    x(0)=x_0=0
        \end{cases}
    .\] 
    Il sistema non è lineare poiché
    \[
        \left(x+y\right)^{2 /3} \neq x^{2 /3}+y^{2 /3}
    .\] 
    Possiamo subito notare che una prima soluzione è la nulla: $x_1(t)=0$.
    Un'altra soluzione è invece $x_2(t)=t^3$, infatti sostituendo nella equazione per la derivata di $x$:
    \[
	3t^2 = 3(t^3)^{2 /3}
    .\] 
    Che è appunto verificata. \\
    Possiamo notare che $F(x)$ è continua in $x_0$, tuttavia non lo è la sua derivata rispetto a $x$: diverge a $\pm \infty$. Questo fatto è strettamente correlato alla non unicità della soluzione.
\end{exmp}
\noindent
La non unicità della soluzione non è l'unico problema nel caso di sistemi dinamici a tempo continuo, può anche accadere che la soluzione non esista per tutti i tempi $\in \mathbb{R}$.
\begin{exmp}[Soluzione con discontinuità nel tempo]
    \[
        \begin{cases}
	    \frac{\text{d} x}{\text{d} t} = x^2=F(x)\\
	    x(0)=1
        \end{cases}
    \] 
    In questo caso $F(x)$ è derivabile infinite volte e le sue derivata sono sempre continue. Cerchiamo la soluzione:
    \[
	\int\frac{dx}{x} = \int  dt \implies  x(t)=-\frac{1}{t+c}
    .\] 
    Inserendo la condizione iniziale:
    \[
	x(t) = \frac{1}{1-t}
    .\] 
    Notiamo che la soluzione non è continua $\forall t \in \mathbb{R}$, infatti è definita in $]-\infty, 1 [ \ \cup \ ]1, \infty[$.
\end{exmp}
\noindent
La soluzione del problema di Cauchy non deve necessariamente esser definita in tutto $\mathbb{R}$, quello che conta per noi è che sia definita almeno asintoticamente.
\begin{defn}[Funzione $C^r$]
    Una funzione $F(\vect{x}):$
    \[
	F(\vect{x}): \mathbb{R}^n\to \mathbb{R}^n \qquad \vect{x}\in \mathbb{R}^n
    .\] si dice $C^r$ se è $r$ volte derivabile e le derivate fino all'ordine $r$ sono continue.
\end{defn}
\noindent
\begin{thm}[Esistenza locale della soluzione]
   Dato un SD a tempo continuo:
   \[
       \begin{cases}
	   \frac{\text{d} \vect{x}}{\text{d} t} = F(\vect{x}, t)\\
	   \vect{x} (t_0) = \vect{x}_0
       \end{cases}
   \] 
   Con $\left(\vect{x}_0, t_0\right) \in U\times \mathbb{R} \in $. Assumendo che:
   \begin{itemize}
       \item $F(\vect{x},t)$ sia $C^r$ rispetto a $\vect{x}$ con $r\ge 1$.
       \item $F(\vect{x}, t)$ continua in $t$.
   \end{itemize}
   Allora esiste un intorno di $t_0$ ($t_0-\epsilon  < t < t_0+\epsilon$) nel quale la soluzione dell'IVP esiste ed è unica.
\end{thm}
\noindent
Questo teorema è locale poiché ci assicura una soluzione in un intervallo temporale, non asintoticamente.\\
Alcuni libri sostituiscono la richiesta di avere $F(\vect{x}, t)$ funzione $C^r$ con la richiesta che quest'ultima funzione sia Lipschitziana:
\[
    \left|F(\vect{x}, t)-F(\vect{y}, t)\right| \le k \left|\vect{x}-\vect{y}\right|
.\] 
In cui se $k$  è una quantità indipendente dal punto $\vect{x}$  considerato allora si ha una ed una sola soluzione all'IVP.
\begin{ex}[Esercizio]
    Studiare al variare del parametro $x_0$  il seguente IVP:
    \[
        \begin{cases}
            \frac{\text{d} x}{\text{d} t} = x^2\\
	    x(0) = x_0
        \end{cases}
    \] 
\end{ex}
\noindent
\begin{ex}[Esercizio]
    Studiare al variare del parametro $a$ il seguente IVP:
    \[
        \begin{cases}
            \frac{\text{d} x}{\text{d} t} = \sqrt{x} \\
	    x(0)=a
        \end{cases}
    \] 
\end{ex}
\noindent
\begin{thm}[Esistenza Globale della soluzione]
    Supponiamo di avere il sistema di equazioni differenziali:
    \[
	\frac{\text{d} \vect{x}}{\text{d} t} = F(\vect{x},t); \qquad \vect{x}(0) = \vect{x}_0
    .\] 
    Con le quantità definite nei seguenti intervalli:
    \[
        \vect{x}\in \mathbb{R}^n; \qquad t \in [a,\infty[; \qquad F: \mathbb{R}^n\times [a,\infty[ \ \to \mathbb{R}^n
    .\] 
    Se valgono i due seguenti:
    \begin{itemize}
        \item $F$ è $C^r$ con $r\ge 1$ e continua in $t$.
	\item $\exists \ h(t), k(t)$ con $\left[h, k > 0 \ \forall \ t\right]$ tali che:
	    \[
		\left|F(\vect{x}, t)\right|\le h(t)\left|\vect{x}\right|+k(t); \qquad \text{per } \vect{x}, t \in \mathbb{R}^n\times [a, \infty[
	    .\] 
    \end{itemize}
    Allora esiste ed è unica la soluzione dell'IVP definito in $\mathbb{R}^n\times [a,\infty[$.
\end{thm}
\noindent
\begin{exmp}[Applicazione del teorema]
   \[
       \begin{cases}
	   \frac{\text{d} x}{\text{d} t} = \frac{3t^2x(t)}{1+x(t)^2}+x(t) = F(x,t)\\
	   x(t_0)=x_0
       \end{cases}
   \]  
   La soluzione esiste? \'E unica? \\
   La funzione $F$  sicuramente è almeno $C^1$  in $x$  ed è continua in $t$, quindi sicuramente la soluzione esiste almeno in un intorno del punto iniziale ed è unica sempre in questo intorno.\\
   Per l'esistenza ed unicità globali invece è necessario qualche altro passaggio algebrico:
   \[
       \left|F(x,t)\right| = \left|\frac{3t^2x(t)}{1+x^2(t)} + x(t)\right|\le \left|x\right|+\left|\frac{3t^2x}{1+x^2}\right| \le
			    \left|x\right|\left|3t^2+1\right|
   .\] 
   Quindi scegliendo le funzioni:
   \[
       k(t)=0 \qquad h(t)=3t^2+1
   .\] 
   Abbiamo che le ipotesi del teorema di esistenza globale sono rispettate, quindi la soluzione esiste globalmente (asintoticamente).\\
   Un ulteriore esercizio (per il lettore) è quello di dimostrare che $x(t)$ non diverge per $t\to \infty$. Un suggerimento: moltiplicare l'equazione differenziale a destra e sinistra per $2x$, scrivere la nuova eq. differenziale per $x^2$ e minorare la $F(x^2)$\ldots 
\end{exmp}
\noindent
\begin{defn}[Sistema deterministico]
    Un SD a tempo continuo descritto da
    \[
	\frac{\text{d} \vect{x}}{\text{d} t} = F(\vect{x},t); \qquad \vect{x} (t_0)=\vect{x}_0
    .\] si dice deterministico se esiste ed è unica la corrispondente soluzione dell'IVP.
\end{defn}
\noindent

