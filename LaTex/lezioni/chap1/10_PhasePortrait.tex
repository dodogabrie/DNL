\section{Phase Portrait}%
\label{sub:Phase Portrait}
Si definisce Phase Portrait (PP) una determinata collezione di orbite nello spazio delle fasi. \\
Possiamo dire che il PP è una specie di arte: per fare un buon PP è necessario selezionare le orbite significative del sistema, quelle che esprimono al meglio tutta la possibile dinamica che il sistema può presentare.
\begin{exmp}[Oscillatore armonico]
    Partiamo da un esempio semplice di PP:
    \[\begin{dcases}
        \frac{\text{d} x}{\text{d} t} = y \\
	\frac{\text{d} y}{\text{d} t} = -x
    \end{dcases}\] 
    Espresso in termini di campo vettoriale:
    \[
     \vect{V}=\begin{pmatrix} x \\ y \end{pmatrix};
    \ F(\vect{V})=\begin{pmatrix} y \\ -x \end{pmatrix}  = \begin{pmatrix} F_1(\vect{V}) \\  F_2(\vect{V}) \end{pmatrix} 
    \] 
    Il sistema presenta la seguente legge di conservazione:
    \marginpar{
        \captionsetup{type=figure}
            \incfig{10_1}
        \caption{\scriptsize Phase Portrait per l'oscillatore armonico.}
        \label{fig:10_1}
    }
    \[
	\frac{\text{d} }{\text{d} t} (x^2 + y^2) = 0 \qquad \forall \ \vect{V}_0 \in \mathbb{R}^2
    \] 
    è possibile dimostrarlo semplicemente esplicitando le derivate ed inserendo le equazioni del moto.\\
    Questa legge di conservazione ci permette di concludere subito che le orbite descritte dal sistema nello spazio delle fasi sono circonferenze centrate nell'origine.
    \[
        x^2+y^2=\text{cost} = r_0^2 \qquad r_0 = \sqrt{x_0^2+y_0^2} 
    \] 
    La direzione di rotazione è data dai segni nel campo vettoriale, ad esempio scegliendo $(x_0, y_0) = (1, 0)$ si vede che il campo è: $F_0 = (0, -1)$: rotazione antioraria.\\
    Un'altra riprova del fatto che le orbite sono circonferenze è il fatto che il campo vettoriale è sempre tangente al vettore $\vect{V}$ nello spazio delle fasi:
    \[
	\begin{pmatrix} -x \\ y \end{pmatrix} \cdot \begin{pmatrix} x \\ y \end{pmatrix} = 0 \qquad \forall \ (x, y) \in \mathbb{R}^2
    \] 
\end{exmp}
\noindent
\begin{exmp}[Oscillatore di Duffling (semplificato)]
    Prendiamo il sistema descritto dalle seguenti equazioni:
    \[\begin{dcases}
        \frac{\text{d} x}{\text{d} t} = y \\
	\frac{\text{d} y}{\text{d} t} = x - x^3 - b y
    \end{dcases}\] 
    Questo rappresenta una semplificazione dell'oscillatore di Duffling, nel sistema originale si ha in più una forzante periodica.\\
    Il sistema presenta due punti che "arrestano la dinamica", ovvero ci sono delle condizioni iniziali per il quale vale che:
    \[
	\frac{\text{d} \vect{x}}{\text{d} t} = 0 \qquad \forall \vect{x}  = (x, y) \in \mathbb{R}^2
    \] 
    Infatti scegliendo $\vect{x}_0 = (1, 0)$ oppure $\vect{x}_0 = (-1, 0)$ entrambe le equazioni differenziali si annullano.\\
    Questi punti sono detti \textit{punti fissi}, gli approfondiremo nelle prossime sezioni.
\end{exmp}
\noindent
