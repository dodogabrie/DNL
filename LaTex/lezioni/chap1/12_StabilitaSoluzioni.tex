\section{Stabilità delle soluzioni}%
\label{sub:Stabilità delle soluzioni}
Quando si parla di stabilità di un sistema si intende la stabilità rispetto ad una perturbazione esterna, osservandone il comportamento dopo la perturbazione.\\
Inoltre si parla di stabilità sempre in un contesto asintotico: serve che il sistema sia definito in $t\to \infty$. Alcuni teoremi che ci garantisce l'esistenza della soluzione asintotica per sistemi non autonomi sono i seguenti.
\begin{thm}[Bounded Global Existence]
    Preso un sistema dinamico:
    \[\begin{dcases}
        \frac{\text{d} \vect{x}}{\text{d} t} = F(\vect{x}) \\
	\vect{x} (0)=\vect{x}_0
    \end{dcases}\] 
    \[
	\vect{x}\in \mathbb{R}^n \ F: \mathbb{R}^n \to \mathbb{R}^n  
    \] 
    Se valgono le seguenti condizioni:
    \begin{enumerate}
        \item $F$ è localmente Lipschitziana: $(\forall \ \vect{y}: \ \left|F(\vect{x})- F(\vect{y})\right|\le k(\vect{x})\left|\vect{x}-\vect{y}\right|)$
	\item $F$ è limitata: $\exists \ M >0 : \left|F(\vect{x})\right|\le M$.
    \end{enumerate}
    Allora la soluzione dell'IVP è globalmente definita.
\end{thm}
\noindent
\begin{thm}[Esistenza Globale della soluzione]
    Se $F$ è globalmente Lipshitziana ($k(\vect{x})=k$ indipendente da $\vect{x}$) allora la soluzione è globalmente definita.
\end{thm}
\noindent
\begin{defn}[Stabilità secondo Lyapunov]
    \marginpar{
        \captionsetup{type=figure}
            \incfig{12_1}
        \caption{\scriptsize Concettualmente la soluzione e stabile se tutte le altre soluzioni con diverse condizioni iniziali nel suo intorno rimangono nel tubo di flusso nel tempo.}
        \label{fig:12_1}
    }
    Dato il sistema dinamico a tempo continuo autonomo: $\frac{\text{d} \vect{x}}{\text{d} t} = F(\vect{x})$, $\vect{x}\in \mathbb{R}^n$, $F: \mathbb{R}^n\to \mathbb{R}^n$ e sia $\vect{x}_p(t)$ una soluzione dell'IVP.\\
    Diciamo che $\vect{x}_p(t)$ è \textbf{stabile secondo Lyapunov} se
    \[
    \forall \ \epsilon  > 0 \quad \exists \ \delta (\epsilon)>0:
    \] 
    \[
	\text{se } \left|\left|\vect{x}(0)-\vect{x}_p(0)\right|\right|< \delta (\epsilon) \implies  
	\left|\left|\vect{x} (t)-\vect{x}_p(t)\right|\right|<\epsilon  \quad \forall \ t >0
    \] 
    Con $\vect{x} (t)$ soluzione dell'IVP.
\end{defn}
\noindent
\begin{defn}[Stabilità asintotitca]
    \marginpar{
        \captionsetup{type=figure}
            \incfig{12_2}
        \caption{\scriptsize Concettualmente se la soluzione è asintotica allora tutte le soluzioni nell'intorno cadono in essa per $t\to \infty$.}
        \label{fig:12_2}
    }
    Nelle stesse ipotesi della precedente definizione diciamo che $\vect{x}_p(t)$ (soluzione di riferimento) è \textbf{asintoticamente stabile} se è:
    \begin{enumerate}
        \item Stabile secondo Lyapunov.
	\item $\lim\limits_{t \to \infty} \left|\left|\vect{x}(t)-\vect{x}_p(t)\right|\right| = 0$.
    \end{enumerate}
\end{defn}
\noindent
\begin{exmp}[Sistema stabile ma non stabile asintoticamente]
    \[\begin{dcases}
        \frac{\text{d} \vect{x}}{\text{d} t} = 0\\
	\vect{x}(0)=\vect{x}_0
    \end{dcases}\] 
    La soluzione è banalmente: $\vect{x}_p = \vect{x}_0$. Questa soluzione è stabile secondo Lyapunov:\\
    Presa un'altra soluzione $\vect{x}$ tale che $\vect{x} (0)=\vect{x}_0$.
    \[
	\epsilon >0 \quad \left|\left|\vect{x}_p(t) - \vect{x} (t)\right|\right|=
	\left|\left|\vect{x}_0 - \vect{y}_0 \right|\right| < \epsilon
    \] 
    Allora basta prendere $\delta (\epsilon) = \epsilon$.\\
    Notiamo che questa soluzione non è asintoticamente stabile, le due soluzioni restano sempre distanti nel tempo.
\end{exmp}
\noindent

\subsection{Stabilità a tempo discreto}%
\label{sub:Stabilità a tempo discreto}
La cosa particolare della seconda definizione è che condizione 2) non basta per dire che una soluzione sia stabile anche secondo Lyapunov.
\begin{defn}[Stabilità per SD a tempo discreto autonomi secondo Lyapunov]
    \marginpar{
        \captionsetup{type=figure}
            \incfig{12_3}
	    \caption{\scriptsize L'idea è che una soluzione che si trovi ad un certo punto a meno di $\delta (\epsilon)$ di distanza da $\vect{x}_s$ nel tempo iniziale non è in grado di uscire dalla bolla di raggio $\epsilon$.}
        \label{fig:12_3}
    }
    Data la mappa
    \[
	\vect{x}_{k+1}=G(\vect{x}_k) \ \text{con} \vect{x}_k \in \mathbb{R}^n \ \text{e } G: \mathbb{R}^n\to \mathbb{R}^n.
    \] 
    Diciamo che un orbita\sidenote{\scriptsize $\left\{\vect{u}_k\right\}$ Inteso come insieme di valori} $\left\{\vect{u}_k\right\}$ è \textbf{stabile secondo Lyapunov} se 
    \[
	\forall \epsilon > 0 \ \exists \delta (\epsilon)>0:
    \] 
    Per ogni altra orbita $\vect{V}_k$ per la quale vale che:
    \[
	\left|\left|\vect{V}_m - \vect{u}_m\right|\right|< \delta (\epsilon), \ m \in \mathbb{N} \implies 
        \left|\left|\vect{V}_K - \vect{u}_K\right|\right|<\epsilon \quad \forall K > m
    \]
\end{defn}
\noindent
\begin{defn}[Stabilità asintotica di SD a tempo discreto]

    Nelle stesse ipotesi della definizione precedente, diciamo che l'orbita $\left\{\vect{u}_k\right\}$ è asintoticamente stabile se:
    \begin{enumerate}
        \item è stabile secondo Lyapunov.
	\item $\lim_{k \to \infty} \left|\left|\vect{V}_k-\vect{u}_k\right|\right|=0$ per ogni altra orbita $\vect{V_k}$ nell'intorno $\delta (\epsilon)$.
    \end{enumerate}
\end{defn}
\noindent
\subsection{Stabilità secondo Lyapunov di stati stazionari di SD a tempo continuo}%
\label{sub:Stabilità secondo Lyapunov di stati stazionari di SD a tempo continuo}
Sia $\frac{\text{d} \vect{x}}{\text{d} t} = F(\vect{x})$  con $\vect{x}_s$ stazionario ($F(\vect{x}_s) = 0$). Ricordiamo che lo stato stazionario è anch'esso una soluzione.
\begin{defn}[Stabilità di stato stazionario secondo Lyapunov]
    Si dice che $\vect{x}_s$ è stabile secondo Lyapunov se
    \[
	\forall \epsilon  > 0 \exists \ \delta (\epsilon) > 0:
    \] 
    \[
	\text{se } \left|\left|\vect{x}_0-\vect{x}_s\right|\right|< \delta (\epsilon) \implies  \left|\left|\vect{x} (t)-\vect{x}_s\right|\right| < \epsilon
    \] 
\end{defn}
\noindent
\begin{defn}[Stabilità asintotica di stato stazionario]
    Nelle stesse ipotesi della definizione precedente diciamo che $\vect{x}_s$ è asintoticamente stabile se 
    \begin{enumerate}
        \item $\vect{x}_s$ è stabile secondo Lyapunov.
	\item $\lim\limits_{t \to \infty} \left|\left|\vect{x} (t)-\vect{x}_s\right|\right|=0$.
    \end{enumerate}
\end{defn}
\noindent
\begin{ex}[Oscillatore armonico]
    Dato il sistema dinamico a tempo continuo
    \[\begin{dcases}
        \frac{\text{d} x}{\text{d} t} = y\\
	\frac{\text{d} y}{\text{d} t} = -x
    \end{dcases}\] 
    Dimostrare che $\vect{V}_s= \begin{pmatrix} 0\\ 0 \end{pmatrix} $ è stabile secondo Lyapunov e dire se tale soluzione è asintoticamente stabile.
\end{ex}
\noindent
\begin{exmp}[Vinograd]
In un articolo di Vinograd del 1957 (ineguality of the method of deterministic experiments for the study of nonlinear diff. equations) è stato dimostrato che il seguente sistema:
\[\begin{dcases}
    \frac{\text{d} x}{\text{d} t} = x^2(y-x)+y^5 = F_1(x, y)\\
    \frac{\text{d} y}{\text{d} t} = y^2(y-2x)= F_2(x, y)
\end{dcases}\] 
Che l'unico stato stazionario è $\vect{V}_s = \begin{pmatrix} 0 \\ 0 \end{pmatrix}$ (per casa)  e che $\vect{V}_s$  è tale che:
\[
    \lim_{t \to \infty} \left|\left|\vect{V} (t)-\vect{V}_s\right|\right|=0
\] 
Quindi lo stato stazionario è asintoticamente stabile ma non è stabile secondo Lyapunov.
\end{exmp}
\noindent
\begin{ex}[Stabilità soluzione]
    Dato il SD $\frac{\text{d} \vect{x}}{\text{d} t} = F(\vect{x})$ con $\vect{x}\in \mathbb{R}^n$ e $F: \mathbb{R}^n\to \mathbb{R}^n$.\\
    Assumiamo che $\exists \ \alpha, \beta$ con $(\beta >0)$:
    \[
	F(\vect{x})\cdot \vect{x}\le \alpha\left|\vect{x}\right|^2+ \beta
    \] 	
    \begin{itemize}
        \item Dimostrare che le soluzioni sono globalmente definite.
	\item Dimostrare, nel caso $\alpha <0$, che esiste $r$ (raggio di una palla in $\mathbb{R}^n$) e $T$ tali per cui se $t>T$ allora $\left|\vect{x} (t)\right|<r$.
	\item Determinare $r$.
    \end{itemize}

\end{ex}
\noindent
\subsection{Esempi sulla stabilità secondo Lyapunov di soluzioni di ODE}%
\label{sub:Esempi sulla stabilità secondo Lyapunov di soluzioni di ODE}
\begin{exmp}[1]
    Prendiamo il campo vettoriale definito come:
    \[\begin{dcases}
        \frac{\text{d} \vect{x}}{\text{d} t} = - a^2x \qquad a \in \mathbb{R}, a\neq 0\\
	x(0)=x_0	
    \end{dcases}\] 
    La soluzione dell'IVP è: $x_p(t)= x_0e^{-a^2t}$. Abbiamo inoltre lo stato stazionario nullo: $x_s = 0$. \\
    Dimostriamo che la soluzione stazionaria è stabile secondo Lyapunov, ovvero che:
    \[
	\forall \epsilon>0  \  \text{ (assegnato)}: \ \left|x(t)-x(s)\right|<\epsilon
    \] 
    \[
	\left|x(t)-x_s\right|= \left|x_0e^{-a^2t}-0\right| = e^{-a^2t}\left|x_0\right| \le \left|x_0\right|
    \] 
    Basta allora prendere $\delta (\epsilon)=\epsilon$:
    \[
	\implies  \ \forall x_0: \ \left|x_0\right|<\delta (\epsilon) \to \left|x(t)-x_s\right|<\epsilon
    \] 
    Quindi la soluzione è stabile secondo Lyapunov. Inoltre:
    \[
	\left|x(t)-x_s\right|=\left|x_0e^{-a^2t}\right|\to 0 \text{ con } t\to \infty
    \] 
    Allora $x_s$ è anche asintoticamente stabile.
\end{exmp}
\noindent
\begin{ex}[2]
    Prendiamo il campo vettoriale non autonomo\sidenote{\scriptsize Tutte le definizioni sono analoghe, l'unica modifica da tenere in considerazione è che l'intorno del punto iniziale (il tubo) può essere dipendente dal tempo.}:
    \[
        \frac{\text{d} x}{\text{d} t} = - tx
    \] 
    Dimostrare (per casa) che $x(t)= x_0\exp\left(-\frac{1}{2}(t^2-t_0^2)\right)$ soddisfa l'IVP:
    \[\begin{dcases}
        \frac{\text{d} x}{\text{d} t} = -tx\\
	x(t_0)= x_0
    \end{dcases}\] 
    Dimostriamo adesso che la soluzione di riferimento:
    \[
        x_p(t)= x_p\exp\left(-\frac{1}{2}(t^2-t_0^2)\right)
    \] 
    è stabile secondo Lyapunov.
    \[
	\left|x(t)-x_p(t)\right|= \left|x_0e^{-1 /2(t^2-t_0^2)} - x_pe^{-1 /2(t^2-t_0^2)}\right|
    \] 
    Se $t\ge t_0$ $\implies$ $\exp \left(- \frac{1}{2}(t^2-t_0^2)\right)<1$ Quindi:
    \[
	\left|x(t)-x_p(t)\right|\le \left|x_0-x_p\right|< \epsilon  \implies  \delta (\epsilon)=\epsilon
    \] 
    La soluzione è anche asintoticamente stabile.
    \[
	\lim_{t \to \infty} \left|x(t)-x_p(t)\right| = 0
    \] 
\end{ex}
\noindent
\begin{exmp}[3]
    Dato il campo vettoriale in $\mathbb{R}^2$:
    \[\begin{dcases}
	\frac{\text{d} x}{\text{d} t} = x-10y = F_1(x,y)\\
	\frac{\text{d} y}{\text{d} t} = 10x-y = F_2(x, y)
    \end{dcases}\] 
    L'unico stato stazionario è dato da:
    \[\begin{dcases}
        x + 10 y = 0\\
	10 x - y = 0
    \end{dcases}
    \implies  \vect{V}_s P \begin{pmatrix} 0 \\ 0 \end{pmatrix} 
\]  
Per casa trovare la generica soluzione dell'IVP con 
\[
    x(0)= x_0, \quad y(0)=y_0
\] Verificare che la soluzione è:
\[\begin{aligned}
    x(t)=e^{-t}\left[x_0\cos (10t)-y_0\sin (10t)\right]\\
    y(t)=e^{-t}\left[x_0\sin (10t)+y_0\cos (10t)\right]
.\end{aligned}\]
Vogliamo dimostrare che $\vect{V}_s$ è stabile secondo Lyapunov.
\[\begin{aligned}
    \left|\vect{V} (t)-\vect{V}_s\right| =& e^{-t}\left[x_0^2\cos^2(10t)+ \right.\\ 
					  &+ y_0^2\sin^2(10t)-2x_0y_0\sin (10t)\cos (10t) + \\
					  &+ x_0^2\sin^2(10t)+ y_0^2\cos^2(10t)+\\ 
					  &\left.+ 2x_0y_0\sin (10t)\cos (10t)\right] = \\
					  &= e^{-t}\left[x_0^2+ y_0^2\right]^{1 /2}
.\end{aligned}\]
Visto che stiamo considerando $t>0$ allora:
\[
    \left|\vect{V} (t)-\vect{V}_s\right| <\epsilon  \text{ se } \delta (\epsilon)=\epsilon  \text{ e } \left[x_0^2+y_0^2\right]^{1 /2}<\epsilon
\] 
Quindi $\vect{V}_s$ è stabile secondo Lyapunov.\\
Osserviamo anche che $\vect{V}_s$ è anche asintoticamente stabile:
\[
    \lim_{t \to \infty} \left|\vect{V} (t)-\vect{V}_s\right|=0
\] 
\end{exmp}
\noindent
La definizione di stabilità secondo Lyapunov ha una utilità concettuale poiché ci da una idea di quello che è un punto stabile, a livello applicativo non è banale dimostrare la stabilità: in moltissimi casi nemmeno conosciamo le soluzioni!\\
Quello che è possibile fare è sviluppare una teoria sui SD a tempo continuo (e discreto) che ci consente di testare la definizione di Lyapunov in modo diretto.
