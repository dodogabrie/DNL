\section{Introduzione ai Manifold}%
\label{sub:Introduzione ai Manifold}
Abbiamo fin'ora affermato che lo stato di un sistema dinamico è descritto da un vettore di $\mathbb{R}^n$, in questa sezione cerchiamo di essere più precisi riguardo a questa quantità.
\begin{exmp}[Pendolo nello spazio delle fasi]
    \[
        \begin{cases}
            \frac{\text{d} \theta}{\text{d} t} = y\\
	    \frac{\text{d} y}{\text{d} t} = -\frac{g}{l}\sin\theta
        \end{cases}
    \] 
    In questo caso abbiamo che lo stato $\vect{x} = (\theta, y)$ non è un vettore di $\mathbb{R}^n$ generico: 
    \begin{itemize}
        \item $\theta$ è un angolo. 
	\item $y$ è una velocità angolare.
    \end{itemize}
    Lo stato è descritto in $\mathbb{R}^2$, la dinamica del sistema giace su una superficie dello spazio delle fasi detto \textbf{Manifold}.\\
    Il manifold per il problema del pendolo è una superficie cilindrica:
    \[
        \theta\in S_1 \qquad y \in \mathbb{R}\qquad \text{Con $S_1$ cerchio}
    .\] 
\end{exmp}
\noindent
Anche se un manifold non coincide con $\mathbb{R}^n$ localmente (sulla varietà) può essere caratterizzato da $\mathbb{R}^n$.
\begin{defn}[Omomorfismo]
    Sia $h: U\to V$ con $U, V \subset \mathbb{R}^n$. Supponiamo che $\exists \ h^{-1}$, allora $h$ è omomorfismo se $h$ e $h^{-1}$ sono entrambe continue.
\end{defn}
\noindent
\begin{defn}[Manifold n-dimensionale]
    Sia $M\subset \mathbb{R}^n$ e $\vect{x}  \in M$, sia $W$ un intorno di $\vect{x}$. Diciamo che $M$ è un manifold $k$-dimensionale ($k<n$ ) se $\exists$ un omomorfismo $h: W\to \mathbb{R}^n$.
\end{defn}
\noindent
In pratica l'omomorfismo manda i punti appartenenti al manifold in un sottoinsieme $U \subset \mathbb{R}^n$. L'insieme $U$, in cui viene mappato l'intorno $W$ di $\vect{x}\in M$ è detto carta del manifold: $U = h(W)$.\\
\begin{defn}[Atlante di un manifold]
Se è possibile costruire per tutti i punti di $M$ un intorno in cui vale l'omomorfismo allora l'insieme $U \subset \mathbb{R}^n$ in cui i punti di $M$ vengono mappati è detto Atlante di $M$.
\end{defn}
\noindent
La cosa importante è che tramite $h$ è possibile introdurre le proprietà di differenziabilità sul manifold utilizzando le definizioni di differenziabilità su $\mathbb{R}^n$ che sono ben definite.
\begin{figure}[H]
    \centering
    \fbox{\import{./figures/}{3_2.pdf_tex}}
    \caption{\scriptsize Azione dell'omomorfismo sul manifold.}
    \label{fig:3_2}
\end{figure}
\subsection{Mappare la dinamica di un Manifold in $R^n$}%
\label{sub:Mappare la dinamica di un Manifold in Rn }
Supponiamo di avere la mappa $G: W\to W$, ovvero manda punti di $W$ (un intorno del punto $\vect{x}  \in M$) in punti di $W$.\\
Prendiamo $\vect{x}_1 \in W$: $\vect{x}_2 = G(\vect{x}_1)\in W$.\\
Possiamo mappare la $G$ in $\mathbb{R}^n$ nel seguente modo:
\[
    \vect{y}_1 = h(\vect{x}_1); \qquad \vect{y}_2 = h(\vect{x}_2)
.\] 
I punti $\vect{y}_{1,2}$ appartengono a $\mathbb{R}^n$. Il modo in cui si trasporta la differenziabilità all'interno del manifold è il seguente:
\[
    \vect{y}_2 = h(G(\vect{x}_1)) = h(G(h^{-1}(\vect{y}_1)))
.\] 
Visto che $h$ e $G$ sono note, che $h$ è omomorfismo e che $\vect{y}_1, \vect{y}_2 \in \mathbb{R}^n$ abbiamo che le proprietà di diff. sono applicabili ai funzionali sul manifold nello stesso modo in cui gli applichiamo su $\mathbb{R}^n$.
\section{Spazio delle fasi esteso (SD a tempi continui)}%
\label{sub:Spazio delle fasi esteso (SD a tempi continui)}
Si prende un sistema dinamico a tempi continui autonomo e lo si perturba con una componente dipendente dal tempo (un fattore esterno). Il sistema in questo modo diventa non autonomo, l'equazione generale che regola questo tipo di sistema è:
\[
    \frac{\text{d} \vect{x}}{\text{d} t} = F(\vect{x},t) \qquad \vect{x}\in \mathbb{R}^n; \ F: \mathbb{R}^n \times \mathbb{R}\to \mathbb{R}^n
.\] 
Possiamo ricondurre questo sistema ad un sistema autonomo tramite una trasformazione nella variabile temporale:
\[
    t = m(s) = s \implies  \frac{\text{d} }{\text{d} t} = \frac{\text{d} s}{\text{d} t} \frac{\text{d} }{\text{d} s} 
.\] 
Inserendo nella equazione del moto:
\[
    \frac{\text{d} \vect{x}}{\text{d} t} = \frac{\text{d} s}{\text{d} t} \frac{\text{d} \vect{x}}{\text{d} s} = F(\vect{x}, t)
.\] 
Possiamo definire il differenziale di $t$  rispetto a $s$: $dt /ds = 1$.
\[
    \begin{cases}
	\frac{\text{d} \vect{x} (s)}{\text{d} s} = F(\vect{x}, t)\\
	\frac{\text{d} t}{\text{d} s} = 1
    \end{cases}
\] 
\begin{defn}[Spazio delle fasi esteso]
    Si definisce spazio delle fasi esteso la quantità:
    \[
	\vect{y} =(\vect{x}, t) \in \mathbb{R}^n \times \mathbb{R}
    .\] 
\end{defn}
\noindent
In questo modo, definendo anche il funzionale esteso:
\[
    H = (F(\vect{x}, t), 1)
.\] 
Si possono generalizzare le equazioni del moto come:
\[
    \frac{\text{d} \vect{y}}{\text{d} s} = H(\vect{y})
.\] 
Per quanto il problema sia formalmente risolto si deve tenere in considerazione che il nuovo spazio delle fasi potrebbe non essere più un compatto.\\
Questa mancanza potrebbe diventare un problema nei nostri scopi in quanto siamo spesso interessati alla soluzione asintotica del sistema (che potrebbe smettere di esistere).\\
In ogni caso aggiungiamo che, se la forzante è periodica, il sistema può essere sempre gestito con questo metodo.
\begin{exmp}[Forzante oscillante]
    \[
	\frac{\text{d} ^2x}{\text{d} t^2} = -x + A\sin (\omega t)
    .\] 
    Come sempre si riporta l'equazione ad una di primo ordine:
    \[
        \begin{cases}
            \frac{\text{d} x}{\text{d} t} = y \\
	    \frac{\text{d} y}{\text{d} t} = -x + A \sin (\omega t)
        \end{cases}
    \] 
    Adesso si introduce la variabile $\theta (t)=\omega t$. Il nuovo sistema, con questa variabile, è descritto nello spazio delle fasi generalizzato e le equazioni sono le seguenti:
    \[
        \begin{cases}
            \frac{\text{d} x}{\text{d} t} = y\\
	    \frac{\text{d} x}{\text{d} t} = -x + A\sin\theta\\
	    \frac{\text{d} \theta}{\text{d} t} =\omega
        \end{cases}
    \] 
    Si noti che la variabile $\theta$ non è limitata, quindi lo spazio delle fasi non è più un compatto.
\end{exmp}
\noindent
\subsection{Flusso di fase}%
\label{sub:Flusso di fase}
Dato un sistema dinamico a tempo continuo in $\mathbb{R}^2$:
\[
    \frac{\text{d} \vect{x}}{\text{d} t} = A\vect{x}
.\] 
\[
    \vect{x} = \begin{pmatrix} x_1 \\ x_2 \end{pmatrix}  \qquad A = \begin{pmatrix} - \Gamma  & 0 \\ 0 & \Gamma \end{pmatrix}; \ \Gamma  \in \mathbb{R}
.\] 
Studiamone l'evoluzione risolvendo il problema alle condizioni iniziali:
\[
    \begin{cases}
        \frac{\text{d} x_1}{\text{d} t} = -\Gamma  x_1\\
	\frac{\text{d} x_2}{\text{d} t} = \Gamma x_2\\
	\vect{x} (0)=\vect{x}_0
    \end{cases}
\] 
La soluzione può essere espressa tramite il seguente vettore:
\[
    \vect{x} (t)= 
    \begin{pmatrix}  
	x_{10}e^{-\Gamma t}\\
	x_{20}e^{\Gamma t}
    \end{pmatrix} 
.\] 
Oppure possiamo scriverla in termini di matrice:
\[
    \vect{x} (t)=
    \begin{pmatrix} 
    e^{-\Gamma t} & 0 \\
    0 & e^{\Gamma t}
    \end{pmatrix} 
    \begin{pmatrix} 
	x_{10}\\
	x_{20}
    \end{pmatrix} 
    \equiv \varphi_t (\vect{x}_0)
.\] 
\begin{defn}[Flusso di fase]
    L'operatore $\varphi_t$ definito come
    \[
        \varphi_t: \mathbb{R}^2 \to \mathbb{R}^2; \quad 
	\varphi_t = 
	\begin{pmatrix} 
	    e^{-\Gamma t} & 0 \\
	    0 & e^{\Gamma t} 
        \end{pmatrix} 
    .\] 
    Si dice flusso di fase del sistema.
\end{defn}
\noindent
\paragraph{Proprietà del flusso di fase}%
\label{par:Proprietà del flusso di fase}
\begin{enumerate}
    \item $\varphi_t(\vect{x}_0)$ è una soluzione dell'IVP.
    \item $\varphi_0(\vect{x}_0) = \vect{x}_0$ 
    \item $\varphi_{t+s}(\vect{x}_0)=\varphi_t(\varphi_s(\vect{x}_0))$ 
\end{enumerate}
\begin{ex}[Sul flusso di fase]
    Verificare la validità delle 3 proprietà per:
    \[
        \varphi_t = 
	\begin{pmatrix} 
	    e^{-\Gamma t} & 0 \\
	    0 & e^{\Gamma t} 
        \end{pmatrix} 
    .\] 
\end{ex}
\noindent
Notiamo che se $\varphi_t$  è invertibile allora il suo inverso è $\varphi_{-t}$.  
\begin{exmp}[Flusso unodimensionale]
    \[
        \begin{cases}
            \frac{\text{d} x}{\text{d} t} = x^2-1\\
	    x(0) = x_0
        \end{cases}
    .\] 
    Prima di ricavare il flusso di fase determiniamo la soluzione:
    \[
        \frac{dx}{x^2-1} = dt \implies  dx \left[\frac{1}{x-1}-\frac{1}{x+1}\right] = dt
    .\] 
    Integrando a destra e sinistra:
    \[
	\log (\frac{\left|x-1\right|}{\left|x+1\right|}) = 2t + c
    .\] 
    Per ricavare $x(t)$  è necessario uno studio di funzione all'interno del logaritmo per capire quando è necessaria una inversione di segno nel suo argomento.\\
    Per $\left|x\right| > 1$  l'argomento è positivo, possiamo procedere in tal caso a risolvere con l'elevamento a potenza:
    \[
        \frac{x-1}{x+1}=e^{2t}B
    .\] 
    La costante $B$  si determina imponendo la condizione iniziale $x(0)=x_0$:
    \[
        B=\frac{x_0-1}{x_0+1}
    .\] 
    In conclusione la soluzione è:
    \[
	x(t) = \frac{(x_0+1)+e^{2t}(x_0-1)}{(x_0+1) -e^{2t}(x_0-1)} = \varphi_t(x_0)
    .\] 
    In questo caso abbiamo un flusso che non è rappresentato da una matrice ma da un funzionale. Possiamo dimostrare che è un flusso: le prime due richieste sono ovvie. La terza invece è lasciata per esercizio, si tratta di fare tanti conti.
\end{exmp}
\noindent
