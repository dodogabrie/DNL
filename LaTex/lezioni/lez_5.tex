\subsection{Flusso di Fase per sistemi non autonomi}%
\label{sub:Flusso di Fase per sistemi non autonomi}
Prendiamo nuovamente la definizione di Flusso partendo dal solito sistema:
\[
    \begin{dcases}
	\frac{\text{d} \vect{x}}{\text{d} t} = F(\vect{x})\\
	\vect{x} (0)=\vect{x_0}
    \end{dcases}
    \qquad 
    F \in C^r, \ F:\mathbb{R}^n\to \mathbb{R}^n; \ \vect{x}_0 \in \mathbb{R}^n
.\] 
Possiamo caratterizzare la soluzione tramite il funzionale flusso:
\[
    \varphi (t,\vect{x}): \mathbb{R}^n\to \mathbb{R}^n
.\] 
L'applicazione del funzionale manda la variabile $\vect{x}$ nella soluzione, in questo modo il funzionale caratterizza completamente il sistema.\\
Le proprietà della $\varphi$ sono:
\begin{enumerate}
    \item $\varphi (t, \vect{x}) \in C^r$.
    \item $\varphi (0, \vect{x}_0) = \vect{x}_0$.
    \item $\varphi (t + s, \vect{x}_0) = \varphi (t, \varphi (s, \vect{x}_0))$.
\end{enumerate}
Introduciamo adesso il flusso nel caso in cui il sistema non è autonomo. Un sistema non autonomo è generalmente caratterizzato dalle equazioni:
\[
    \begin{dcases}
	\frac{\text{d} \vect{x}}{\text{d} t} = F(\vect{x}, t)\\
	\vect{x}(t_0) = \vect{x}_0
    \end{dcases}
    \qquad 
    \vect{x} \in \mathbb{R}^n; \ F \in C^r; \ F: \mathbb{R}^n\to \mathbb{R}^n
\] 
Notiamo che nella condizione iniziale si è messo come tempo iniziale $t_0$, questo è dovuto al fatto che, in un sistema non autonomo, la soluzione dipende dalla variabile $t_0$ (e non solo da $t-t_0$ come si avrebbe per un sistema autonomo). Questa caratteristica corrisponde alla perdita di invarianza per traslazione temporale della soluzione.\\
Le metodologie che permettono di introdurre il flusso in questi sistemi sono $2$:
\begin{itemize}
    \item Process Formulation.
    \item Skew Product Flow Formulation.
\end{itemize}
\paragraph{Process Formulation.}%
\label{par:Process Formulation.}
Supponiamo che esista e sia unica la soluzione del IVP e che tale soluzione sia globale (definita $\forall \ t$).\\
Definiamo il flusso di questo sistema come la soluzione dell'IVP $\Phi(t, t_0, \vect{x}_0)$. Le proprietà di $\Phi$  sono:
\begin{enumerate}
    \item $\Phi(t, t_0, \vect{x}_0)$ eredità tutte le proprietà del funzionale $F$.
    \item $\Phi(t, t_0, \vect{x}_0) = \vect{x}_0$ (Proprietà di identità).
    \item $\Phi(t_2, t_0, \vect{x}_0) = \Phi(t_2, t_1, \Phi(t_1, t_0, \vect{x}_0))$ con $t_0 \le t_1\le t_2$.
\end{enumerate}
Potremmo essere più formali definendo lo spazio:
\[
    \mathbb{R}^2_{\ge } \equiv \left\{ (t, t_0) \in \mathbb{R}^2 \ | \ t \ge t_0\right\}
.\] 
Quindi definiamo il flusso di fase come il funzionale (di variabile generica $*$):
\[
    \varphi (t,t_0,*):\mathbb{R}^n\to \mathbb{R}^n \qquad \text{con } (t, t_0) \in \mathbb{R}^2_{\ge }
.\] 
Che gode delle solide proprietà di flusso, che ripetiamo:
\begin{enumerate}
    \item $\varphi(t, t_0, \vect{x}_0) \in C^r$ con $r\ge 1$.
    \item $\varphi(t, t_0, \vect{x}_0) = \vect{x}_0$ (Proprietà di identità).
    \item $\varphi(t_2, t_0, \vect{x}_0) = \varphi(t_2, t_1, \varphi(t_1, t_0, \vect{x}_0))$ con $(t_2, t_1) \in \mathbb{R}^2_{\ge }$, e anche $(t_1, t_0) \in \mathbb{R}^2_{\ge }$ .
\end{enumerate}
\begin{exmp}[Flusso per Process Formulation]
    Prendiamo il sistema:
    \[
        \begin{dcases}
	    \frac{\text{d} x}{\text{d} t} = -2tx\\
	    x(t_0) = x_0
        \end{dcases}
    .\] 
    Si può dimostrare (esercizio) che la soluzione ha la forma:
    \[
	x(t)=x_0e^{-(t^2-t_0^2)} \equiv \varphi (t, t_0, x_0)
    .\] 
    La dipendenza da $t_0$ non può essere eliminata in questo caso con una traslazione temporale, questo è dovuto al fatto che l'argomento dell'esponenziale non è riscrivibile come funzione di $t-t_0$:
    \[
	t^2-t_0^2 = \left(t-t_0\right)^2 + 2(t-t_0)t_0
    .\] 
\end{exmp}
\noindent
\paragraph{Skew Product Flow Formulation}%
\label{par:Skew Product Flow Formulation}
L'idea alla base del metodo è quella di aggiungere ulteriori equazioni del moto in modo tale da rendere il sistema nuovamente autonomo. A quel punto il flusso di fase sarà quello già visto in precedenza.
\begin{exmp}[Pendolo]
    Nel caso del pendolo l'equazione del moto abbiamo visto che è:
    \[
	\ddot{x} = - x + A \sin (\omega t)
    .\] 
    E per rendere autonomo il sistema nuovamente abbiamo introdotto la variabile $\theta  = \omega t$. La chiave del funzionamento del metodo è proprio il fatto che $\theta$ ha una evoluzione autonoma.
\end{exmp}
\noindent
Formalmente prendiamo di nuovo il sistema di partenza:
\[
    \begin{dcases}
	\frac{\text{d} \vect{x}}{\text{d} t} = F(\vect{x}, t)\\
	\vect{x}(t_0) = \vect{x}_0
    \end{dcases}
    \qquad 
    \vect{x} \in \mathbb{R}^n; \ F \in C^r; \ F: \mathbb{R}^n\to \mathbb{R}^n
\]
Introduciamo un sistema dinamico da affiancare a questo:
\[
    \begin{dcases}
	\frac{\text{d} \vect{q}}{\text{d} t} = G(\vect{q})\\
	\vect{q} (t_0)= \vect{q}_0
    \end{dcases}
.\] 
Questo nuovo sistema è autonomo, possiamo allora risolvere il problema nel sistema di variabili:
\[
    \vect{y} =(\vect{x}, \vect{q}) \in \mathbb{R}^n \times \mathbb{R}^d
.\] 
Con $\mathbb{R}^d$ spazio di definizione di $\vect{q}$.
\[
    \begin{dcases}
	\frac{\text{d} \vect{x}}{\text{d} t} = F(\vect{x}, \vect{q})\\
	\vect{x} (t_0)=\vect{x_0}\\
	\frac{\text{d} \vect{q}}{\text{d} t} = G(\vect{q})\\
	\vect{q} (t_0)=\vect{q}_0
    \end{dcases}
.\] 
Il sistema in $\vect{q}$ è definito "Driver", il sistema in $\vect{x}$ invece è spesso detto "schiavizzato" dal Driver. Il sistema complessivo risulta comunque autonomo.\\
Quindi possiamo definire il flusso come lo spazio delle soluzioni in $\vect{x}$ e $\vect{q}$:
\[
    \varphi_t(\vect{x}_0, \vect{q}_0) = (\vect{x} (t, \vect{x}_0, \vect{q}_0), \vect{q} (t, \vect{q}_0))
.\] 
Essendo un sistema autonomo valgono le proprietà di flusso già viste:
\begin{enumerate}
    \item $\varphi_t \in C^r$ con $r\ge 1$.
    \item $\varphi_{t_0}(\vect{x}_0, \vect{q}_0) = (\vect{x}_0,\vect{q}_0)$.
    \item $\varphi_{t+s}(\vect{x}_0, \vect{q}_0) = \varphi_t(\varphi_s(\vect{x}_0, \vect{q}_0))$.
\end{enumerate}
Concentriamoci sulla terza proprietà ed esplicitiamola in modo diverso:
\[\begin{aligned}
    \varphi_{t+q} (\vect{x}_0, \vect{q}_0) =&  (\vect{x} (t+s , \vect{x}_0, \vect{q}_0), \vect{q} (t+s, \vect{q}_0)) = \\
                                           =& (\vect{x} (t, \vect{x} (s, \vect{x}_0,\vect{q}_0) \vect{q}(s, \vect{q}_0) ), \vect{q} (t, \vect{q} (s, \vect{q}_0)) ) = \\
					   =&(\vect{x} (t, \vect{x} (s, \vect{x}_0,\vect{q}_0),  \vect{q}(s, \vect{q}_0) ), \vect{q} (t+s, \vect{q}_0))
.\end{aligned}\]
Ed uguagliando la prima dopo l'uguale con l'ultima deve esser vero che:
\begin{defn}[Cocycle Property]
    \[
	\vect{x} (t+s, \vect{x}_0, \vect{q}_0) = \vect{x} (t, \vect{x} (s, \vect{x}_0, \vect{q}_0), \vect{q} (s, \vect{q}_0))
    .\] 
\end{defn}
\noindent
\begin{exmp}[Esempio di Cocycle Property]
    Prendiamo la seguente variabile "Driver":
    \[
	q(t) = t \in \mathbb{R} \qquad q(t_0)=t_0
    .\] 
    La proprietà in questo caso si esprime come:
    \[
	\vect{x} (t+s, \vect{x}_0, t_0) = \vect{x} ( t, \vect{x} (s, \vect{x}_0, t_0), t_0 + s)
    .\] 
\end{exmp}
\noindent
