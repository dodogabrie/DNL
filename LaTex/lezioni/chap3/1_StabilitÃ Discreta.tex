\section{Concetto di stabilità per SD a tempo discreto}%
Parliamo di mappe ricorive autonome (non compare il $k$.\\
Prendiamo la mappa ricorsiva:
\[
    \v{x}_{k+1} = G(\v{x}_k) \qquad \v{x}_k \in \mathbb{R}^n \qquad G:\mathbb{R}^n\to \mathbb{R}^n
.\] 
\begin{defn}[Stabilità secondo Lyapunov a t.d.]
    Sia $\v{\mu}_k$ un orbita del sistema dinamico a tempo discreto con $k \in \mathbb{N}$. Diciamo che $\v{\mu}_k$ è stabile secondo Lyapunov se $\forall \ \epsilon  > 0$ $\exists \ \delta (\epsilon)$ tale che per ogni altra orbita $\v{v}_k$ che per un certo $n\in \mathbb{N}$ soddisfa 
    \[
	\left|\v{v}_n - \v{\mu}_n\right|< \delta (\epsilon) 
    .\] 
    si ha che
    \[
        \left|\v{v}_k- \v{u}_k\right|<\epsilon  \quad  \forall \ k > n
    .\] 
\end{defn}
\noindent
\begin{defn}[Stabilità asintotica a t.d.]
    L'orbita $\v{u}_k$ è asintoticamente stabile se $\v{u}_k$ è stabile secondo Lyapunov ed inoltre
    \[
        \lim_{k \to \infty} \left|\v{v}_k - \v{u}_k\right|=0
    .\] 
\end{defn}
\noindent
\subsection{Stabilità di stati stazionari di mappe ricorsive}%
\begin{defn}[Stato stazionario a t.d.]
Data la mappa ricorsiva:
\[
    \v{x}_{k+1}=G(\v{x}_k) \qquad  \v{x}_k \in \mathbb{R}^n, \qquad  G:\mathbb{R}^n\to \mathbb{R}^n
.\] 
Diciamo che $\v{x}_s$ è uno stato stazionario della mappa se:
\[
    \v{x}_s = G(\v{x}_s) 
.\] 
\end{defn}
\noindent
Anche nel caso delle mappe ricorsive, determinato lo stato stazionario, dimostrare che uno stato stazionario sia stabile tramite Lyapunov non è affatto semplice. Ci sono allora delle tecniche opportune (quando gli stati stazionari sono iperbolici) di linearizzazione.
\begin{exmp}[]
\[
    x_{n+1}=1-x_n^2 = G(x_n) 
.\]     
Procediamo con il calcolo degli stazionari:
\[
    x_s = G(x_s) \implies  x_s = 1-x_s^2 \implies  x_{s_1, s_2}= \frac{-1\pm \sqrt{5} }{2}
.\] 
\end{exmp}
\noindent
\subsection{Stabilità di stati stazionari di mappe ricorsive mediante linarizzazione}%
Supponiamo di avere una mappa del tipo:
\[
    \v{x}_{k+1} = G(\v{x}_k) \quad  \v{x}_k \in \mathbb{R}^n \quad  G: \mathbb{R}^n\to \mathbb{R}^n
.\] 
E sia $\v{x}_s$ uno stato stazionario. Per studiarne la stabilità mediante linearizzazione si perturba lo stato stazionario stesso:
\[
    \v{x}_k = \v{x}_s + \v{y}_k \qquad  \v{y}_k \in \mathbb{R}^n \qquad  \left|\v{y}\right|\ll 1
.\] 
Allora si sviluppa la mappa al primo ordine:
\[
    \v{x}_s + \v{y}_{k+1} = G(\v{x}_s + \v{y}_k) \simeq G(\v{x}_s) + DG(\v{x}_s) \v{y}_k + o(\left|\v{y}_k\right|^2) 
.\] 
Tanto più è piccola la perturbazione e tanto più vale la approssimazione, semplificando $\v{x}_s$:
\[
    \v{y}_{k+1}=DG(\v{x}_s) \v{y}_{k}
.\] 
\begin{exmp}[]
    Prendiamo la mappa 1D:
    \[
	x_{k+1}=  \alpha x_k = F(x_k)  \qquad \alpha \in \mathbb{R} - \left\{0\right\}
    .\] 
    Lo stato stazionario è l'origine $x_s = 0$. Allora l'evoluzione della perturbazione è:
    \[
	y_{k+1} = \left.\frac{\text{d} F(x_k) }{\text{d} x_k}\right|_{x = 0} y_k = \alpha y_k
    .\] 
    Si risolve immediatamente per una forma non ricorsiva in $y$ in funzione del punto iniziale:
    \[
        y_k = \alpha^k y_0
    .\] 
    Quindi a seconda del segno di $\alpha$ abbiamo la stabilità/instabilità della mappa ricorsiva:
    \begin{itemize}
	\item $\alpha >1$ $\implies  \alpha^ky_0 $ diverge per $k\to \infty$, questa determina una instabilità del sistema.
	\item $\alpha < -1$ $\implies \alpha^ky_0 $ diverge per $k\to \infty$, questa determina una instabilità del sistema. 
	\item $\alpha = 1$ abbiamo una specie di centro: la perturbazione rimane costante nel tempo.
	\item $\alpha  = -1$ $\implies  y_k = (-1)^ky_0$, il modulo di $y_k$ rimane costante. 
	\item $\left|\alpha\right|<1$ $\implies  \alpha^ky_0 \to 0$ con $k\to \infty$, questa determina una situazione stabile.
    \end{itemize}
    Questo esempio ci permette di dare una definizione generale di stabilità. Notiamo che in questo caso $\alpha$ rappresenta proprio l'autovalore della matrice Jacobiana.
    \[
	\text{det}\left[DF(0) - \lambda\right] = 0 \implies  \alpha-\lambda = 0 \implies  \lambda  = \alpha
    .\] 
\end{exmp}
\noindent
In generale dobbiamo calcolare tutti gli autovalori della matrice $DG(\v{x}_s)$ e si ha il risultato seguente:
\begin{enumerate}
    \item Se  $\left|\lambda_J\right| < 1$ $(\forall J = 1, 2, \ldots, n)$ con $\lambda_J$ autovalore di $DG(\v{\v{x}_s})$ allora $\v{x}_s$ è stabile secondo Lyapunov.
    \item Se esiste \textbf{almeno un valore} di $J$ tale che $\left|\lambda_J\right| > 1$ con $\lambda_J$ autovalore di $DG(\v{\v{x}_s})$ allora $\v{x}_s$ è instabile secondo Lyapunov.
\end{enumerate}
\begin{defn}[Stato stazioario iperbolico]
    $\v{x}_s$ è stato stazionario iperbolico se $\left|\lambda_J\right|\neq 1$ $\forall J=1, \ldots, n$ con $\lambda_J$ autovalori di $DG(\v{x}_s)$.
\end{defn}
\noindent
\begin{defn}[Sella]
    $\v{x}_s$ è una sella se $\v{x}_s$ è iperbolico e ci sono autovalori aventi modulo maggiore di uno ed altri aventi modulo minore di uno.
\end{defn}
\noindent
\begin{defn}[Pozzo]
    Se $\left|\lambda_J\right|<1$ $\forall J = 1, 2, \ldots, n$ allora $\v{x}_s$ è un pozzo.
\end{defn}
\noindent
\begin{defn}[Sorgente]
    Se $\left|\lambda_J\right|>1$ $\forall J = 1, 2, \ldots, n$ allora $\v{x}_s$ è una sorgente.
\end{defn}
\noindent
Esiste una versione del teorema di Lyapunov anche nel caso dello studio della stabilità di stati stazionario di mappe ricorsive.
Esiste anche la versione del teorema HH, gli autovalori delle mappe con modulo unitario anche qui vanno trattati con cautela\ldots \\
Possiamo introdurre le varietà stabili, instabili e centro lineari ($E^s, E^u, E^c$) nel seguente modo:
\begin{enumerate}
    \item Sia $\v{x}_s$ stato stazionario.
    \item Si determina $DG(\v{x}_s)$ e si calcolano autovalori ed autovettori generalizzati di tale matrice.
    \item Si definisce $E^s = \text{span}\left\{ns 
	\text{ autovalori gen. corrisp. ad autovalori con } \left|\lambda\right|<1\right\}$ 
    \item Si definisce $E^u = \text{span}\left\{ns 
	\text{ autovalori gen. corrisp. ad autovalori con } \left|\lambda\right|>1\right\}$ 
    \item Si definisce $E^c = \text{span}\left\{ns 
	\text{ autovalori gen. corrisp. ad autovalori con } \left|\lambda\right|=1\right\}$ 
\end{enumerate}
Si possono anche definire le varietà non lineari $W$ ed applicare i medesimi teoremi del caso di SD a tempo continuo.\\
\subsection{Manifold locali stabili ed instabili per mappe ricorsive}%
Data una mappa:
    \[
	x_{k+1}=G(\v{x}_k) \quad  \v{x}\in \mathbb{R}^n, \quad  G:\mathbb{R}^n\to \mathbb{R}^n
    .\] 
Con $\v{x}_s$ stato stazionario della mappa. 
\begin{defn}[Manifold locale stabile per SD a tempo discreto]
Il manifold locale stabile della mappa è definito come:
\[
    W_{loc}^s(\v{x}_s) = \left\{\v{x}\in U | G^n(\v{x}) \to \v{x}_s \text{ con } n\to \infty \text{ e } G^n(\v{x}) \in U \ \forall n \ge 0\right\}
.\] 
\end{defn}
\noindent
\begin{defn}[Manfold locale stabile]
Se la mappa è invertibile allora il manifold locale instabile di tale mappa è definito come:
\[
    W_{loc}^u(\v{x}_s) = \left\{\v{x}\in U | G^{-n}(\v{x}) \to \v{x}_s \text{ con } n\to \infty \text{ e } G^{-n}(\v{x}) \in U \ \forall n \ge 0\right\}
.\] 
Se la mappa non è invertibile si può comunque definire questo insieme, per farlo si ragiona sulle preimmagini di $\v{x}$.
\end{defn}
\noindent
\begin{defn}[Manifold globale stabile]
    \[
	W^s(\v{x}_s) = \bigcup_{n\ge 0} G^{-n}(W_{loc}^s(\v{x}_s) ) 
    .\] 
\end{defn}
\noindent
\begin{defn}[Manifold globale instabile]
    \[
	W^u(\v{x}_s) = \bigcup_{n\ge 0} G^{n}(W_{loc}^u(\v{x}_s) ) 
    .\] 
\end{defn}
\noindent
Per i manifold locali vale il teorema di esistenza e di unicità (non valida per il manifold di centro), inoltre tali manifold sono tangenti alle corrispondenti varietà lineari stabili, instabili e di centro.\\
Quando si aggiunge una perturbazione si possono sempre definire le varietà lineari/instabili/centro? Soltanto se il sistema presenta una \textbf{dicotomia esponenziale} (Wiggins per dettagli).
