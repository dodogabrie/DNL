\section{Insiemi invarianti (stabili, instabili e centro)}%
\label{sec: Insiemi invarianti (stabili, instabili e centro)}%
\begin{defn}[Insieme invariante rispetto al flusso di fase]
    Sia $E \subset \mathbb{R}^n$ e sia $\varphi_t(\v{x}_0): \mathbb{R}^n\to \mathbb{R}^n$ un flusso di fase tale che:
    \[
	\varphi_t(\v{x}_0) = e^{At}\v{x}_0
    .\] 
    Con $A$ matrice reale $n \times n$.
    Diciamo che $E$ è invariante rispetto a $\varphi_t$ se:
    \[
        \varphi_t(E) \subset E \quad \forall t \in \mathbb{R}
    .\] 
    Inoltre $E$ è positivamente invariante rispetto a $\varphi_t$ se vale che:
    \[
	\varphi_t(E) \subset E \quad  \forall t \in \left\{0\right\} \cup \mathbb{R}^+
    .\] 
\end{defn}
\noindent
\begin{thm}[Invarianza dell'autospazio generalizzato]
Dato il sistema dinamico $\frac{\text{d} \v{x}}{\text{d} t} = A\v{x} $ con $\v{x}\in \mathbb{R}^n$ e $A$ matrice reale $n \times n$\sidenote{Ipotizziamo di aver gia linearizzato questo sistema}. \\
Supponiamo che $\Lambda$ sia un autovalore di $A$ e sia $E$ l'autospazio generalizzato di tale autovalore.\\
Allora $A(E) \subset E$.
\end{thm}
\noindent
Si dimostra ricordando la definizione di autospazio generalizzato.\\
Data una matrice reale $n \times n$ (associata al sistema dinamico del teorema sopra). Supponiamo di aver calcolato tutti gli autovalori di questa matrice, in tal caso possiamo trovare anche tutti gli autovettori generalizzati. \\
Possiamo suddividere gli autovalori di $A$ in tre gruppi.
\begin{defn}[Linear Stable manifold ]
    Indichiamo questo insieme come $E^S$, definito come:
    \[
        E^S = \text{span}\left\{\v{v}_1^s, \ldots \v{v}_{n_s}^s\right\}
    .\] 
    Dove i $\left\{\v{v}_J^s\right\}$ con $J = 1,\ldots, n_s$ sono gli autovettori generalizzati corrispondenti agli autovalori con parte reale \textbf{negativa}.
\end{defn}
\noindent
\begin{defn}[Linear Unstable manifold ]
    Indichiamo questo insieme come $E^U$, definito come:
    \[
        E^U = \text{span}\left\{\v{v}_1^u, \ldots \v{v}_{n_u}^u\right\}
    .\] 
    Dove i $\left\{\v{v}_J^u\right\}$ con $J = 1,\ldots, n_u$ sono gli autovettori generalizzati corrispondenti agli autovalori con parte reale \textbf{positiva}.
\end{defn}
\noindent
\begin{defn}[Linear Center manifold ]
    Indichiamo questo insieme come $E^C$, definito come:
    \[
        E^C = \text{span}\left\{\v{v}_1^c, \ldots \v{v}_{n_c}^c\right\}
    .\] 
    Dove i $\left\{\v{v}_J^c\right\}$ con $J = 1,\ldots, n_c$ sono gli autovettori generalizzati corrispondenti agli autovalori con parte reale \textbf{nulla}.
\end{defn}
\noindent
La dimensionalità del sistema in questione, date le precedenti definizioni, sarà data da $n = n_s + n_u + n_c$.
\begin{exmp}[Manifold stabili e centro]
    Sia dato il seguente sistema dinamico a tempo continuo:
    \[
        \frac{\text{d} \v{x}}{\text{d} t} = A \v{x} \quad  \v{x}\in \mathbb{R}^2 \quad
	A = 
    \begin{pmatrix}
	0  & 1 \\
	0 & -4 \\
    \end{pmatrix}
    .\] 
    Determinare le varietà (o Manifold) stabili.\\
    Calcoliamo per prima cosa gli autovalori:
    \[
	P(\lambda) = \text{det}\left[A-\lambda\mathbb{I}\right] = -\lambda (-4-\lambda) =0
    .\] 
    Quindi $\lambda_{1, 2} = 0; -4$. Troviamo adesso gli autovettori:
    \[
	\v{v}_{1}^c = \begin{pmatrix} 1 \\ 0 \end{pmatrix}; \qquad  \v{v}_{1}^s = \begin{pmatrix} 1 \\ - 4 \end{pmatrix} 
    .\] 
    Notiamo, per esempio, che lo span di $\v{v}_1^c$ è tutto l'asse $x$.
    \marginpar{
        \captionsetup{type=figure}
            \incfig{2_8_1}
        \caption{\scriptsize Varietà centro e varietà stabile }
	\label{fig:2_8_1}
    }
    \[\begin{aligned}
	&E^S = \text{span}\left\{\v{v}_1^s = \begin{pmatrix} 1 \\ -4 \end{pmatrix} \right\}\\
	&E^U = \left\{0\right\}\\
	&E^C = \text{span}\left\{\v{v}_1^c = \begin{pmatrix} 1 \\ 0 \end{pmatrix} \right\}
    .\end{aligned}\]
    Notiamo quindi che un qualunque punto dell'asse $x$ è uno stato stazionario. Tutti i punti con $y\neq 0$ invece tendono a decadere esponenzialmente sull'asse $x$ con traiettorie aventi la stessa pendenza della curva $\v{v}_1^s$ di figura \ref{fig:2_8_1}
\end{exmp}
\noindent 
\begin{exmp}[Sistema dinamico in $\mathbb{R}^3$]
    Dato il sistema dinamico:
    \[
        \frac{\text{d} \v{x}}{\text{d} t} = A \v{x}, \quad \v{x}\in \mathbb{R}^3 \quad  A = 
    \begin{pmatrix}
	-1  & -1  & 0  \\
	1  & -1 & 0 \\
	0 & 0 & 2 \\
    \end{pmatrix}
    .\] 
    Determinare $E^S, E^u, E^c$. Notiamo che la matrice è a blocchi, in quello in alto abbiamo una matrice di Jordan che abbiamo studiato, in basso abbiamo un numero positivo che ci porta a pensare ad un asse instabile.\\
    Visto che vale:
    \[
        A^{up} = 
    \begin{pmatrix}
	a & -b   \\
	-b & a \\
    \end{pmatrix} 
    \implies 
    e^{A^{up}t} = e^{at}
    \begin{pmatrix}
    \cos(bt) & - \sin(bt)  \\
    \sin(bt) & \cos(bt)  \\
    \end{pmatrix}
    .\] 
    con $a = -1$ nel caso in questione abbiamo una varietà stabile nelle due direzioni $x, y$.\\
    Troviamo gli autovalori:
    \[
	P(\lambda) = \text{det}
    \begin{pmatrix}
	-1-\lambda & -1 & 0 \\
	1 & -1-\lambda & 0 \\
	0 & 0 & 2-\lambda \\
    \end{pmatrix}
    =
    (2-\lambda) \left[(1+\lambda)^2 + 1\right] = 0
    .\] 
    Quindi abbiamo $\lambda_1 = 2$, $\lambda_{2,3} = -1 \pm i$.
    Troviamo gli autovettori corrispondenti:
    \[
        \v{v}_1^u = \begin{pmatrix} 0 \\ 0 \\ 1 \end{pmatrix} \qquad 
	\v{w}_{1,2}^s = \begin{pmatrix} 1 \\ \pm i \\ 0 \end{pmatrix}
    .\] 
    Abbiamo usato la notazione con $\v{w}$ poichè è un vettore complesso $\v{w} \in \mathbb{C}^3$. Si ha quindi che:
    \[
        \v{w}_{1,2}^s = \v{v}_{1}^s + i \v{v}_2^s = \begin{pmatrix} 1  \\ 0 \\ 0 \end{pmatrix}  + 
	i \begin{pmatrix} 0  \\ 1 \\ 0 \end{pmatrix}
    .\] 
    Ed in conclusione:
    \marginpar{
        \captionsetup{type=figure}
            \incfig{2_8_2}
        \caption{\scriptsize Varietà dell'esercizio che ci indicano il flusso di fase, sul piano $x, y$ abbiamo una rotazione dovuta alla componente complessa.}
	\label{fig:2_8_2}
    }
    \[\begin{aligned}
	& E^s = \text{span}\left\{\v{v}_1^s, \v{v}_2^s\right\}\\
	& E^u = \text{span}\left\{\v{v}_1^u\right\}\\
	& E^c = \left\{0\right\}
    .\end{aligned}\]
    La matrice di trasformazione $P$ è:
    \[
        P = \left[\v{v}_1^u \v{v}_2^s \v{v}_1^s\right] = 
    \begin{pmatrix}
	0  & 0 & 1 \\
	0 & 1 & 0 \\
	1 & 0 & 0 \\
    \end{pmatrix}
    .\] 
    Quindi la soluzione si scrive come:
    \[\begin{aligned}
	\v{v}(t) &= P \text{diag}\left[e^{\lambda_1 t}\ldots e^{\lambda_k}t B_{k+1}\ldots B_{k + \frac{n-k}{2}}\right]P^{-1}\v{x}_0\\
		 & = P \text{diag}\left[e^{2t}e^{-t}
	\begin{pmatrix}
	    \cos t & \sin t  \\
	    -\sin t & \cos  t \\
	\end{pmatrix}
		 \right]
		 P^{-1}\v{v}_0
    .\end{aligned}\]
\end{exmp}
\noindent
\begin{thm}[Proprietà dei sottospazi generalizzati]
    Dato il sistema dinamico \[
	\frac{\text{d} \v{x}}{\text{d} t} = A \v{x} \qquad  \v{x}\in \mathbb{R}^n \qquad  A \in L(\mathbb{R}^n) 
    .\] 
    Sia $\varphi_t(\v{x})$ il corrispondente flusso di fase. Siano $E^s, E^u, E^c$ i sottospazi generalizzati\sidenote{\scriptsize Ovvero costruiti utilizzando gli autovettori generalizzati come nell'esempio precedente} corrispondenti agli autovalori di $A$. Allora vale che:
    \begin{itemize}
	\item 
	    $\varphi_t(E^s) \subset E^s$.\\
            $\varphi_t(E^u) \subset E^u$.\\
            $\varphi_t(E^c) \subset E^c$.
	\item $\mathbb{R}^n = E^s \oplus E^u \oplus E^c$ 
    \end{itemize}
\end{thm}
\noindent
Per dimostrarlo si supponga $S \subset \mathbb{R}^n$ e $S$ invariante rispetto ad $A$:
\[
    A(S) \subset S \qquad (\forall \v{v}\in S: A\v{v}\in S) 
.\] 
Possiamo dire che $S$  è invariante rispetto a $cA$  con $c\in \mathbb{R}$. Inoltre se $S$  è invariante rispetto a $A_1$  e $A_2$  (matrici $n \times n$) allora sarà invariante anche rispetto a $c_1A_1 + c_2A_2$. Infine
\[
    e^{At}=\lim_{N \to \infty}\sum_{J=0}^{N} \frac{A^J t^J}{J!} \equiv \lim_{N \to \infty} L_N(t) 
.\] 
Questo limite corrisponde ad una convergenza uniforme, se tale serie è uniformenmente convergente allora esiste un teorema che garantisce che tale successione sia di Cauchy. Quindi quando applico $L_N$ ad un vettore in $S$ allora il limite rimarrà in $S$\ldots\sidenote{\scriptsize Questi sono i punti chiave della dimostrazione} 


\begin{defn}[Insiemi omeomorfi]
    Sia $X$ uno spazio metrico e $A, B \subset X$. Diciamo che $A$ e $B$ sono omeomorfi (oppure topologicamente equivalenti) se esiste $h:A\to B$ che abbia le seguenti proprietà:
    \begin{itemize}
	\item Bigettiva.
	\item $h$ e $h^{-1}$ sono continue: appartengono a $C^0$.
    \end{itemize}
    $h$ è chiamato anche omomorfismo.\sidenote{\scriptsize Se $h$ e $h^{-1}$ sono $C^2$ abbiamo un diffeomorfismo.}
\end{defn}
\noindent
Dato il campo vettoriale 
\[
    \frac{\text{d} \v{x}}{\text{d} t} = F(\v{x}) \quad  \v{x}\in \mathbb{R}^n \quad  F:\mathbb{R}^n\to \mathbb{R}^n
.\] 
Con uno stato stazionario $\v{x}_s: $  $F(\v{x}_s) =0$. A questo sistema abbiamo associato un sistema linearizzato nel punto stazionario:
\[
    \frac{\text{d} \v{y}}{\text{d} t} = J(\v{x}_s) \v{y}
.\] 
Quello che ci chiediamo è quanto queste orbite linearizzate rispecchino il sistema non lineare in un intorno di $\v{x}_s$. Quindi ci chiediamo quanto il sistema linearizzato e quello non lineare siano equivalenti in un intorno di $\v{x}_s$.\\
Notiamo che, dato uno stato stazionario $\v{x}_s$ possiamo sempre definire $\v{x}=\v{x}_s + \v{z}$, quindi:
\[
    \frac{\text{d} \v{z}}{\text{d} t} = F(\v{x}_s - \v{x}) = F(\v{z}) 
.\] 
Quindi lo stato stazionario si sposta in $\v{z}=0$. Questo è comodo per il seguente teorema.
\marginpar{
        \captionsetup{type=figure}
        \incfig{2_8_4}
	\caption{\scriptsize Soluzione lineare ($\v{y}$) e soluzione non lineare ($\v{x}$) in un intorno $U$ di $\v{x}_s$.}
    \label{fig:2_8_4}
    }
\begin{thm}[Hartman-Grobman]
Sia dato il sistema dinamico 
\[
    \frac{\text{d} \v{x}}{\text{d} t} = F(\v{x}); \quad \v{x}\in \mathbb{R}^n; \quad F:\mathbb{R}^n\to \mathbb{R}^n; \quad \v{x}_s \text{ staz}
.\] 
Sia $\varphi_t(\v{x})$ il flusso di fase associato al SD, indichiamo con $\tilde{\varphi}_t(\v{x})$ il flusso del campo linearizzato $\dot{\v{y}}=J(\v{x}_s) \v{y}$. Sia inoltre $\v{x}_s$ uno stato stazionario iperbolico.\\
Allora esiste un intorno $U$ di $\v{x}_s$ nel quale si ha un omomorfismo $H$ che manda le orbite del sistema lineare in quelle del sistema non lineare e viceversa. 
Tale omeomorfismo preserva l'orientamento temporale.
\end{thm}
\noindent
Questo significa che se $\v{y}_0\to \v{y}_1 \implies  H(\v{y}_0) \to H(\v{y}_1)$.

% Stack Overflow %%%%%%%%%%%%%%%%%%%%%%%%%%%%%%%%%%%%%%%%%%%%%%%%%%%
\catcode`\@=11
\newdimen\cdsep
\cdsep=2em

\def\cdstrut{\vrule height .5\cdsep width 0pt depth .4\cdsep}
\def\@cdstrut{{\advance\cdsep by 2em\cdstrut}}

\def\arrow#1#2{
  \ifx d#1
    \llap{$\scriptstyle#2$}\left\downarrow\cdstrut\right.\@cdstrut\fi
  \ifx u#1
    \llap{$\scriptstyle#2$}\left\uparrow\cdstrut\right.\@cdstrut\fi
  \ifx r#1
    \mathop{\hbox to \cdsep{\rightarrowfill}}\limits^{#2}\fi
  \ifx l#1
    \mathop{\hbox to \cdsep{\leftarrowfill}}\limits^{#2}\fi
}
\catcode`\@=12
\cdsep=3em
%%%%%%%%%%%%%%%%%%%%%%%%%%%%%%%%%%%%%%%%%%%%%%%%%%%%%%%%%%%%%%%%%%%%%
\[
    \v{x}(t) =\varphi_t(\v{x}_0); \quad  \v{x}(t) =H(\v{y}(t) ) = \varphi_t(\v{x}_0) = \varphi_t(H(\v{y}_0) ) 
.\] 
La struttura dell'omeomorfismo è riassunta a lato (\sidenote{
$$
\begin{matrix}
  \v{x}_0              & \arrow{r}{\varphi_t}   & \v{x}(t)   \cr
  \arrow{u}{H}        &                & \arrow{d}{H^{-1}}        \cr
  \v{y}_0             & \arrow{r}{\tilde{\varphi}_t} & \v{y}(t)                   \cr
\end{matrix}
$$
}).\\
Abbiamo quindi le conseguenti relazioni: 
\[\begin{aligned}
    &\v{y}(t) = H^{-1}\varphi_t H \v{y}_0\\
    &\tilde{\varphi_t}\v{y}(t) = H^{-1}\varphi_t H \v{y}_0 \implies  \tilde{\varphi_t}= H^{-1}\varphi_t H
.\end{aligned}\]
\begin{exmp}[Applicazione del teorema in $\mathbb{R}^2$]
Prendiamo il seguente sistema dinamico:
\[
\begin{dcases}
\frac{\text{d} x}{\text{d} t} = x-y^2\\
\frac{\text{d} y}{\text{d} t} = -y
\end{dcases}
\]
Gli stati stazionari sono definiti da:
\[\begin{aligned}
    &x-y^2=0\\
    &y = 0
.\end{aligned}
\implies 
\v{v}_s = \begin{pmatrix} 0 \\ 0 \end{pmatrix}
\]
Per gli autovalori si ha che:
\[
    J = 
\begin{pmatrix}
    1  & 0 \\
    0 & -1 \\
\end{pmatrix}
\implies  \lambda_{1, 2} = \pm 1
.\] 
Quindi abbiamo una sella. Il sistema linearizzato in un intorno di $\v{v}_s$  è:
\[
\begin{dcases}
\frac{\text{d} x}{\text{d} t} = x \\
\frac{\text{d} y}{\text{d} t} = -y
\end{dcases}
\]
Quindi la soluzione locale è:
\[
\begin{dcases}
    x(t) = x_0e^t\\
    y(t) = y_0e^{-t}
\end{dcases}
\]
(Per casa) dimostrare che per il sistema dinamico non lineare vale:
\begin{equation}
\begin{dcases}
    x(t) = x_0e^{t} + \frac{y_0^2}{3}(e^{-2t}-e^t) \\
    y(t) = y_0e^{-t}
\end{dcases}
\label{eq:8_1}
\end{equation}
Possiamo cercare di capire come deve essere fatto l'omeomorfismo locale $H$, si scopre che:
\[
    H:\mathbb{R}^2\to \mathbb{R}^2; \quad  
    \begin{pmatrix} x \\ y \end{pmatrix} \xrightarrow{H} \begin{pmatrix} x-\frac{y^3}{3} \\ y \end{pmatrix}
.\] 
Si può dimostrare che la mappa cercata è proprio questa, inoltre si ha che $H^{-1}$  esiste e vale:
\[
    H^{-1}:\mathbb{R}^2\to \mathbb{R}^2; \quad  
    \begin{pmatrix} x \\ y \end{pmatrix} \xrightarrow{H^{-1}} \begin{pmatrix} x + \frac{y^2}{3} \\ y \end{pmatrix}
.\] 
Preso il campo vettoriale di equazione \ref{eq:8_1} ed applicandogli la trasformazione $H$ si ha che:
\[
    H\left(\begin{pmatrix} x(t)  \\ y(t)  \end{pmatrix}\right) = 
    \begin{pmatrix} x_0e^t + \frac{y_0^2}{3}(e^{-2t}-e^t) - \frac{y_0^2}{3}e^{-2t}  \\ y_0e^{-t} \end{pmatrix} =
    \begin{pmatrix} x_0e^t-\frac{y_0^2}{3}e^t \\ y_0e^{-t} \end{pmatrix}
.\] 
Possiamo riscrivere la mappa che abbiamo ottenuto in questo modo:
\[
    H\left(\varphi_t\left(\begin{pmatrix} x_0 \\ y_0 \end{pmatrix}\right) \right) =
\begin{pmatrix}
    e^t & 0 \\
    0 & e^{-t} \\
\end{pmatrix}
\begin{pmatrix} x_0-\frac{y_0^2}{3} \\ y_0 \end{pmatrix} = \tilde{\varphi}_t\left(\begin{pmatrix} x_0-\frac{y_0^2}{3} \\ y_0 \end{pmatrix}\right)
.\] 
\end{exmp}
\noindent
La prima matrice corrisponde proprio al flusso di fase lineare.
\[
    H(\varphi_t(\v{x})) = \tilde{\varphi}_t \left(\begin{pmatrix} x_0-\frac{y_0^2}{3} \\ y_0 \end{pmatrix}\right) = 
    \tilde{\varphi_t}(H(\v{x}))
.\] 
