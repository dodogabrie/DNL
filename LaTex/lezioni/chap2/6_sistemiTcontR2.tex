\section{SD a tempo continuo in $\mathbb{R}^2$  della forma $\frac{\text{d}^2x}{\text{d} t^2} = F(t, x, \dot{x})$}%
\label{sub:SD a tempo continuo in   della forma}
In questa sezione ci limitiamo a studiare il caso in cui il sistema dinamico è autonomo, vedremo più avanti il caso dipendente dal tempo.
\[
    \frac{\text{d} ^2x}{\text{d} t^2} = F(x, \frac{\text{d} x}{\text{d} t} )
\] 
\begin{exmp}[Oscillatore armonico smorzato]
    \[
	\frac{\text{d} ^2x}{\text{d} t^2} = -k x - \mu\frac{\text{d} x}{\text{d} t} 
    \] 
\end{exmp}
\noindent
\begin{exmp}[Circuito Van Der Pol]
    \[
	I_R(V)=aV+bV^2
    \] 
    L'equazione che descrive tale circuito è del tipo di quelle che studiamo adesso:
    \[
	\frac{\text{d} ^2x}{\text{d} t^2} = F(x, \dot{x})
    \] 
\end{exmp}
\noindent
Nello studio di questi sistemi solitamente la prima cosa da fare è ricondurci ad un sistema di equazioni del primo ordine: ponendo $y = \frac{\text{d} x}{\text{d} t} $  allora possiamo riscrivere il sistema nella forma:
\[\begin{dcases}
    \frac{\text{d} x}{\text{d} t} = y\\
    \frac{\text{d} y}{\text{d} t} = F(x,y)
\end{dcases}\] 
\paragraph{Stati stazionari del SD}%
\[\begin{dcases}
    y=0\\
    F(x,0)=0
\end{dcases}\] 
Quindi gli stati stazionari si trovano tutti sull'asse $x$.\\
Consideriamo il rapporto delle derivate rispetto a $x$ e $y$:
\[
    \frac{\frac{\text{d} y}{\text{d} t}}{\frac{\text{d} x}{\text{d} t} } = \frac{F(x,y)}{y} \qquad   \text{con } y\neq 0
\] 
Riscrivendo il rapporto da bravi fisici possiamo dire che:
\begin{equation}
    \frac{\text{d} y}{\text{d} x} = \frac{F(x, y)}{y}
    \label{eq:2_6_1}
\end{equation}
Questa relazione porta alcune interessanti conseguenze: immaginando un'orbita che si avvicina all'asse $x$ e passa per un punto dell'asse $x$ non stazionario come in figura \ldots, grazie alla relazione siamo in grado di dire che la tangente sull'asse $x$ è perpendicolare all'asse stesso per via della divergenza della tangente quando $y\to 0$.
\begin{enumerate}
    \item L'attraversamento di un punto sull'asse $x$ non stazionario è perpendicolare all'asse stesso.
    \item La soluzione della equazione \ref{eq:2_6_1} ci permette di tracciare tutte le possibili orbite del SD.
\end{enumerate}
\begin{exmp}[Applicazione a pendolo non smorzato]
    Prendiamo il sistema dinamico:
    \[
	\frac{\text{d} ^2x}{\text{d} t^2} = - \omega^2\sin (x) \implies  
	\begin{dcases}
	    \frac{\text{d} x}{\text{d} t} = y \\
	    \frac{\text{d} y}{\text{d} t} = -\omega^2\sin x
	\end{dcases}
    \quad x \in S^1, y \in \mathbb{R}
    \] 
    Caratterizziamo il comportamento degli stati stazionari:
    \[\begin{dcases}
        y = 0\\
	F(x, 0)=0
    \end{dcases}
    \implies
    \begin{cases}
        y=0\\
	\sin x = 0
    \end{cases}
    \implies 
    (k\pi, 0)
    \quad 
    k \in \mathbb{Z}
\] 
Visto che abbiamo dei manifold localmente è sempre possibile mappare $S_1\times \mathbb{R} \to \mathbb{R}^2$. Procediamo quindi come se $x$ fosse una variabile ordinaria e cartesiana:
\[
    J(x, y) = \begin{pmatrix} 0 & 1 \\ - \omega^2\cos x & 0 \end{pmatrix} 
\] 
\textbf{1.} $\left|k\right|$ pari:
\[
    J_{\text{pari}} = \begin{pmatrix} 0 & 1 \\ -\omega^2 & 0 \end{pmatrix} \implies  \Lambda^2 + \omega^2 = 0 \implies  \Lambda  = \pm i \omega
\] 
Quindi in $0, 2\pi, 4\pi\ldots$ localmente possiamo aspettarci dei centri.
\textbf{2.} $\left|k\right|$ dispari:
\[
    J_{\text{dis}} = \begin{pmatrix} 0 & 1 \\ \omega^2 & 0 \end{pmatrix} \implies  \Lambda^2 - \omega^2 = 0 \implies  \Lambda  = \pm  \omega
\] 
Quindi abbiamo in questo caso due autovalori reali, uno positivo e l'altro negativo: delle selle.\\
Calcoliamo gli autovettori di questo caso, dobbiamo studiare il problema:
\[
    \Lambda  = 1 \implies  \begin{pmatrix} 0 & 1 \\ \omega^2 & 0 \end{pmatrix} \begin{pmatrix} x \\ y \end{pmatrix} = \omega\begin{pmatrix} x \\ y \end{pmatrix} \implies  \begin{pmatrix} y \\ \omega x \end{pmatrix} = \begin{pmatrix} \omega x \\ \omega y \end{pmatrix} 
\] 
\[\begin{dcases}
    y = \omega x\\
    \omega y = \omega^2x
\end{dcases}\implies  y = \omega x \] 
Quindi la direzione instabile è rappresentata dal vettore:
\[
    \vect{v}_{\Lambda_1} = \begin{pmatrix} 1 \\ \omega \end{pmatrix} 
\] 
Procedendo in maniera analoga per la direzione stabile si ottiene:
\[
    \vect{v}_{\Lambda_2} = \begin{pmatrix} 1 \\ -\omega \end{pmatrix} 
\] 
Come possiamo vedere in figura le selle separano lo spazio delle fasi in due regimi: all'interno dell'occhio le orbite sono chiuse (l'orbita è confinata all'interno), all'esterno dell'occhio invece le orbite sono "aperte". Le due regioni non si parlano e sono separate proprio dalle orbite che attraversano le selle.\\
Utilizziamo l'equazione \ref{eq:2_6_1} nel nostro esempio:
\[
    \frac{\text{d} y}{\text{d} x} = \frac{-\omega^2\sin x}{y}
\] 
Quindi la soluzione è subito:
\[
    dy y = -\omega^2\sin xdx \implies  \frac{y^2}{2} = \omega^2  \cos x + c
\] 
Tale relazione ha senso soltanto se:
\[
    \omega^2 \cos x + c \ge 0
\] 
Se $\omega  = 1$ allora è sufficiente prendere $c \ge  1$. In questo caso abbiamo che:
\[
    y = \pm \sqrt{2(\cos x + c)} 
\] 
Quanto vale $c$ nel caso in cui la curva debba passare per $(\pi, 0)$? Deve essere vero che:
\[
    0 = \pm \sqrt{2(-1 + c)} \implies  c = 1
\] 

\end{exmp}
\noindent
