\section{Studio della stabilità mediante linearizzazione}%
\label{sub:Studio della stabilità mediante linearizzazione per SD a tempo continuo autonomo}
Prendiamo il sistema dinamico autonomo a tempo continuo:
\[
    \frac{\text{d} \vect{x}}{\text{d} t} = F(\vect{x}),  \quad \vect{x}\in \mathbb{R}^n, \quad F:\mathbb{R}^n\to \mathbb{R}^n, \quad F \in C^r,\ r\ge 2
\] 
Supponiamo che le soluzioni esistano globalmente. \\
\marginpar{
    \captionsetup{type=figure}
        \incfig{2_3_1}
	\caption{\scriptsize Perturbazione ($y(t)$) della soluzione $x_p(t)$.}
    \label{fig:2_3_1}
}
Figura: Se l'orbita è stabile allora $y(t)$  è confinata ad un tubo di flusso\\
La soluzione allora è definita dalla somma della soluzione imperturbata e del disturbo.
\[
    \vect{x(t)} =\vect{x}_p(t)+\vect{y} (t) \qquad \text{ con } \left|\vect{y} (t)\right|\ll 1
\] 
Quindi possiamo dire che:
\[
    \frac{\text{d} \vect{x}}{\text{d} t} = \frac{\text{d} \vect{x}_p}{\text{d} t} + \frac{\text{d} \vect{t}}{\text{d} t} = F(\vect{x}_p + \vect{y}_p)
\] 
\begin{defn}[Funzione differenziabilie]
    Sia $F:I\to J$ con $I \subset \mathbb{R}^n, \ J \subset \mathbb{R}^n$. $I$ è un aperto e $\vect{x}_0 \in I$.\\
    Si dice che $F$ è differenziabile in $\vect{x}_0$ se $\exists$ $DF(\vect{x}_0) \in L(\mathbb{R}^n, \mathbb{R}^n)$ (spazio delle applicazioni lineari) tale che:
    \[
	\lim_{\left|\vect{h}\right| \to 0} \frac{\left|F(\vect{x}_0+ \vect{h})-F(\vect{x}_0)-DF(\vect{x}_0) \vect{h}\right|}{\left|h\right|} = 0
    \] 
\end{defn}
\noindent
\begin{thm}[Sullo Jacobiano]
    Sia $F:I\to J$, $I \subset \mathbb{R}^n, \ J \subset \mathbb{R}^n$, supponiamo $I$  aperto.\\
    Se $F$  è differenziabile in $\vect{x}_0$  allora:
    \begin{enumerate}
        \item esistono le derivate parziali: $\left.\frac{\partial F_I}{\partial x_J}\right|_{\vect{x}_0} $ con $i, J = 1, 2, \ldots n$. 
	    \item $\forall \ \vect{h}  \in I$ si ha che:
		\[
		    \left.\left[DF(\vect{x}_0) \vect{h}\right]\right|_{i} = 
			\sum_{J=1}^{n} \frac{\partial F_i}{\partial x_J} h_J, 
			\qquad \left.\left[DF(x_0)\right]\right|_{i, J} = \frac{\partial F_i}{\partial X_J} 
		\] 
    \end{enumerate}
\end{thm}
\noindent
Tornando alla nostra linearizzazione:
\[
    \frac{\text{d} \vect{x}_p(t)}{\text{d} t} + \frac{\text{d} \vect{y} (t)}{\text{d} t} = 
    F(\vect{x}_p(t)+ \vect{y} (t)) \simeq 
    F(\vect{x}_p(t))+DF(\vect{x}_p(t))\vect{y}  + O(\vect{y})
\] 
Quindi eliminando l'identità nella precedente equazione ci si riduce alla sola dinamica della perturbazione:
\[
    \frac{\text{d} \vect{y} (t)}{\text{d} t} = DF(\vect{x}_p(t))\vect{y}\equiv
    J(\vect{x}_p)\vect{y}
\] 
Questo campo vettoriale ha una soluzione soluzione stazionaria: 
\[
\vect{y}_s = \begin{pmatrix} 0 \\ \vdots \\ 0 \end{pmatrix}
\] 
L'unica cosa da tenere a mente è che $J(\vect{x}_p(t))$  potrebbe dipendere dal tempo (se $\vect{x}_p(t)$ non è una soluzione stazionaria il sistema non è autonomo). Per adesso ci limitiamo a considerare le soluzioni $\vect{x}_p(t)$  stazionarie: $\vect{x}_s$:
\[
    \frac{\text{d} \vect{y}}{\text{d} t} = J(\vect{x}_s)\vect{y}  \qquad \vect{J(\vect{x}_s)}  \text{: matrice costante}
\] 
\begin{thm}[Stabilità delle soluzioni stazionarie]
    Dato $\frac{\text{d} \vect{x}}{\text{d} t} = F(\vect{x})$ con $\vect{x}\in\mathbb{R}^n$, $F:\mathbb{R}^n \to \mathbb{R}^n$, $F \in C^r, \ r\ge 2$ e sia $\vect{x}_s$ tale che $F(\vect{x}_s)=0$.\\
    Se tutti gli autovalori di $J(\vect{x}_s)$ hanno parte reale negativa allora $\vect{x}_s$ è asintoticamente stabile.
\end{thm}
\noindent
Teniamo presente che la stabilità espressa dal teorema è locale, se si può dimostrare anche che la condizione vale anche per intorni arbitrari allora possiamo decretare anche la stabilità globale.\\
Potrei utilizzare lo stesso approccio per un sistema non autonomo? Gli autovalori della matrice $J$ non autonoma sono ricavabili analiticamente. \\
Se in tal caso trovassi autovalori con parte reale minore di zero potrei concludere la stabilità del sistema? \textbf{NO}.
\begin{exmp}[Il teorema non funziona per sistemi non autonomi]
    \[
	\frac{\text{d} }{\text{d} t} \begin{pmatrix} x_1 \\ x_2 \end{pmatrix} = A(t)\begin{pmatrix} x_1 \\ x_2 \end{pmatrix} 
    \] 
    Con:
    \[
	A(t) = 
	\begin{pmatrix} 
	    -1 + \frac{3}{2}\cos^2t & 1- \frac{3}{2}\sin t\cos t \\
	    -1 -3\sin t \cos t & -1 + \frac{3}{2}\sin^2t
	\end{pmatrix} 
    \] 
    Prendendo il vettore nullo avrei uno stato stazionario per il sistema. Cerchiamo gli autovalori della matrice $A$:
    \[\begin{aligned}
	\text{det}(A(t)- \Lambda  \mathbb{I}) =& (-1 + \frac{3}{2}\cos^2t - \Lambda)(-1+\frac{3}{2}\sin^2t -\Lambda)+\\
					       &+(1+\frac{3}{2}\sin t\cos t)(1-\frac{3}{2}\sin t\cos t)=0
    .\end{aligned}\]
    Dalla equazione secolare si ottiene:
    \[
        \Lambda^2 + \frac{\Lambda}{2}+\frac{1}{2}=0 \implies  \Delta  = \frac{1}{4}-2 = -\frac{7}{4} <0
    \] 
    Quindi abbiamo autovalori complessi coniugati (CC):
    \[
	\Lambda_{12} = \frac{-1 \pm i \sqrt{\frac{7}{4}}}{4} \implies  \text{Re}(\Lambda_{12})=-\frac{1}{4}
    \] 
    Applicando alla lettera il teorema la soluzione stazionaria deve essere stabile. Se consideriamo invece:
    \[
	\vect{x}_1(t)=e^{t /2}\begin{pmatrix} -\cos t \\ \sin t \end{pmatrix}  \qquad \vect{x}_1(t)=e^{-t}\begin{pmatrix} \sin t \\ \cos t \end{pmatrix} 
    \] 
    Si scopre che queste due sono soluzioni indipendenti (per casa). \\
    Quindi prendendo queste due soluzioni per descrivere la soluzione del sistema dinamico e ponendoci in un intorno della soluzione stazionaria si vede che una direzione non è stabile ($\vect{x_1}$), mentre una direzione è stabile $\vect{x}_2$. Il fatto che la soluzione in $\vect{x}_1$ diverga rende il punto $\vect{V}_0 = \begin{pmatrix} 0\\ 0 \end{pmatrix} $ instabile. \\
   Questo dimostra che quando la matrice Jacobiana non è autonoma il teorema non si applica.
\end{exmp}
\noindent
\begin{exmp}[Sistema autonomo]
    \[\begin{dcases}
	\frac{\text{d} x}{\text{d} t} = - y + x(x^2+y^2)\\
	\frac{\text{d} y}{\text{d} t} = x + y(x^2+y^2)
    \end{dcases}
    =
    \begin{pmatrix} F_1(x, y)\\ F_2(x, y) \end{pmatrix} 
    \] 
    L'unico stato stazionario in questo caso è il vettore $\vect{V}_s = (0,0)$.\\
    Per determinare la stabilità come prima cosa dobbiamo calcolare la generica matrice $J$:
    \[
	J(\vect{V})=
        \left.
	\begin{pmatrix} 
	    3x^2 + y^2 & -1 + 2xy \\
	    1 + 2xy & x^2 + 3y^2
	\end{pmatrix} 
        \right|_{\vect{V} = \vect{V}_s} = 
	\begin{pmatrix} 0 & -1 \\ 1 & 0 \end{pmatrix} 
    \] 
    Gli autovalori di questa matrice sono dati dalla equazione secolare:
    \[
        \Lambda^2+1 = 0 \implies  \Lambda_{12} = \pm i
    \] 
    Quindi la parte reale è nulla\ldots\\
    Notiamo che la sola parte lineare di questo sistema rappresenta un oscillatore armonico, l'unico punto fisso dell'oscillatore armonico (l'origine) è stabile secondo Lyapunov. In realtà questa conclusione è errata: lo stato stazionario non è stabile. \\
    Dobbiamo allora stare attenti al fatto che quando qualcuno degli autovalori ha una parte reale nulla c'è bisogno di molta cautela nella interpretazione dei risultati.\\
    Possiamo dimostrare l'instabilità di tale punto fisso sfruttando la simmetria del termine non lineare:
    \[\begin{aligned}
	&x(t)= r(t)\cos (\theta (t))\\
	&y(t)= r(t)\sin (\theta (t))
    .\end{aligned}\]
    con $x^2+y^2 = r^2$.
    \[\begin{aligned}
	&\frac{\text{d} x(t)}{\text{d} t} = \frac{\text{d} r}{\text{d} t} \cos (\theta)- r\sin\theta\frac{\text{d} \theta}{\text{d} t} \\
	&\frac{\text{d} y}{\text{d} t} = \frac{\text{d} r}{\text{d} t} \sin\theta  + r\cos\theta  \frac{\text{d} \theta}{\text{d} t} 
    .\end{aligned}\]
    Mettendo nelle equazioni del moto:
    \[
	-r\sin\theta  \frac{\text{d} \theta}{\text{d} t} + \frac{\text{d} r}{\text{d} t} \cos\theta  = 
	    - r(t)\sin\theta  + r\cos\theta r^2
    \] 
    \[
        r\cos\theta  \frac{\text{d} \theta}{\text{d} t} + \frac{\text{d} r}{\text{d} t} \sin\theta  = r\cos\theta  + r \sin\theta r^2
    \] 
    Moltiplicando la prima equazione per il seno di $\theta$ e la seconda per il coseno di $\theta$ e sottraendo membro a membro le equazioni:
    \[
	r \frac{\text{d} \theta}{\text{d} t} = r \implies  \frac{\text{d} \theta}{\text{d} t} = 1 \implies  \theta (t)=t + \theta_0
    \] 
    Per casa: moltiplicare la la prima equazione per $\cos\theta$ e la seconda per $\sin\theta$ e sommarle. Si ottiene che:
    \[
        \frac{\text{d} r}{\text{d} t} = r^3
    \] 
    Quindi il sistema dinamico di partenza si è ridotto a:
    \[\begin{dcases}
        \frac{\text{d} r}{\text{d} t} = r^3\\
	\frac{\text{d} \theta}{\text{d} t} =1
    \end{dcases}\] 
    Abbiamo un sistema dinamico definito in un manifold: 
    \[
        S^1 \times R^+ \cup \left\{0\right\}
    \] 
    L'equazione interessante è la prima: questa ci dice che il sistema evolve sempre verso $r \to \infty$ per qualsiasi intorno del punto fisso. Quindi lo stato stazionario non è stabile.
\end{exmp}
\noindent
\subsection{Concetto di iperbolicità}%
\label{sub:Concetto di iperbolicità}
Introduciamo un concetto importante nella caratterizzazione degli stati stazionari:
\begin{defn}[Soluzione stazionaria iperbolica]
    Dato il seguente campo vettoriale: $\frac{\text{d} \vect{x}}{\text{d} t} = F(\vect{x})$ con $\vect{x}\in\mathbb{R}^n$, $F:\mathbb{R}^n \to \mathbb{R}^n$ e $\vect{x}_s$ tale che $F(\vect{x}_s)=0$.\\
    Diciamo che $\vect{x}_s$ è una soluzione stazionaria iperbolica se nessuno degli autovalori di $J(\vect{x}_s)$ ha parte reale nulla.
\end{defn}
\noindent
\begin{defn}[Soluzione stazionaria non iperbolica]
    Dato il seguente campo vettoriale: $\frac{\text{d} \vect{x}}{\text{d} t} = F(\vect{x})$ con $\vect{x}\in\mathbb{R}^n$, $F:\mathbb{R}^n \to \mathbb{R}^n$ e $\vect{x}_s$ tale che $F(\vect{x}_s)=0$.\\
    Diciamo che $\vect{x}_s$ è una soluzione stazionaria non iperbolica se non è iperbolica.
\end{defn}
\noindent
\subsection{Classificazione dei possibili stati stazionari non iperbolici in $\mathbb{R}^2$}%
\label{sub:Classificazione dei possibili stati stazionari non iperbolici in R2}
Prendiamo un sistema dinamico:
\[\begin{dcases}
    \frac{\text{d} x}{\text{d} t} = F(x, y)\\
    \frac{\text{d} y}{\text{d} t} = G(x, y)
\end{dcases}\] 
E supponiamo che esista uno stato stazionario $\vect{V}_s$:
\[
    \vect{V }_s = \begin{pmatrix} x_s \\ y_s \end{pmatrix} 
\]
Identifichiamo con $J$ lo Jacobiano:
\[
    J = 
    \begin{pmatrix} 
	\frac{\partial F}{\partial x} & \frac{\partial F}{\partial y} \\
	\frac{\partial G}{\partial x} & \frac{\partial G}{\partial y} 
    \end{pmatrix} 
    = 
    \begin{pmatrix} a & b \\ c & d \end{pmatrix} 
\] 
Cerchiamo le condizioni per il quale lo stato stazionario non è iperbolico: almeno un autovalore di $J$ calcolato in $\vect{V}_s$ deve essere nullo.\\
La prima situazione banale è quella in cui $a = b = c = d = 0$. In tal caso tutti gli autovalori sono nulli. \\
Più in generale utilizziamo un trucchetto per il nostro scopo: introduciamo la traccia $T$ ed il determinante $D$ di $J$. Gli autovalori della matrice possono essere espressi in termini di questi due parametri:
\[
    \text{det}\left[\begin{pmatrix} a & b \\ c & d \end{pmatrix} - \Lambda  \mathbb{I}\right] = 0 
\] 
\[
    \implies  a d - \Lambda (a + d) + \Lambda^2 - c b = \Lambda^2 - T\Lambda + D = 0
\] 
Quindi gli autovalori sono:
\[
    \Lambda_{12} = \frac{T \pm \sqrt{T^2 - 4D}}{2}
\] 
Andiamo ad analizzare in quali situazioni si può avere un autovalore nullo: 
\begin{enumerate}
    \item Matrice nulla. 
    \item Se $T=0$, $D > 0$ (resta solo una radice negativa)
    \item Se $T = 0$, $D = 0$.
    \item $D = 0$ e $T \neq 0$ $\implies$ $\Lambda_{12} = \frac{T \pm \sqrt{T^2} }{2}$  
\end{enumerate}
In dimensione maggiore di $2$ la chiave è caratterizzare gli zeri del polinomio caratteristico.\\
Supponiamo di avere un polinomio: 
\[\begin{aligned}
    &P(x) = a_0 x^{n}+ a_1x^{n-1} + a_2 x^{n-2}+ \ldots + a_{n} \\
    &\text{ con  } a_i \in \mathbb{R}, \ \forall  \ i = 0 , \ldots, n \ (a_0\neq 0)
.\end{aligned}\]
Cerchiamo gli zeri di questo polinomio caratteristico.\\
Il teorema fondamentale dell'algebra ci assicura $n$ soluzioni per questo polinomio. \\ 
Data una soluzione $\overline{x}$ si ha che $P(\overline{x}) = 0$ con $\overline{x}\in \mathbb{C}$ allora si ha che $\overline{x}^*$ è anch'esso radice del polinomio.
\begin{thm}[Regola dei segni]
    Dato il polinomio $P(x)$ se nella sequenza:
    \[
        a_0, a_1, a_2, \ldots, a_n
    \] 
    ci sono $k$ variazioni del segno tra $a_i $ e $a_{i+1}$ allora il numero di radici positive è $k$ o $k-2m$ con $m \in \mathbb{N}$.
\end{thm}
\noindent
\begin{exmp}[Sulla regola dei segni]
    Prendiamo i seguenti due polinomi:
    \begin{itemize}
        \item 
	    \[
        	x^3 + 4 x^2 - 5x - 4 
        	\] 
        	In tal caso le radici sono: $(1, -4, -1)$. Quindi $k = 1$.
	\item 
        \[
        x^3 + 6 x^2 + 9 x + 4
    \] 
    Le radici sono (-1. -4, -1) e $k=0$.
    \end{itemize}
\end{exmp}
\noindent
\subsection{Tabella di Rauth}%
\label{sub:Tabella di Rauth}
Riprendiamo il polinomio generico:
\[\begin{aligned}
    &P(x) = a_0 x^{n}+ a_1x^{n-1} + a_2 x^{n-2}+ \ldots + a_{n} \\
    &\text{ con  } a_i \in \mathbb{R}, \ \forall  \ i = 0 , \ldots, n \ (a_0\neq 0)
.\end{aligned}\]
La tabella di Routh si costruisce prendendo i termini $a_{2i}$ ed i termini $a_{2i + 1}$ separatamente.
\[
    \begin{matrix}
	   & \left\{r_{1,i}\right\}: & a_0  & a_2  & a_4  &\ldots\\
	   & \left\{r_{2, i}\right\}:& a_1 & a_3  & a_5 & \ldots\\
	   & \left\{r_{3, i}\right\}: & r_{3, 1} & r_{3,2} & r_{3,3} & \ldots\\
	   & \left\{r_{4, i}\right\}: & r_{4, 1} & r_{4,2} & r_{4,3} & \ldots
    \end{matrix}
\] 
Prendiamo $i \ge 3$, in tal caso possiamo definire la sequenza:
\[\begin{aligned}
    \left(r_{i, 1} \quad r_{i, 2} \quad r_{i, 3} \quad \ldots\right) = &\left(r_{i-2, 2} \quad r_{i-2, 3} \quad r_{i-2, 4} \quad \ldots\right) + \\
    & - \frac{r_{i-2, 1}}{r_{i-1,1}}(r_{i-1,2} \quad r_{i-1,3} \quad r_{i-1,4} \quad \ldots)
.\end{aligned}\]
Ad esempio abbiamo che:
\[
    r_{i,1}= r_{i-2,2} - \frac{r_{i-2,1}}{r_{i-1,1}}r_{i-1,2} = 
    \frac{r_{i-2,2}r_{i-1,1}- (r_{i-2, 1}r_{i-1,2})}{r_{i-1,1}}
\] 
Possiamo anche calcolare un altro termine:
\[\begin{aligned}
     r_{3,1} = r_{1,2}- &\frac{r_{1,1}}{r_{2,1}}r_{2,2} = a_2 - \frac{a_0}{a_1} a_3 = \\
    &=\frac{ a_1a_2-a_0a_3}{a_1} = - \frac{\text{det}\begin{pmatrix} a_0 & a_2 \\ a_1 & a_3 \end{pmatrix} }{a_1}
.\end{aligned}\]
Per casa verificare che:
\[
    r_{3,2} = - \frac{\text{det}\begin{pmatrix} a_0 & a_4 \\ a_1 & a_5 \end{pmatrix} }{a_1}
\] 
\begin{exmp}[Calcolo della tabella di Routh]
    Prendiamo il polinomio:
    \[
	P(x)=x^3 + 6x^2 + 11x + 6 
    \] 
    Quindi $a_0 = 1, \ a_1 = 6 , \ a_2 = 11, \ a_3 = 6$. \\
    Costruiamo la tabella di Rauth:
    \[\begin{aligned}
	&\left\{r_{1,i}\right\}:  1 \quad 11 \quad 0 \\
	&\left\{r_{2,i}\right\}: 6 \quad 6 \quad 0
    .\end{aligned}\]
    Calcoliamo anche gli elementi della terza riga:
    \[
	r_{3, 1}= \frac{r_{2,1}r_{1,2}-r_{1,1}r_{2,2}}{r_{2,1}}= \frac{6 \cdot  11 - 1 \cdot  6}{6} = \frac{60}{6} = 10
    \] 
    Per casa dimostrare che:
    \[
        r_{3,2} = 0
    \] 
    Questo perché questo termine deriva da una colonna di zeri nella tabella di Routh. Proviamo a calcolare altri termini:
    \[
        r_{4, 1} = \frac{6 \cdot r_{3, 1}}{10} = 6 \qquad r_{4, 2} = 0
    \] 
    L'annullarsi del secondo coefficiente deriva ancora una volta dalla colonna di zeri nella tabella.\\
    La tabella finale è quindi composta da\sidenote{
    \[
        \begin{matrix}
	    & 1 & 11& 0 \\
	    & 6 & 6 & 0 \\
	    & 10& 0 & 0 \\
	    & 6 & 0 & 0 
        \end{matrix}
    \]}.\\
     In questo caso abbiamo che, per il teorema sotto, le radici del polinomio sono tutte negative (in parte reale). Per completezza le radici sono sono: $-1, -2, -3$.
\end{exmp}
\noindent
\begin{thm}[Teorema di Routh-Hurwitz]
    Dato il polinomio $P(x)$ di ordine $n$. Allora le sue radici hanno parte reale strettamente minore di $0$ se e solo se gli elementi della prima colonna della tabella di Routh sono diversi da zero e hanno tutti lo stesso segno.
\end{thm}
\noindent
In inoltre ad ogni cambiamento di segno corrisponde una radice reale positiva.
\begin{exmp}[Sul teorema di RH]
    \[
	P(x)= x^3 + 4x^2 - 5x - 4
    \] 
    La tabella di Routh si scrive come:
    \[\begin{aligned}
	\left\{r_{1, i}\right\}: &1 \quad - 5 \quad 0 \\
	\left\{r_{2, i}\right\}: &4 \quad - 4 \quad 0 \\
	   &r_{3, 1} \quad r_{3, 2} \quad \\
	   &r_{4, 1} \quad r_{4,2}
    .\end{aligned}\]
    \[
        r_{3,1}= \frac{r_{1, 2}r_{2, 1}- r_{1, 1}r_{2, 2}}{r_{2, 1}} = -4 \qquad r_{3, 2}= 0 \qquad r_{4, 1} = 0
    \] 
    L'ultimo termine non banale è:
    \[
	r_{4, 1} = r_{2, 2}- \frac{r_{2, 1}}{r_{3, 1}}r_{3,3} = \frac{-4 \cdot -4 - (4)\cdot 0 }{-4} = -4
    \] 
    Quindi la tabella finale è\sidenote{
    \[
        \begin{matrix}
	    1  & -5 & 0 \\
	    4  & -4 & 0 \\
	    -4 & 0  & 0 \\
	    -4 & 0  & 0
        \end{matrix}
    \] }.\\
    Visto che c'è un cambiamento di segno allora non tutti gli autovalori hanno parte reale minore di zero ed inoltre ce n'è una positiva. \\
    Per casa verificare che le radici del polinomio sono: $(1, -4, -1)$.
\end{exmp}
\noindent
\subsection{Terminologia associata alla classificazione degli stati stazionari costanti per SD a tempo continuo autonomi}%
\label{sub:Terminologia associata alla classificazione degli stati stazionari costanti}
Preso il solito sistema dinamico:
\[
    \frac{\text{d} \vect{x}}{\text{d} t} = F(\vect{x}) \qquad \vect{x}\in \mathbb{R}^n \quad F:\mathbb{R}^n\to \mathbb{R}^n
\] 
e preso $\vect{x}_s$ stato stazionario.
\paragraph{1) $\vect{x}_s$ iperbolico.}%
\begin{defn}[Punto sella]
    \marginpar{
        \captionsetup{type=figure}
            \incfig{2_3_2}
        \caption{\scriptsize Punto sella: le curve indicano 4 direzioni che devono necessariamente esistere.}
        \label{fig:2_3_2}
    }
    Si dice che $\vect{x}_s$ è punto sella se gli autovalori della matrice Jacobiana $J(\vect{x}_s)$ hanno parti reali sia positive che negative.
\end{defn}
\noindent
\begin{defn}[Pozzo]
    \marginpar{
        \captionsetup{type=figure}
            \incfig{2_3_3}
        \caption{\scriptsize Pozzo.}
        \label{fig:2_3_3}
    }
    $\vect{x}_s$ è un pozzo se tutti gli autovalori di $J(\vect{x}_s)$ hanno parte reale negativa.
\end{defn}
\noindent
\begin{defn}[Sorgente]
    \marginpar{
        \captionsetup{type=figure}
            \incfig{2_3_4}
        \caption{\scriptsize Sorgente.}
        \label{fig:2_3_4}
    }
    $\vect{x}_s$ è una sorgente se tutti gli autovalori di $J(\vect{x}_s)$ hanno parte reale positiva.
\end{defn}
\noindent
\paragraph{2) Non iperbolico}%
\begin{defn}[Centro]
    $\vect{x}_s$ è un centro se gli autovalori di $J(\vect{x}_s)$ sono tutti immaginari puri e diversi da zero.
\end{defn}
\noindent
\begin{exmp}[1]
    Prendiamo il sistema dinamico in $\mathbb{R}^2$:
    \[\begin{dcases}
	\frac{\text{d} x}{\text{d} t} = 3x + y = F_1(x, y)\\
	\frac{\text{d} y}{\text{d} t} = 3y = F_2(x, y)
    \end{dcases}\] 
    Lo stato stazionario è definito da:
    \[
	\vect{V}_s: \ F(\vect{V}_s) = 0 \ \implies  \vect{V}_s = \begin{pmatrix} 0 \\ 0 \end{pmatrix} 
    \] 
    Lo Jacobiano è dato da:
    \[
	J = \begin{pmatrix} 3 & 1 \\ 0 & 3 \end{pmatrix} 
    \] 
    Il polinomio caratteristico del sistema è:
    \[
	\text{det}\left[J-\Lambda\mathbb{I}\right] = 0 \implies  (3-\Lambda)^2 = 0
    \] 
    Gli autovalori sono $3$ con molteplicità 2. Questo implica che $\vect{V}_s$ è una sorgente.
\end{exmp}
\noindent
\begin{exmp}[Oscillatore armonico]
    Preso l'oscillatore armonico:
   \[\begin{dcases}
       \frac{\text{d} x}{\text{d} t} = y\\
       \frac{\text{d} y}{\text{d} t} = -x
   \end{dcases}\]  
   Sappiamo che lo stato stazionario è l'origine, inoltre la matrice Jacobiana è:
   \[
       J = \begin{pmatrix} 0 & 1 \\ -1 & 0 \end{pmatrix} 
   \] 
   Quindi abbiamo subito che:
   \[
       \Lambda^2 + 1 = 0 \implies  \Lambda_{1, 2} = \pm i
   \] 
   Nella definizione espressa sopra questo implica che il punto stazionario è un centro.
\end{exmp}
\noindent
\begin{exmp}[Sistema in $\mathbb{R}^2$]
    \[\begin{dcases}
        \frac{\text{d} x}{\text{d} t}  = - x - 10 y \\
	\frac{\text{d} y}{\text{d} t} = 10x - y
    \end{dcases}\] 
    L'unico stato stazionario è $\vect{V}_s = \begin{pmatrix} 0 \\ 0 \end{pmatrix} $, lo Jacobiano è invece:
    \[
	J = \begin{pmatrix} -1 & -10 \\ 10 & -1 \end{pmatrix} \implies\text{det}\left[J-\Lambda\mathbb{I}\right] = 0
    \] 
    \[
	\left(\Lambda+1\right)^2+ 100 = 0 \implies  P(\Lambda)=\Lambda^2 + 2\Lambda  + 101 = 0
    \] 
    Il discriminante di questa equazione è: $\Delta =-400$ quindi abbiamo due soluzioni complesse:
    \[
        \Lambda_1=-1-10i, \qquad \Lambda_2 = -1+10i
    \] 
    Quindi questo è un pozzo poiché la parte reale è minore di zero.\\
    Se volessimo applicare il teorema di Routh dobbiamo calcolare la tabella:
    \[
        \begin{matrix}
	    1 & 101 & 0 \\
	    2 & 0   & 0
        \end{matrix}
	\implies  r_{i, J}=- \frac{\text{det}\begin{pmatrix} r_{i-2,1} & r_{i-2, J+1} \\ r_{i-1, 1} & r_{i-1, J+1} \end{pmatrix}}{r_{i-1,1}}
    \] 
\end{exmp}
\noindent
\begin{exmp}[Di punto sella]
    \[\begin{dcases}
        \frac{\text{d} x}{\text{d} t} = x + 5 y\\
	\frac{\text{d} y}{\text{d} t} = -3y
    \end{dcases}\] 
    L'unico stato stazionario è anche in questo caso l'origine. 
    \[
	J = \begin{pmatrix} 1 & 5 \\ 0 & -3 \end{pmatrix} \implies  P(\Lambda)=(1-\Lambda)(-3-\Lambda)=0
    \] 
    Quindi $\Lambda_1=1$ e $\Lambda_2=-3$: abbiamo una sella.
\end{exmp}
\noindent
\begin{exmp}[Oscillatore di Duffling]
    \[
    \ddot{x}+ k \dot{x}+ \alpha x + \beta x^3 = A\cos (\omega t) \qquad \alpha =1, \ \beta  = -1, \ A = 0, \ k \ge 0
    \] 
    Il sistema si riduce a:
    \[\begin{dcases}
	\frac{\text{d} x}{\text{d} t} = y = F_1(x, y)\\
	\frac{\text{d} y}{\text{d} t} = -ky -x+x^3 = F_2(x,y)
    \end{dcases}\] 
    Gli stati stazionari $\vect{V}_s$ sono tali per cui $F(\vect{V}_s)=0$.\\
    Per casa verificare che:
    \[
        \vect{V}_{1,s}=\begin{pmatrix} 0\\ 0 \end{pmatrix} \qquad \vect{V}_{2,3,s}= \begin{pmatrix} \pm 1 \\ 0 \end{pmatrix} 
    \] 
    Possiamo trovare lo Jacobiano:
    \[
	J(x,y)=
	\begin{pmatrix} 0 & 1 \\ 3x^2-1 & -k \end{pmatrix} \equiv J(\vect{V})
    \] 
    Per calcolare gli autovalori del sistema dobbiamo metterci in uno stato stazionario:
    \[
	J(\vect{V}_{1,s})=\begin{pmatrix} 0 & 1 \\ -1 & -k \end{pmatrix} \implies  \text{det}\left[J-\Lambda\mathbb{I}\right]=0
    \] 
    \[
	P(\Lambda)=\Lambda^2+k\Lambda +1=0
    \] 
    Possiamo utilizzare il metodo di Cartesio: essendo $k$ positivo allora il polinomio non ha variazione di segno nelle soluzioni, sono tutte negative.\\
    Il discriminante vale:
    \[
        \Delta =k^2-4
    \] 
    Quindi, essendo $k>0$ serve $k\ge 2$ per avere radici reali. In questo caso si ha che:
    \[
        \Lambda_{12} = \frac{-k \pm \sqrt{k^2-4} }{2}
    \] 
    Quindi anche nel caso in cui il discriminante è reale le radici restano sempre entrambe negative.\\
    Se invece prendiamo $0<k<2$ allora si ha che il $\Delta <0$ e le due radici hanno comunque parte reale negativa: $-k$.\\
    Prendiamo un altro stato stazionario adesso: $\vect{V}_{2,s}$:
    \[
	J(\vect{V}_{2,s}) = \begin{pmatrix} 0 & 1 \\ 2 & -k \end{pmatrix} 
    \] 
    Quindi abbiamo il polinomio caratteristico:
    \[
	P(\Lambda)= (\Lambda+k)\Lambda  - 2 = \Lambda^2+k\Lambda-2=0
    \] 
    Usando Cartesio essendo $k$ positivo c'è una variazione di segno: c'è una radice positiva. Questo implica che il punto non è stabile.
    \[
        \Lambda_{12} = \frac{-k\pm\sqrt{k^2+8} }{2}
    \] 
    Quindi lo stato stazionario è una sella. Per casa si può verificare che anche $\vect{V}_{3,s}$ è una sella.
\end{exmp}
\noindent
