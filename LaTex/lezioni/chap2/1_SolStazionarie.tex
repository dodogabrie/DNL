\section{Soluzioni stazionarie di SD}%
\label{sub:Sistema dinamico a tempo continuo}
\subsection{Sistema dinamico a tempo continuo}%
\label{sub:Soluzioni stazionarie di SD a tempo continuo}
\begin{defn}[Stato stazionario o Soluzione Stazionaria per SD autonomo]
    Preso il sistema dinamico:
    \[
	\frac{\text{d} \vect{x}}{\text{d} t} = F(\vect{x}) \qquad F: U\to \mathbb{R}^n;  F \in C^r \ (r\ge 1); \ \vect{x}\in\mathbb{R}^n 
    \] 
    Uno stato $\vect{x}_s \in \mathbb{R}^n$ si dice stazionario se è soluzione del SD e vale che $F(\vect{x}_s) = 0$.
\end{defn}
\noindent
La definizione non è valida nel caso di sistemi non autonomi.
\begin{exmp}[Sistema non autonomo non ha sol. Stazionarie]
    Prendiamo il seguente:
    \[\begin{dcases}
        \frac{\text{d} x}{\text{d} t} = - x + t\\
	x(0)=x_0
    \end{dcases}\] 
    In questo caso la soluzione è dipendente dal tempo (in modo indipendente da $t-t_0$):
    \[
	x(t)= e^{-t}(x_0+1)+t-1
    \] 
    Quindi non può esistere la soluzione stazionaria in questo caso: non esiste una soluzione che annulli la $F$ al variare di $t$.
\end{exmp}
\noindent
Vediamo adesso un esempio molto esplicativo per il Phase Portrait e per le soluzioni stazionarie.
\begin{exmp}[Sistema non lineare con parametro]
    \[
	\frac{\text{d} x}{\text{d} t} = -x + ax^3 \equiv F(x)
    \] 
    Le soluzioni stazionarie devono rispettare la seguente equazione:
    \[
        -x_s + ax_s^3 = 0 \implies  
	\begin{cases}
	    x_s = 0 & \forall a \in \mathbb{R}\\
	    x_s = \pm 1 / \sqrt{a} & \forall a > 0
        \end{cases}
    \] 
    Si vede che al variare del parametro di controllo $a$ compaiono o scompaiono multipli punti fissi, questa è una peculiarità dei sistemi non lineari che approfondiremo in seguito. 
    \begin{description}
	\item[1) $a = 0$.] In questo caso il sistema è lineare:
	    \marginpar{
	        \captionsetup{type=figure}
		\incfig{2_1_1}
		\caption{\scriptsize Caratterizzare la dinamica prendendo delle condizioni iniziali vicine al punto fisso nel caso $a=0$.}
	        \label{fig:2_1_1}
	    }
	    \[
	        \frac{\text{d} x}{\text{d} t} = -x
	    \] 
	    Vogliamo classificare l'unica soluzione stazionaria in $x_s = 0$. Applicando una perturbazione a questa soluzione il sistema torna a stazionarietà o inizia una evoluzione diversa? \\
	    Per rispondere a questa domanda si può prendere delle condizioni iniziali a destra ed a sinistra dell'unico punto fisso come in figura \ref{fig:2_1_1}: $x_0^+, x_0^-$. \\
	    Si può subito notare che in $x_0^+$ si ha $F(x)$ negativa, quindi il punto tenderà ad avvicinarsi all'origine, viceversa per $x_0^-$. La soluzione stazionaria è quindi stabile.
        \item[2) $a < 0$.] 
	    In questo caso l'equazione del moto diventa:
	    \marginpar{
	        \captionsetup{type=figure}
	            \incfig{2_1_2}
		    \caption{\scriptsize Caratterizzare la dinamica prendendo delle condizioni iniziali vicine al punto fisso nel caso $a<0$, in arancio il punto fisso.}
	        \label{fig:2_1_2}
	    }
	    \[
	        \frac{\text{d} x}{\text{d} t} = -x + ax^3
	    \] 
	    Le orbite hanno lo stesso comportamento del caso analizzato in precedenza, qui però si ha un avvicinamento all'origine non lineare per via del termine cubico (figura \ref{fig:2_1_2}).
	\item[3) $a>0$.]
	    \marginpar{
	        \captionsetup{type=figure}
	            \incfig{2_1_3}
	        \caption{\scriptsize Caratterizzare la dinamica prendendo delle condizioni iniziali vicine al punto fisso nel caso $a>0$, in arancione le 3 soluzioni stazionarie.}
	        \label{fig:2_1_3}
	    }
	    La dinamica di questo caso è più ricca delle due precedenti per via degli ulteriori due punti fissi in $\pm 1 /\sqrt{a}$. Lo stesso $F(x)$ in questo caso presenta un andamento diverso: con $a>0$ non è più monotono, decrescente ma assume la forma di figura \ref{fig:2_1_3}.\\
	    Le direzioni sono tracciate sempre valutando il segno di $F(x)$, notiamo subito che il punto nell'origine attrae la dinamica (è ancora stabile) mentre le altre due soluzioni stazionare non godono della stessa proprietà.\\
	    Ponendo un punto nei pressi di $x_s = \pm 1 /\sqrt{a} $ il SD tenderà a divergere o ad avvicinarsi a $x=0$, queste soluzioni sono quindi stazionarie ma instabili.
    \end{description}
\end{exmp}
\noindent
L'esempio precedente mostra che per risolvere il sistema e determinare la dinamica non è sempre necessario trovare la soluzione analitica, è possibile determinare i punti fissi e valutarne la stabilità.\\
In questo modo si ottiene il quadro complessivo dell'evoluzione del sistema (possiamo disegnare una approssimazione del Phase Portrait). Questo tipo di approccio è stato inventato da un grande esperto di sistemi dinamici: Henry Poicaré.
\subsection{Interpretazione fisica: Gradient Dynamical System}%
\label{sub:Interpretazione fisica: Gradient Dynamical System}
Quando è possibile esprimere il SD (a tempo continuo, autonomo) nel seguente modo:
\[
    \frac{\text{d} \vect{x}}{\text{d} t} = F(\vect{x}) = - \frac{\text{d} V(\vect{x})}{\text{d} t} 
\] 
Allora il sistema si presta ad una interpretazione intuitivamente semplice: $V(\vect{x})$  rappresenta il potenziale in cui il corpo che compie la traiettoria $\vect{x} (t)$ si trova immerso.\\
Riprendendo l'esempio unidimensionale visto sopra:
\marginpar{
    \captionsetup{type=figure}
        \incfig{2_1_4}
    \caption{\scriptsize Andamento del potenziale per l'esempio sopra nel caso $a>0$, i punti arancioni corrispondono alle 3 soluzioni stazionarie.}
    \label{fig:2_1_4}
}
\[
    \frac{\text{d} x}{\text{d} t} = - x + a x^3 = - \frac{\text{d} V(x)}{\text{d} t} 
\] 
Possiamo integrare per ottenere il potenziale:
\[
    V(x)=\frac{x^2}{2}-\frac{a}{4}x^4
\] 
Tale potenziale gode delle seguenti proprietà:
\begin{itemize}
    \item è simmetrico $V(x)=V(-x)$.
    \item $\lim\limits_{x \to \pm\infty} V(x)=-\infty$.
    \item Si annulla in $(0, \ \pm \sqrt{2 / a})$ se $a>0$, altrimenti si annulla solo nell'origine.
\end{itemize}
Per $a>0$ il potenziale assume la forma a doppio monte in figura \ref{fig:2_1_4}, negli altri due casi invece si ha un paraboloide con minimo in $x=0$: l'unica soluzione stazionaria.
\begin{exmp}[Punti fissi dell'oscillatore di Duffling]
    Analizziamo la seguente equazione differenziale:
    \[
	\frac{\text{d} ^2x}{\text{d} t^2} + k \frac{\text{d} x}{\text{d} t} + \alpha x + \beta x^3=A\cos (\omega t)
    \] 
    Valutiamo il sistema nel caso semplificato:
    \[
        A = 0 \quad \alpha  = 1 \quad \beta  = -1 \quad k > 0
    \] 
    Selezionare l'ultimo parametro nel dominio positivo ($k>0$) significa dire che il sistema presenta dissipazione.\\
    Conduciamo il SD ad un sistema di equazioni differenziali del primo ordine:
    \[\begin{dcases}
	\frac{\text{d} x}{\text{d} t} = y \equiv F_1(x, y)\\
	\frac{\text{d} y}{\text{d} t} = -ky - x + x^3 \equiv F_2(x, y)
    \end{dcases}\] 
    Possiamo ricavare i punti fissi richiedendo l'annullamento di $F = (F_1, F_2)$:
    \[\begin{dcases}
        y = 0\\
	-ky - x + x^3 = 0
    \end{dcases}\] 
    Prendendo il caso semplice in cui $k = 0$, è immediato trovare i seguenti punti fissi:
    \[
        V_{1s} = \begin{pmatrix} 0 \\ 0 \end{pmatrix} \qquad
        V_{2s} = \begin{pmatrix} 1 \\ 0 \end{pmatrix} \qquad
        V_{3s} = \begin{pmatrix} -1 \\ 0 \end{pmatrix}
    \] 
    Lo studio della stabilità di questi punti non è scontato. Si deve considerare le direzioni di tutte le orbite in $x$ e in $y$ a destra e sinistra di ogni punto fisso.
\end{exmp}
\noindent
\subsection{Stati Stazionari di SD a tempo discreto autonomo}%
\label{sub:Stati Stazionari di SD a tempo discreto autonomo}
\begin{defn}[Stato Stazionario SD a tempo discreto]
    Data la mappa $\vect{x}_{k+1}= G(\vect{x}_k)$ con $G: U \subset \mathbb{R}^n \to \mathbb{R}^n$ e $\vect{x}_k \in U$.\\
    Una soluzione $\vect{x}_{s}$ si dice stazionaria se:
    \[
	\vect{x}_s = G(\vect{x}_s)
    \] 
    Questo in termini di risposta del sistema implica che l'input deve essere uguale all'output.
\end{defn}
\noindent
\begin{exmp}[Mappa logistica]
    Prendiamo la solita mappa logistica:
    \[
	x_{k+1}= \mu x_k(1-x_k) \quad x_k \in \left[0,1\right]; \ \mu  \in \left[0, 4\right]
    \] 
    La richiesta di stato stazionario si traduce in:
    \[
	x_s = G(x_s) \implies  x_s = \mu x_s(1-x_s)
    \] 
    Risolvendo l'equazione si trovano i candidati:
    \[\begin{aligned}
	&x_{s_1}= 0\\
	& x_{s_2}=\frac{\mu-1}{\mu}
    .\end{aligned}\]
    Visto che la dinamica è definita tra $0$ e $1$ la condizione di esistenza del punto fisso $x_{s_2}$ è $\mu >1$.
\end{exmp}
\noindent
\begin{exmp}[Stati stazionari della Mappa di Henon]
   \[\begin{dcases}
       x_{n+1}=1+y_n - \alpha x_n^2\\
       y_{n+1}=\beta x_n
   \end{dcases}\]  
   Cerchiamo uno stato stazionari $\vect{V}_s = \begin{pmatrix} x_s \\ y_s \end{pmatrix} $ tale che:
   \[
       \vect{V}_s = G(\vect{V}_s)
   \] 
   Quindi serve che:
   \[\begin{dcases}
       x_{s}=1+y_s - \alpha x_s^2\\
       y_{s}=\beta x_s
   \end{dcases}
   \implies 
   \begin{dcases}
       x_s = 1 + \beta x_s - \alpha x_s^2\\
       y_s = \beta x_s
   \end{dcases}
   \begin{dcases}
       \alpha x_s^2 + x_s(1-\beta)-1 = 0\\
       y_s = \beta x_s
   \end{dcases}
   \] 
   Cercando soluzioni reali la condizione di esistenza per la prima equazione è:
   \[
       (1-\beta)^2 + 4\alpha  \ge 0 \implies  \alpha  \ge \frac{-(1-\beta)^2}{4}
   \] 
   Scegliendo valori per il quale la mappa presenta un comportamento complesso:
   \[
       \alpha  = 1.4, \ \beta  = 0.3
   \] 
   Abbiamo che la condizione di esistenza è rispettata.\\
   Le soluzioni stazionarie della mappa sono:
   \[
       \vect{V}_{s_1}= \begin{pmatrix} \frac{-(1-\beta)+ \sqrt{(1-\beta)^2+4\alpha} }{2\alpha}\\ \beta x_s \end{pmatrix} \quad
       \vect{V}_{s_1}= \begin{pmatrix} \frac{-(1-\beta)- \sqrt{(1-\beta)^2+4\alpha} }{2\alpha}\\ \beta x_s \end{pmatrix} 
   \] 
\end{exmp}
\noindent

