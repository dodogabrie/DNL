\section{Sistemi lineari in dimensione $n$}%
\label{sub:Sistemi lineari in dimensione n}
Le proprietà dei sistemi di $\mathbb{R}^2$  ci permettono di caratterizzare quasi completamente anche quello che succede in $\mathbb{R}^n$.\\
Supponiamo di avere un sistema dinamico autonomo:
\[
    \frac{\text{d} \vect{x}}{\text{d} t} = F(\vect{x}) \qquad \vect{x}\in \mathbb{R}^n, \qquad F: \mathbb{R}^n \to \mathbb{R}^n \in C^r, r\ge 2
\] 
Dato uno stato stazionario $\vect{x}_s$: $F(\vect{x}_s) = 0$ vogliamo riutilizzare tutta la metodologia affrontata per $\mathbb{R}^2$  anche in questo caso multidimensionale.
\[
    \frac{\text{d} \vect{y}}{\text{d} t} = A \vect{y}
\] 
Nel caso in cui $A$ è una matrice reale con $n$ autovalori distinti e reali allora possiamo diagonalizzare la matrice e integrare subito il sistema essendo completamente separabile.\\
\begin{thm}[]
    Sia $A$ matrice reale e sia $\Lambda \in \mathbb{C}$ un autovalore\sidenote{\scriptsize Ci sarà anche $\Lambda^*$: il suo complesso coniugato} di $A$ ($\text{det}(A-\Lambda\mathbb{I})=0$) con $\vect{W}$ il relativo autovettore. Possiamo dire che l'autovalore del complesso coniugato è:
    \[
        A\vect{W}^* = \Lambda\vect{W}^*
    \] 
    Ovvero anche gli autovettori sono complessi coniugati.
\end{thm}
\noindent
\begin{thm}[]
    Dato il SD: 
    \[
        \frac{\text{d} \vect{x}}{\text{d} t} = A \vect{x} \qquad \vect{x}\in \mathbb{R}^{2n}
    \] 
    con $A$ che possiede $2n$ autovalori distinti 
    \[
	\Lambda_J = a_J + i b_J \quad \text{e} \quad \Lambda^*_{J} = a_J - i b_{J} \quad \text{con} \quad J = 1, 2, \ldots, n.
    \] 
    Siano $\vect{W}_J = \vect{u}_J + i \vect{v}_J$ e $\vect{W}_J = \vect{u}_J + i \vect{v}_J$ ($\vect{u}_J, \vect{v}_J$ reali). Allora si ha:
    \begin{enumerate}
        \item L'insieme $\left\{u_1,u_2,\ldots, u_n, v_1, v_2, \ldots, v_n\right\}$ è una base di $\mathbb{R}^{2n}$.
	\item La matrice $P = \left[v_1 u_1 v_2 u_2 \ldots v_n u_n\right]$ è invertibile.
	\item $P^{-1}AP = \text{diag}\left\{\begin{pmatrix} a_J & -b_J \\ b_J & a_J \end{pmatrix} \right\}$.
	\item La soluzione dell'IVP $\left\{\frac{\text{d} \vect{x}}{\text{d} t} = A \vect{x} \qquad \vect{x}(0) = \vect{x}_0\right\}$  è data da:
	    \[
		\vect{x} (t)=P\text{diag}\left\{e^{a_{J}t}\begin{pmatrix} \cos (b_Jt) & - \sin (b_Jt) \\ \sin (b_Jt) & \cos (b_{j}t) \end{pmatrix} \right\} P^{-1}\vect{x}_0
	    \] 
    \end{enumerate}
\end{thm}
\noindent
Notiamo la totale simmetria rispetto ai teoremi di $\mathbb{R}^2$.
\begin{exmp}[]
    Dato il sistema dinamico:
    \[
	\frac{\text{d} \vect{x}}{\text{d} t} = A \vect{x}  \qquad 
	A = 
	\begin{pmatrix} 
	    1 & - 1 & 0 & 0 \\
	    1 & 1 & 0 & 0 \\
	    0 & 0 & 3 & -2 \\
	    0 & 0 & 1 & 1
        \end{pmatrix} 
    \] 
    Gli autovalori si calcolano tramite l'equazione secolare:
    \[\begin{aligned}
	\ldots \quad &(1-\Lambda)^2\left[(3-\Lambda)(1-\Lambda)+2\right]+\left[(3-\Lambda)(1-\Lambda)+2\right] = \\
		     &\left[(1-\Lambda)^2 + 1\right]\left[(3-\Lambda)(1-\Lambda) +2\right]= 0
    .\end{aligned}\]
    Quindi abbiamo il set di autovalori:
    \[
        \Lambda_{12} = 1 \pm i \qquad \Lambda_{34} = 2\pm i
    \] 
    In questo modo possono essere inseriti i vari parametri nella matrice di cambio di variabile del teorema.
\end{exmp}
\noindent
