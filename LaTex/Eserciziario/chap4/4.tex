\begin{ex}[]
    Sia dato il sistema dinamico a tempo continuo:
    \[
    \begin{dcases}
	\frac{\text{d} x}{\text{d} t} = -y + x\left[(x^2+y^2) ^2 - 3 (x^2+y^2) + 1\right]\\
	\frac{\text{d} y}{\text{d} t} = rx + y\left[(x^2+y^2) ^2 - 3 (x^2+y^2) + 1\right]
    \end{dcases}
    \]
    Utilizzando il teorema di Poincare-Bendixon dimostrare che tale sistema dinamico ha (almeno) una orbita periodica.
\end{ex}
\noindent
\begin{ex}[]
    Prendiamo il sistema dinamico:
    \[
    \begin{dcases}
    \frac{\text{d} x}{\text{d} t} = x - y -x^3\\
    \frac{\text{d} y}{\text{d} t} = x - y -y^3\\
    \end{dcases}
    \]
    Utilizzando il teorema di Poincare-Bendixon dimostrare che tale sistema dinamico ha (almeno) una orbita periodica.
\end{ex}
\noindent
\begin{ex}[]
    I sistemi dinamici non autonomi possono possedere orbite chiuse non autonome.\\
    Dato il sistema dinamico a tempo continuo non autonomo:
    \[
    \begin{dcases}
    \frac{\text{d} x}{\text{d} t} = nt^{n-1}y\\
    \frac{\text{d} y}{\text{d} t} = -nt^{n-1}x
    \end{dcases}
    \]
    Con $n\in \mathbb{N}$ e $n>1$.\\
    \begin{enumerate}
	\item Determinare la soluzione generale (pensare all'oscillatore armonico). 
	\item Dimostrare che esistono orbite chiuse non periodiche (verificare che $x^2 + y^2 \ldots$).
    \end{enumerate}
\end{ex}
\noindent
