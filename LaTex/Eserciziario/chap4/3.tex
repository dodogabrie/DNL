\begin{ex}[]
    \[
    \begin{dcases}
    \frac{\text{d} x}{\text{d} t} = \\
    \frac{\text{d} y}{\text{d} t} = x-x^3-\delta y + x^2y
    \end{dcases}
    \]
    In questo caso si ha:
    \[
        \frac{\partial F}{\partial x} + \frac{\partial G}{\partial y} = -\delta  + x^2
    .\] 
    \begin{itemize}
        \item Trovare gli stati stazioari e studiare la stabilità
	\item Si consideri $\delta >1$ e si discuta se possono esserci delle regioni di $\mathbb{R}^2$ con orbite chiuse.
    \end{itemize}
\end{ex}
\noindent
\begin{ex}[]
    Dato il sistema dinamico 
    \[
	\frac{\text{d}^2 x}{\text{d} t^2} + f(x) \frac{\text{d} x}{\text{d} t} + g(x) = 0
    .\] 
    Con $f$ e $g$ regolari. Quali condizioni su $f, g$ devono valere affinché non ci siano orbite chiuse.

\end{ex}
\noindent
\begin{ex}[]
    Prendiamo il campo vettoriale:
    \[
    \begin{dcases}
	\frac{\text{d} x}{\text{d} t} = -y + x(x^2 + y^2 -1) \\
	\frac{\text{d} y}{\text{d} t} = x + y (x^2 + y^2 -1)
    \end{dcases}
    \]
    Studiamo tale SD in $D$:
    \[
	D = \left\{(x, y) | x^2 + y^2 < \frac{1}{2}\right\}
    .\] 
    Dimostrare che in $D$ non possono esserci orbite chiuse.
\end{ex}
\noindent
