\begin{ex}[]
    Dimostrare che il sistema
    \[
    \begin{dcases}
    \frac{\text{d} x}{\text{d} t} = \sin y\\
    \frac{\text{d} y}{\text{d} t} = x\cos y
    \end{dcases}
    \]
    Non può avere orbite chiuse.
\end{ex}
\noindent
\begin{ex}[]
    Dimostrare il teorema "Minimo locale quindi asintoticamente stabile" utilizzando il teorema di Lyapunov. (introdurre una funzione di Lyapunov definita come $V(\v{x}) - V(\v{x}_s)$ nell'intorno in cui non vi sono stati stazionari) 
\end{ex}
\noindent
\begin{ex}[]
    Dato il campo
    \[
    \begin{dcases}
    \frac{\text{d} x}{\text{d} t} = y\\
    \frac{\text{d} y}{\text{d} t} = -x	
    \end{dcases}
    \v{V}= \begin{pmatrix} x \\  y\end{pmatrix}	
    \]
    Definiamo l'operatore di involuzione come:
    \[
	G(\v{V}) = G\begin{pmatrix} x \\ y \end{pmatrix} = \begin{pmatrix} x \\ -y \end{pmatrix} 
    .\] 
    \begin{itemize}
        \item Determinare $G$ e provare che è un operatore di involuzione.
	\item Verificare che 
	    \[
		\frac{\text{d} G(\v{V}) }{\text{d} t} = F(G(\v{V}) ) 
	    .\] 
    \end{itemize}
\end{ex}
\noindent
