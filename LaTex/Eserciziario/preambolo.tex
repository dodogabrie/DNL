\documentclass[a4paper]{report}
\usepackage{float}
\usepackage{verbatim}
\usepackage[utf8]{inputenc}
\usepackage[italian]{babel}
\usepackage{amsmath}
\usepackage{mathtools}
\usepackage{amsbsy,amssymb,amsfonts, amsthm, mhchem, multicol}
\usepackage{graphicx}
\usepackage[left=2cm,right=2cm,top=2cm,bottom=2cm]{geometry}
\usepackage{xcolor}
\usepackage[hypertexnames=false]{hyperref}
\usepackage{nameref}
\usepackage{framed}
\usepackage[framemethod=TikZ]{mdframed}

% figure support
\usepackage{import}
\usepackage{xifthen}
\pdfminorversion=7
\usepackage{pdfpages}
\usepackage{transparent}
\newcommand{\incfig}[1]{%
    \def\svgwidth{\columnwidth}
    \import{./figures/}{#1.pdf_tex}
}
\newcommand{\ffrac}[2]{\ensuremath{\frac{\displaystyle #1}{\displaystyle #2}}}
\newcommand{\angstrom}{\mbox{\normalfont\AA}}
\newcommand{\vect}[1]{\boldsymbol{#1}}
\renewcommand{\v}[1]{\boldsymbol{#1}}
\renewcommand{\[}{\begin{equation}}
\renewcommand{\]}{\end{equation}}
\renewcommand{\theequation}{\thesection.\arabic{equation}}
\counterwithin*{equation}{section}
\newcommand{\rom}[1]{\uppercase\expandafter{\romannumeral #1\relax}}
% title setup
\makeatother
\def\@lez{}%
\newcommand{\lez}[3]{
    \ifthenelse{\isempty{#3}}{%
	\def\@lez{Lezione #1}%
    }{%
	\def\@lez{Lezione #1: #3}%
    }%
    \section{\@lez}
    \marginpar{\small\textsf{\mbox{#2}}}
}
\makeatletter

\definecolor{mygreen}{rgb}{0.0, 0.4, 0.}
\newtheoremstyle{break}%
	{1em}{1em}%
	{}{}%
	{\bfseries}{}% % Note that final punctuation is omitted.
	{\newline}
	{#1 #2: \normalfont(\textcolor{mygreen}{#3})}

\newtheoremstyle{defn}
	{\topsep}   % ABOVESPACE
	{\topsep}   % BELOWSPACE
	{\itshape}  % BODYFONT
	{0pt}       % INDENT (empty value is the same as 0pt)
	{\bfseries} % HEADFONT
	{.}         % HEADPUNCT
	{5pt plus 1pt minus 1pt} % HEADSPACE
	{#1 #2: \normalfont(\textcolor{red}{#3})}          % CUSTOM-HEAD-SPEC

\newtheoremstyle{thm}% name of the style to be used
	{\topsep}% measure of space to leave above the theorem. E.g.: 3pt
	{\topsep}% measure of space to leave below the theorem. E.g.: 3pt
	{\itshape}% name of font to use in the body of the theorem
	{0pt}% measure of space to indent
	{\bfseries}% name of head font
	{.}% punctuation between head and body
	{ }% space after theorem head; " " = normal interword space
	{#1 #2: \normalfont(\textcolor{red}{\underline{#3}})}

\theoremstyle{thm}
\newtheorem{thm}{Teorema}[section] % reset theorem numbering for each chapter

\theoremstyle{defn}
\newtheorem{defn}{Definizione}[section] % definition numbers are dependent on theorem numbers

\theoremstyle{break}
\newtheorem{exmp}{Esempio}[section]
\newtheorem{ex}{Esercizio}[section] % definition numbers are dependent on theorem numbers
