\section{Sistemi lineari in dimensione $n$}%
\begin{ex}[Base di autovettori generalizzati]
Sia data 
\[
    A = 
    \begin{pmatrix} 
	6 & 2 & 1 \\
	- 7 & -3 & -1 \\
	-11 & -7 & 0 
    \end{pmatrix} 
.\] 	
Determinare una base di autovettori generalizzati di $A$.
\end{ex}
\noindent
\begin{ex}[Soluzione sistema in $\mathbb{R}^4$ (1)]
\[
    \frac{\text{d} \v{x}}{\text{d} t} = A \v{x}, \qquad \v{x}(0) = \v{x}_0 \in R^4, \quad A = 
    \begin{pmatrix} 
	0 & -2 & -1 & -1 \\
	1 & 2 & 1 & 1 \\
	0 & 1 & 1 & 0 \\
	0 & 0 & 0 & 1
    \end{pmatrix} 
.\] 	
\end{ex}
\noindent
 
\begin{ex}[Soluzione di sistema in $\mathbb{R}^4$ (2)]
Risolvere il seguente IVP:
\[
    \frac{\text{d} \v{x}}{\text{d} t} = A \v{x} \qquad A = 
    \begin{pmatrix}   
	0 & -1 & 0 & 0 \\
	1 & 0 & 0 & 0 \\
	0 & 0 & 0 & -1 	\\
	2 & 0 & 1 & 0 
    \end{pmatrix} 
.\] 
\end{ex}
\noindent

\begin{ex}[Autovettori generalizzati ed autovalori]
Dato l'IVP in $R^4$:
\[
    \frac{\text{d} \v{x}}{\text{d} t} = A \v{x}
.\] 
con 
\[
    A = 
    \begin{pmatrix}
	0 & -1 & 0 & 0\\
	1 & 0 & 0 & 0 \\
	0 & 0 & 0 & -1\\
	2 & 0 & 1 & 0
    \end{pmatrix} 
.\] 
Trovare gli autovettori generalizzati e gli autovalori.
\end{ex}
\noindent
