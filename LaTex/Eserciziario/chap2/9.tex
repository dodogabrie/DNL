\section{Teorema di Hartman-Grobman}%
\begin{ex}[Sul teorema di Hartman-Grobman]
    Dato il sistema dinamico 
     \[
        \frac{\text{d} \v{x}}{\text{d} t} = A \v{x} \quad  \v{x}\in \mathbb{R}^2 \quad  A = 
    \begin{pmatrix}
	-1 & -3 \\
	-3 & -1 \\
    \end{pmatrix}
    .\] 
    Sia $R:\mathbb{R}^2\to \mathbb{R}^2 $  tale che:
    \[
        \forall \v{v} = \begin{pmatrix} x \\ y \end{pmatrix} \in \mathbb{R}^2: \ \v{v}\to R\v{v}
    .\] 
    Con:
    \[
        R = \frac{1}{\sqrt{2} } 
    \begin{pmatrix}
	1 & -1 \\
	1 & 1 \\
    \end{pmatrix}
    .\]  
    Determinare come viene trasformato il SD attraverso $H$.
\end{ex}
\paragraph{Soluzione}%
Si definisce la quantità:
\[
    \v{y} = h(\v{x}) = R \v{x}
.\] 
L'azione della trasformazione sul SD può essere espressa mediante questo vettore:
\[
    \frac{\text{d} \v{y}}{\text{d} t} = \frac{\text{d} }{\text{d} t} (R \v{x}) = R\frac{\text{d} \v{x}}{\text{d} t} =
    R A \v{x} = R A R^{-1} R \v{x} \equiv A' \v{x}
.\] 
Invertendo la matrice si ottiene: 
\[
    R^{-1} = \frac{1}{\sqrt{2}}\frac{1}{2} 
    \begin{pmatrix}
        1  & 1 \\
       -1  & 1 \\
    \end{pmatrix}
    \quad\implies\quad 
    A' = 
    \begin{pmatrix}
        1 & 0 \\
        0 & -2 \\
    \end{pmatrix}
.\] 
Quindi si hanno gli autovalori $\lambda_1 = 1$, $\lambda_2 = -2$ per il sistema.
