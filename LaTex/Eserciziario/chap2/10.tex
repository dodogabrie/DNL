\section{Teorema di Lyapunov}%
\begin{ex}[Ripasso sul gradiente]
    Presa 
    \[
	z = S(x, y) \in C^1 \qquad S\ge 0
    .\] 
    E preso l'insieme $E$ tale che $c \in \mathbb{R}^+$: 
    \[
	E = \left\{(x, y) | S(x, y) = c\right\}
    .\] 
    Preso quindi il gradiente:
    $\nabla (S(x, y) -c) = (s_x, s_y) $ 
    DImostrare che se si prende $P = (x_0, y_0) \in E$ tale per cui:
    \[
        \left. \frac{\partial S}{\partial y} \right|_{x_0, y_0}
    .\] 
    Allora $(s_x, s_y)$ è ortogonale alla tangente in $P$.
\end{ex}
\noindent
\begin{ex}[Utilizzo del teorema di Lyapunov]
    Sia dato il sistema dinamico a tempo continuo 
    \[
        \frac{\text{d}^2 x}{\text{d} t} + \epsilon x^2 \frac{\text{d} x}{\text{d} t} + x = 0 
    .\] 
    Dimostrare che $\v{V}_s = \begin{pmatrix} 0 \\ 0 \end{pmatrix}$ è stabile secondo Lyapunov.
\end{ex}
\paragraph{Soluzione}%
Definiamo dapprima la quantità $y = \frac{\text{d} x}{\text{d} t}$ in modo da rendere il sistema un SD in $\mathbb{R}^2$ del primo ordine a tempo continuo.\\
Utilizziamo il teorema di Lyapunov con il seguente funzionale:
\[
    V(x, y): \mathbb{R}^2\to \mathbb{R}^2; \qquad V(x,y) = \frac{1}{2}x^2+\frac{1}{2}y^2
.\] 
Tale quantità rispetta le ipotesi del teorema, la derivata orbitale infatti vale:
\[
    \frac{\text{d} }{\text{d} t} V(\v{x}) = \nabla V \cdot \frac{\text{d} \v{x}}{\text{d} t} =
    (x,y) \cdot (y, -\epsilon x^2y - x) = -\epsilon x^2y^2 
.\] 
Si vede immediatamente che per $\epsilon >0$ il sistema è stabile secondo Lyapunov, viceversa tale stato stazionario è instabile.
\begin{ex}[Sul teorema di Krasovskii]
   Prendiamo le equazioni di Lorenz: 
   \[
   \begin{dcases}
       \frac{\text{d} x}{\text{d} t} = 6(y-x) \\
       \frac{\text{d} y}{\text{d} t} = r x - y - xz \\
       \frac{\text{d} z}{\text{d} t} = xy - bz
   \end{dcases}
   \]
   Con i parametri: $\sigma  >0, r > 0 , b > 0$ ( I parametri utilizzati da Lorenz per mostrare il caos deterministico sono $\sigma  = 10, r = 28, b = \frac{8}{3}$).
   \begin{enumerate}
       \item Trovare gli stati stazionari (ce ne sono 3).
       \item Determinare le proprietà di stabilità di questi stati stazionari.Trovare gli stati stazionari (ce ne sono 3).
       \item Determinare le proprietà di stabilità di $\v{V}_{s_1} = \v{0}$ al variare di $r$.
       \item Mostrare che per $r = 1$ $\v{V}_{s_1}$ è non iperbolico.
       \item Utilizzando la seguente funzione di Lyapunov:
	   \[
	       V(x, y, z) = \frac{1}{2}\left(\frac{x^2}{\delta} + y^2 + z^2\right)
	   .\] 
	 Mostrare che $\v{V}_{s_1}$ è stabile secondo Lyapunov.
   \end{enumerate}
\end{ex}
\noindent
\begin{ex}[]
    Dato il SD a tempo continuo:
    \[
    \begin{dcases}
    \frac{\text{d} x_1}{\text{d} t} = -x_2^3\\
    \frac{\text{d} x_2}{\text{d} t} = x_1^3
    \end{dcases}
    \]
    \begin{enumerate}
        \item Determinare gli stati stazionari.
	\item Determinare le proprietà di stabilità degli stati stazionari.
	\item Dimostrare, utilizzando il teorema di Lyapunov che l'origine è stabile secondo Lyapunov.
	\item Determinare la superficie dove giacciono tutte le orbite del sistema dinamico.
    \end{enumerate}
\end{ex}
\noindent
