\section{Soluzioni stazionarie}%
\begin{ex}[Stati Stazioari]
    Trovare gli stati stazionari dei seguenti SD a tempo continuo autonomi:
    \begin{itemize}
    \item 1) \[
            \frac{\text{d} ^2x}{\text{d} t^2} - \epsilon x\frac{\text{d} x}{\text{d} t}  + x = 0
        \] 
    \item  2)
	\[\begin{dcases}
        \frac{\text{d} x}{\text{d} t} = - x + x^3\\
	\frac{\text{d} y}{\text{d} t} = x + y
        \end{dcases}\] 
    \item 3)
        \[\begin{dcases}
        \frac{\text{d} x}{\text{d} t} = y \\
	\frac{\text{d} y}{\text{d} t} =- y - \mu x - x^2
        \end{dcases}\] 
    \end{itemize}
\end{ex}
\noindent
\begin{ex}[Punto fisso della mappa logistica]
    Dimostrare che per $0\le \mu\le 1$ esiste solo uno stato stazionario.\\
     Suggerimento: utilizzare l'espressione 
    \[
        \frac{\text{d} y}{\text{d} t} = \mu-2\mu x
    \] con $y = \mu x(1-x)$ e fare uso della geometria analitica.
\end{ex}
\noindent
\begin{ex}[Punti stazionari di Mappe ricorsive]
    Determinare gli stati stazionari delle seguenti mappe ricorsive:
    \begin{enumerate}
        \item 
	    \[\begin{dcases}
	        x_{k+1}=x_k\\
		y_{k+1}=x_k + y_k
	    \end{dcases}\] 
	\item
	    \[\begin{dcases}
	        x_{k+1}=x_k^2\\
		y_{k+1}=x_k + y_k
	    \end{dcases}\] 
    \end{enumerate}
\end{ex}
\noindent
