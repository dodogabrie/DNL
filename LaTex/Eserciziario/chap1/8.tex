\section{Campi vettoriali}%
\begin{ex}[Su campo vettoriale]
    Preso il seguente campo vettoriale:
    \[
	\frac{\text{d} x}{\text{d} t} = - (1+x^2)
    .\] 
    e sia $x(t_0)=x_0$.
    \begin{itemize}
        \item Verificare che una soluzione è:
	    \[
		x(t)=- \tan(t-t_0-\arctan (x_0))
	    .\] 
	\item Verificare che $x(t+\tau)$ è ancora soluzione.
    \end{itemize}
\end{ex}
\noindent
\begin{ex}[Teorema di Shift e sistemi non autonomi 1]
    Preso il sistema
    \[
	\frac{\text{d} x}{\text{d} t} = e^t; \qquad  x(0)=x_0
    .\] 
    Dimostrare che la soluzione è:
    \[
	x(t)=e^t-1+x_0
    .\] 
    e verificare che il teorema di invarianza per shift non è verificato.
\end{ex}
\noindent
\begin{ex}[Teorema di Shift e sistemi non autonomi 2]
   Dato il sistema 
   \[
       \frac{\text{d} \vect{x}}{\text{d} t} = F(\vect{x}, t); \qquad \text{Soluzione: }\vect{x}_s(t)
   .\] 
   Verificare che, posti $\vect{x}_{\tau}(t)$ e $F_{\tau}$:
   \[
       \vect{x}_{\tau}(t)=\vect{x}_s(t+\tau); \qquad F_{\tau}(\vect{x}_{\tau}, t) = F(\vect{x}_\tau, t+\tau)
   .\] 
   Allora si ha che $\vect{x}_s(t+\tau)$ è soluzione di:
   \[
       \frac{\text{d} \vect{x}_\tau}{\text{d} t} = F_\tau (\vect{x}_\tau, t)
   .\] 
   In pratica quindi lo shift temporale per un sistema non autonomo richiede di traslare anche il funzionale $F$. 
\end{ex}
\noindent
\begin{ex}[Esercizi sul teorema]
    Determinare i campi vettoriali associati ai seguenti flussi:
    \begin{itemize}
	\item $\varphi (t, x) = \frac{x e^{t}}{xe^t - x + 1}$.
	\item $\varphi (t, x) = \frac{x}{(1-2x^2t)^{1 /2}}$.
	\item $\varphi (t, x, y) = (xe^t, \frac{y}{1-yt})$.
    \end{itemize}
\end{ex}
\noindent
